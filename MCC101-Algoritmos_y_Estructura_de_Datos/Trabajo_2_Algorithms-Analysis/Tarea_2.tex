\documentclass{article}
\usepackage{graphicx}
\usepackage[utf8]{inputenc}
\usepackage{fullpage}
\usepackage{titling}
\usepackage{enumitem}
\usepackage[utf8]{inputenc}


\parindent0in
\pagestyle{plain}
\thispagestyle{plain}

\newcommand{\assignment}{Práctica 2}
\newcommand{\duedate}{1 de Julio, 2018}

\renewcommand\thesubsection{\arabic{subsection}}

\title{Análisis Asintótico}
\date{}

\begin{document}

Universidad Católica San Pablo\hfill\\
Algoritmos y Estructura de Datos\hfill\textbf{\assignment}\\
Prof.\ Jorge Poco\hfill\textbf{Entrega:} \duedate\\
Alumno: Moreno Vera, Felipe Adrian
\smallskip\hrule\bigskip

{\let\newpage\relax\maketitle}
% \maketitle
\section{Análisis Asintótico}
\begin{enumerate}[label=\textbf{\alph*.}]
  \item Ordene las siguientes funciones en orden $O()$ creciente, indicando los grupos que tienen el mismo orden: $\sqrt{n}$, $n\log n$, $n^2$, $n^{1/3}+\log n$, $\log n$, $(1/3)^n$, $n$, $n-n^3+7n^5$, $n^3$, $(\log n)^2$, $n/\log n$, $(3/2)^n$, $2^n$, $n^2+\log n$, $\log n$, $\log \log n$, $6$.

  \textbf{Solución:}
     
  Ordenando en orden creciente las funciones:\\
  $(\frac{1}{3})^n$, $log(log(n))$, $6$, $log(n)$, $n^{1/3}+\log n$, $(\log n)^2$, $\sqrt{n}$, $n/\log n$, $n$, $nlog(n)$, $n^2$, $n^2+\log n$, $n^3$, $n-n^3+7n^5$, $(3/2)^n$ $2^n$.
  
  Ordenando en función de complejidad:\\
  \textbf{$O(1)$}: $6$, \\
  \textbf{$O(\log \log n)$}: $\log \log n$ \\
  \textbf{$O(n^{1/3})$}: $n^{1/3}+\log n$ \\
  \textbf{$O(\log n)$}: $\log n$ \\
  \textbf{$O(\log^2 n)$}: $(\log n)^2$ \\
  \textbf{$O(n^{1/2})$}: $\sqrt{n}$ \\
  \textbf{$O(n/logn)$}: $n/logn$ \\
  \textbf{$O(n)$}: $n$ \\
  \textbf{$O(n\log n)$}: $n\log n$\\
  \textbf{$O(n^2)$}: $n^2$, $n^2+\log n$ \\
  \textbf{$O(n^3)$}: $n^3$ \\
  \textbf{$O(n^5)$}: $n-n^3+7n^5$ \\
  \textbf{$O(c^n)$}: $(1/3)^n$, $(3/2)^n$, $2^n$ \\

  \item Haga los mismo con las siguientes funciones: $\log n$, $n^2(1+\sqrt{n})$, $n^{3/2}$, $n^2/\log n$, $n^2$, $n/\log n$, $n^{1/3}$, $1$, $1/n$, $5^n$, $n^{1.00001}$, $n$, $\log \log n$, $n^n$, $(\log n)^2$, $n^{n^2}$ , $2^n$, $(\log n)^n$, $n^{\log n}$.
  
  \textbf{Solución:}
     
  Ordenando en orden creciente las funciones:\\
  $1/n$, $1$, $\log \log n$, $\log n$, $n^{1/3}$, $(\log n)^2$, $n/\log n$, $n$, $n^{1.00001}$, $n^{3/2}$, $n^2/\log n$, $n^2$, $n^2(1+\sqrt{n})$, $n^{\log n}$, $2^n$, $5^n$, $(\log n)^n$, $n^n$, $n^{n^2}$.
  
  Ordenando en función de complejidad:\\
  \textbf{$O(1/n)$}: $1/n$ \\
  \textbf{$O(1)$}: $1$ \\
  \textbf{$O(\log \log n)$}: $\log \log n$ \\
  \textbf{$O(\log n)$}: $\log n$ \\
  \textbf{$O(n^{1/3})$}: $n^{1/3}$ \\
  \textbf{$O(\log^2 n)$}: $(\log n)^2$ \\
  \textbf{$O(n/\log n)$}: $n/\log n$ \\
  \textbf{$O(n)$}: $n$ \\
  \textbf{$O(n^{1.00001})$}: $n^{1.00001}$ \\
  \textbf{$O(n^{3/2})$}: $n^{3/2}$ \\
  \textbf{$O(n^2/\log n)$}: $n^2/\log n$ \\
  \textbf{$O(n^2)$}: $n^2$ \\
  \textbf{$O(n^{5/2})$}: $n^2(1+\sqrt{n})$ \\
  \textbf{$O(n^{\log n})$}: $n^{\log n}$ \\
  \textbf{$O(c^n)$}: $2^n$, $5^n$ \\
  \textbf{$O((logn)^n)$}: $(logn)^n$ \\
  \textbf{$O(n^n)$}: $n^n$ \\
  \textbf{$O(n^{n^2})$}: $n^{n^2}$ \\
  
  \item De los siguientes pares $f$ y $g$, determine si $f$ es $O(g)$, $\Omega(g)$, $o(g)$, $\Theta(g)$:
  $n^2 +3n+4$ vs $6n+7$, $\sqrt{n}$ vs $\log(n+3)$, $n\sqrt{n}$ vs $n^2 -n$, $n\sqrt{n}$ vs $4n \log(n^2 +1)$, $(n^2 +2)/(1+2-n)$ vs $n+3$, $2^n -n^2$ vs $n^4 +n^2$, $n\log n$ vs $(\log n)^{\log n}$, $n2^n$ vs $3^n$ , $100n+\log n$ vs $n+(\log n)^2$ , $\log n$ vs $\log \log n^2$ , $(\log n)^{10^6}$ vs $n^{10^{-6}}$
  
  \textbf{Solución:}
  
  \textbf{1.} $n^2 +3n+4$ vs $6n+7$:
  \begin{itemize}
      \item O(g):
      
	  $\lim_{x\to\infty} \frac{g(n)}{f(n)}>0$\\
	  $\lim_{x\to\infty} \frac{6n+7}{n^2 +3n+4}>0$\\
      $0>0$, es una contradicción, por tanto $n^2 +3n+4$ no es $O(6n+7)$.
    
      \item $\Omega(g)$
      
      $\lim_{x\to\infty} \frac{g(n)}{f(n)}=c \geq 0$\\
      $\lim_{x\to\infty} \frac{6n+7}{n^2 +3n+4}=c \geq 0$\\
      $c=0 \geq 0$, es verdadero, por tanto $n^2 +3n+4$ si es $\Omega(6n+7)$.
    
      \item $o(g)$\\
      $\lim_{x\to\infty} \frac{g(n)}{f(n)}=\infty$\\
      $\lim_{x\to\infty} \frac{6n+7}{n^2 +3n+4}=\infty$\\
      $0=\infty$, es una contradicción, por tanto $n^2 +3n+4$ no es $o(6n+7)$.
    
      \item $\Theta(g)$\\
      $\lim_{x\to\infty} \frac{g(n)}{f(n)}=c,0<c<\infty$\\
      $\lim_{x\to\infty} \frac{6n+7}{n^2 +3n+4}=c,0<c<\infty$\\
      $0=c$\\
	  $0<0<\infty$ es una contradicción, por tanto $n^2 +3n+4$ no es $\Theta(6n+7)$.\\
    
  \end{itemize}
  
  \textbf{2.} $\sqrt{n}$ vs $\log(n+3)$:
\begin{itemize}
      \item O(g):
      
      $\lim_{x\to\infty} \frac{g(n)}{f(n)}>0$\\
      $\lim_{x\to\infty} \frac{\log(n+3)}{\sqrt{n}}>0$\\
      $0>0$, es una contradicción, por tanto $\sqrt{n}$ no es $O(\log(n+3))$.
    
      \item $\Omega(g)$:
      
      $\lim_{x\to\infty} \frac{g(n)}{f(n)}=c \geq 0$\\
      $\lim_{x\to\infty} \frac{\log(n+3)}{\sqrt{n}}=c \geq 0$\\
      $0=c \geq 0$, es verdadero, por tanto $\sqrt{n}$ si es $\Omega(\log(n+3))$.
    
      \item $o(g)$:
      
      $\lim_{x\to\infty} \frac{g(n)}{f(n)}=\infty$\\
      $\lim_{x\to\infty} \frac{\log(n+3)}{\sqrt{n}}=\infty$\\
      $0=\infty$, es una contradicción, por tanto $\sqrt{n}$ no es $o(\log(n+3))$.
    
      \item $\Theta(g)$:
      
      $\lim_{x\to\infty} \frac{g(n)}{f(n)}=c,0<c<\infty$\\
      $\lim_{x\to\infty} \frac{\log(n+3)}{\sqrt{n}}=c,0<c<\infty$\\
      $0=c$\\
	  $0<0<\infty$ es una contradicción, por tanto $\sqrt{n}$ no es $\Theta(\log(n+3))$.
    
  \end{itemize}  
  \textbf{3.} $n\sqrt{n}$ vs $n^2 -n$:
    \begin{itemize}
      \item O(g):
      
      $\lim_{x\to\infty} \frac{g(n)}{f(n)}>0$\\
      $\lim_{x\to\infty} \frac{n^2 -n}{n\sqrt{n}}>0$\\
      $\infty>0$ es verdadero, por tanto $n\sqrt{n}$ si es $O(n^2 -n)$.
    
      \item $\Omega(g)$:
      
      $\lim_{x\to\infty} \frac{g(n)}{f(n)}=c \geq 0$\\
      $\lim_{x\to\infty} \frac{n^2 -n}{n\sqrt{n}}=c \geq 0$\\
      $\infty \geq 0$ es verdadero, por tanto $n\sqrt{n}$ si es $\Omega(n^2 -n)$.
      
      \item $o(g)$:
      
      $\lim_{x\to\infty} \frac{g(n)}{f(n)}=\infty$\\
      $\lim_{x\to\infty} \frac{n^2 -n}{n\sqrt{n}}=\infty$\\
      $\infty=\infty$ es verdadero, por tanto $n\sqrt{n}$ si es $o(n^2 -n)$.
      
      \item $\Theta(g)$:
      
      $\lim_{x\to\infty} \frac{g(n)}{f(n)}=c,0<c<\infty$\\
      $\lim_{x\to\infty} \frac{n^2 -n}{n\sqrt{n}}=c,0<c<\infty$\\
      $\infty=c$\\
      $0<\infty<\infty$ es una contradicción, por tanto $n\sqrt{n}$ no es $\Theta(n^2 -n)$.
  \end{itemize}
  
  \textbf{4.} $n\sqrt{n}$ vs $4n \log(n^2 +1)$:
    \begin{itemize}
      \item O(g):
      
        $\lim_{x\to\infty} \frac{g(n)}{f(n)}>0$\\
        $\lim_{x\to\infty} \frac{4n \log(n^2 +1)}{n\sqrt{n}}>0$\\
        $0>0$, es una contradicción, por tanto $n\sqrt{n}$ no es $O(4n \log(n^2 +1))$.

      \item $\Omega(g)$:
      
        $\lim_{x\to\infty} \frac{g(n)}{f(n)}=c \geq 0$\\
        $\lim_{x\to\infty} \frac{4n \log(n^2 +1)}{n\sqrt{n}}=c \geq 0$\\
        $c=0 \geq 0$, es verdadero, por tanto $n\sqrt{n}$ si es $\Omega(4n \log(n^2 +1))$.

      \item $o(g)$:
      
        $\lim_{x\to\infty} \frac{g(n)}{f(n)}=\infty$\\
        $\lim_{x\to\infty} \frac{4n \log(n^2 +1)}{n\sqrt{n}}=\infty$\\
        $0=\infty$, es una contradicción, por tanto $n\sqrt{n}$ no es $o(4n \log(n^2 +1))$.

      \item $\Theta(g)$:
      
        $\lim_{x\to\infty} \frac{g(n)}{f(n)}=c,0<c<\infty$\\
        $\\lim_{x\to\infty} \frac{4n \log(n^2 +1)}{n\sqrt{n}}=c,0<c<\infty$\\
        $0=c$\\
        $0<0<\infty$ es una contradicción, por tanto $n\sqrt{n}$ no es $\Theta(4n \log(n^2 +1))$.
    \end{itemize}
  
  \textbf{5.} $(n^2 +2)/(1+2-n)$ vs $n+3$:
    \begin{itemize}
      \item O(g):
      
        $\lim_{x\to\infty} \frac{g(n)}{f(n)}>0$\\
        $\lim_{x\to\infty} \frac{n+3}{(n^2 +2)/(1+2-n)}>0$\\
        $\infty>0$, es una verdadero, por tanto $(n^2 +2)/(1+2-n)$ si es $O(n+3)$.

      \item $\Omega(g)$:
      
        $\lim_{x\to\infty} \frac{g(n)}{f(n)}=c \geq 0$\\
        $\lim_{x\to\infty} \frac{n+3}{(n^2 +2)/(1+2-n)}=c \geq 0$\\
        $c=\infty \geq 0$, es verdadero, por tanto $(n^2 +2)/(1+2-n)$ si es $\Omega(n+3)$.

      \item o(g):
      
        $\lim_{x\to\infty} \frac{g(n)}{f(n)}=\infty$\\
        $\lim_{x\to\infty} \frac{n+3}{(n^2 +2)/(1+2-n)}=\infty$\\
        $\infty=\infty$, es verdad, por tanto $(n^2 +2)/(1+2-n)$ si es $o(n+3)$.

      \item $\Theta(g)$:
      
        $\lim_{x\to\infty} \frac{g(n)}{f(n)}=c,0<c<\infty$\\
        $\\lim_{x\to\infty} \frac{n+3}{(n^2 +2)/(1+2-n)}=c,0<c<\infty$\\
        $\infty=c$\\
	$0<\infty<\infty$ es una contradicción, por tanto $(n^2 +2)/(1+2-n)$ no es $\Theta(n+3)$.
  \end{itemize}
  
  \textbf{6.} $2^n -n^2$ vs $n^4 +n^2$:    \begin{itemize}
      \item O(g):
      
        $\lim_{x\to\infty} \frac{g(n)}{f(n)}>0$\\
        $\lim_{x\to\infty} \frac{n^4 +n^2}{2^n -n^2}>0$\\
        $\infty>0$, es una verdadero, por tanto $2^n -n^2$ si es $O(n^4 +n^2)$.

      \item $\Omega(g)$:
      
        $\lim_{x\to\infty} \frac{g(n)}{f(n)}=c \geq 0$\\
        $\lim_{x\to\infty} \frac{n^4 +n^2}{2^n -n^2}=c \geq 0$\\
        $c=\infty \geq 0$, es verdadero, por tanto $(2^n -n^2$ si es $\Omega(n^4 +n^2)$.

      \item o(g):
      
        $\lim_{x\to\infty} \frac{g(n)}{f(n)}=\infty$\\
        $\lim_{x\to\infty} \frac{n^4 +n^2}{2^n -n^2}=\infty$\\
        $\infty=\infty$, es verdad, por tanto $2^n -n^2$ si es $o(n^4 +n^2)$.

      \item $\Theta(g)$:
      
        $\lim_{x\to\infty} \frac{g(n)}{f(n)}=c,0<c<\infty$\\
        $\\lim_{x\to\infty} \frac{n^4 +n^2}{2^n -n^2}=c,0<c<\infty$\\
        $c=\infty$\\
	$0<\infty<\infty$ es una contradicción, por tanto $2^n -n^2$ no es $\Theta(n^4 +n^2)$.
    \end{itemize}

  
  \textbf{7.} $n\log n$ vs $(\log n)^{\log n}$:
    \begin{itemize}
      \item O(g):
      
        $\lim_{x\to\infty} \frac{g(n)}{f(n)}>0$\\
        $\lim_{x\to\infty} \frac{(\log n)^{\log n}}{n\log n}>0$\\
        $\infty>0$, es una verdadero, por tanto $n\log n$ si es $O((\log n)^{\log n})$.

      \item $\Omega(g)$:
      
        $\lim_{x\to\infty} \frac{g(n)}{f(n)}=c \geq 0$\\
        $\lim_{x\to\infty} \frac{(\log n)^{\log n}}{n\log n}=c \geq 0$\\
        $c=\infty \geq 0$, es verdadero, por tanto $n\log n$ si es $\Omega((\log n)^{\log n})$.

      \item o(g):
      
        $\lim_{x\to\infty} \frac{g(n)}{f(n)}=\infty$\\
        $\lim_{x\to\infty} \frac{(\log n)^{\log n}}{n\log n}=\infty$\\
        $\infty=\infty$, es verdad, por tanto $n\log n$ si es $o((\log n)^{\log n})$.

      \item $\Theta(g)$:
      
        $\lim_{x\to\infty} \frac{g(n)}{f(n)}=c,0<c<\infty$\\
        $\\lim_{x\to\infty} \frac{(\log n)^{\log n}}{n\log n}=c,0<c<\infty$\\
        $c=\infty$\\
	$0<\infty<\infty$ es una contradicción, por tanto $n\log n$ no es $\Theta((\log n)^{\log n})$.
    \end{itemize}

  
  \textbf{8.} $n2^n$ vs $3^n$:    \begin{itemize}
      \item O(g):
      
        $\lim_{x\to\infty} \frac{g(n)}{f(n)}>0$\\
        $\lim_{x\to\infty} \frac{3^n}{n2^n}>0$\\
        $\infty>0$, es una verdadero, por tanto $n2^n$ si es $O(3^n)$.

      \item $\Omega(g)$:
      
        $\lim_{x\to\infty} \frac{g(n)}{f(n)}=c \geq 0$\\
        $\lim_{x\to\infty} \frac{3^n}{n2^n}=c \geq 0$\\
        $c=\infty \geq 0$\\
        $\infty \geq 0$, es verdadero, por tanto $n2^n$ si es $\Omega(3^n)$.

      \item o(g):
      
        $\lim_{x\to\infty} \frac{g(n)}{f(n)}=\infty$\\
        $\lim_{x\to\infty} \frac{3^n}{n2^n}=\infty$\\
        $\infty=\infty$, es verdad, por tanto $n2^n$ si es $o(3^n)$.

      \item $\Theta(g)$:
      
        $\lim_{x\to\infty} \frac{g(n)}{f(n)}=c,0<c<\infty$\\
        $\\lim_{x\to\infty} \frac{3^n}{n2^n}=c,0<c<\infty$\\
        $c=\infty$\\
	$0<\infty<\infty$ es una contradicción, por tanto $n2^n$ no es $\Theta(3^n)$.
    \end{itemize}

  
  \textbf{9.}  $100n+\log n$ vs $n+(\log n)^2$:
    \begin{itemize}
      \item O(g):
      
        $\lim_{x\to\infty} \frac{g(n)}{f(n)}>0$\\
        $\lim_{x\to\infty} \frac{n+(\log n)^2}{100n+\log n}>0$\\
        $1/100>0$, es una verdadero, por tanto $100n+\log n$ si es $O(n+(\log n)^2)$.

      \item $\Omega(g)$:
      
        $\lim_{x\to\infty} \frac{g(n)}{f(n)}=c \geq 0$\\
        $\lim_{x\to\infty} \frac{n+(\log n)^2}{100n+\log n}=c \geq 0$\\
        $c=1/100 \geq 0$, es verdadero, por tanto $100n+\log n$ si es $\Omega(n+(\log n)^2)$.

      \item o(g):
      
        $\lim_{x\to\infty} \frac{g(n)}{f(n)}=\infty$\\
        $\lim_{x\to\infty} \frac{n+(\log n)^2}{100n+\log n}=\infty$\\
        $1/100=\infty$, es una contradicción, por tanto $100n+\log n$ si es $o(n+(\log n)^2)$.

      \item $\Theta(g)$:
      
        $\lim_{x\to\infty} \frac{g(n)}{f(n)}=c,0<c<\infty$\\
        $\\lim_{x\to\infty} \frac{n+(\log n)^2}{100n+\log n}=c,0<c<\infty$\\
        $c=1/100$\\
        $0<1/100<\infty$ es verdadero, por tanto $100n+\log n$ si es $\Theta(n+(\log n)^2)$.
    \end{itemize}
  
  \textbf{10.} $\log n$ vs $\log \log n^2$:
    \begin{itemize}
      \item O(g):
      
        $\lim_{x\to\infty} \frac{g(n)}{f(n)}>0$\\
        $\lim_{x\to\infty} \frac{\log \log n^2}{\log n}>0$\\
        $0>0$, es una contradicción, por tanto $\log n$ no es $O(\log \log n^2)$.
      \item $\Omega(g)$:
      
        $\lim_{x\to\infty} \frac{g(n)}{f(n)}=c \geq 0$\\
        $\lim_{x\to\infty} \frac{\log \log n^2}{\log n}=c \geq 0$\\
        $c=0 \geq 0$ es verdadero, por tanto $\log n$ si es $\Omega(\log \log n^2)$.
      \item o(g):
      
        $\lim_{x\to\infty} \frac{g(n)}{f(n)}=\infty$\\
        $\lim_{x\to\infty} \frac{\log \log n^2}{\log n}=\infty$\\
        $0=\infty$, es una contradicción, por tanto $\log n$ no es $o(\log \log n^2)$.
      \item $\Theta(g)$:
      
        $\lim_{x\to\infty} \frac{g(n)}{f(n)}=c,0<c<\infty$\\
        $\lim_{x\to\infty} \frac{\log \log n^2}{\log n}=c,0<c<\infty$\\
        $c=0$\\
	$0<0<\infty$ es una contradicción, por tanto $\log n$ no es $\Theta(\log \log n^2)$.
    \end{itemize}
  
  \textbf{11.} $(\log n)^{10^6}$ vs $n^{10^{-6}}$:
    \begin{itemize}
      \item O(g):
      
        $\lim_{x\to\infty} \frac{g(n)}{f(n)}>0$\\
        $\lim_{x\to\infty} \frac{n^{10^{-6}}}{(\log n)^{10^6}}>0$\\
        $\infty>0$, es una verdadero, por tanto $(\log n)^{10^6}$ si es $O(n^{10^{-6}})$.

      \item $\Omega(g)$:
      
        $\lim_{x\to\infty} \frac{g(n)}{f(n)}=c \geq 0$\\
        $\lim_{x\to\infty} \frac{n^{10^{-6}}}{(\log n)^{10^6}}=c \geq 0$\\
        $c=\infty \geq 0$, es verdadero, por tanto $(\log n)^{10^6}$ si es $\Omega(n^{10^{-6}})$.

      \item o(g):
      
        $\lim_{x\to\infty} \frac{g(n)}{f(n)}=\infty$\\
        $\lim_{x\to\infty} \frac{n^{10^{-6}}}{(\log n)^{10^6}}=\infty$\\
        $\infty=\infty$, es verdad, por tanto $(\log n)^{10^6}$ si es $o(n^{10^{-6}})$.

      \item $\Theta(g)$:
      
        $\lim_{x\to\infty} \frac{g(n)}{f(n)}=c,0<c<\infty$\\
        $\\lim_{x\to\infty} \frac{n^{10^{-6}}}{(\log n)^{10^6}}=c,0<c<\infty$\\
        $c=\infty$\\
	$0<\infty<\infty$ es una contradicción, por tanto $(\log n)^{10^6}$ no es $\Theta(n^{10^{-6}})$.
    \end{itemize}
  
  
  \item Demostrar que $n^2$ no es $O(n)$
  
  \textbf{Solución:}
  
  Sea: \\
  $f(n) = n^2,  g(n) = n$\\
  Entonces, f(n) = O(g(n)) si:\\
  Existe $n_0, c > 0$, Tal que:\\
  $\forall n \geq n_0 > 0 : f(n) \leq cg(n)$\\
  
  Por definición, se tiene que $n > 0$, entonces:\\
  $n^2=f(n) > 0$ y $n=g(n) > 0$\\
  De donde:\\
  $n_0 > 0$\\
  Entonces, en la ecuación:\\
  $0 \leq n^2 \leq cn$\\
  Vemos que no existe constante c que satisfaga la ecuación. 
  
  \item Demostrar que $7n - 2 = O(n)$
  
  \textbf{Solución:}
  
  Sea: \\
  $f(n) = 7n-2,  g(n) = n$\\
  Por definición, se tiene que $n > 0$, entonces:\\
  $7n-2=f(n) \geq 0$ y $n=g(n) \geq 0$\\
  De donde:\\
  $ n_0 \geq \frac{2}{7}$\\
  Entonces, en la ecuación:\\
  $0 \leq 7n-2 \leq cn$\\
  Basta tomar c=8 para que satisfaga la ecuación.
  
  \item Demostrar que $20n^3 + 10 \log n + 5$ es $O(n^3)$
  
  \textbf{Solución:}
  
  Sea: \\
  $f(n) = 20n^3+10logn + 5,  g(n) = n^3$\\
  Por definición, se tiene que $n > 0$, entonces:\\
  $20n^3+10logn + 5=f(n) \geq 0$ y $n^3=g(n) \geq 0$\\
  De donde:\\
  $ n_0 \geq 1$\\
  Entonces, en la ecuación:\\
  $0 \leq 20n^3+10logn + 5 \leq cn^3$\\
  Basta tomar c=25 para que satisfaga la ecuación.
  
  \item Demostrar que $3 \log n + \log \log n$ es $O(\log n)$
  
  \textbf{Solución:}
  
  Sea: \\
  $f(n) = 3logn + loglogn,  g(n) = logn$\\
  Por definición, se tiene que $n > 0$, entonces:\\
  $3logn + loglogn=f(n) \geq 0$ y $logn=g(n) \geq 0$\\
  De donde:\\
  $ n_0 \geq 2$\\
  Entonces, en la ecuación:\\
  $0 \leq 3logn + loglogn \leq c(logn)$\\
  Basta tomar c=4 para que satisfaga la ecuación.
  
  \item Demostrar que $3 \log n + \log \log n$ es $\Omega (\log n)$
  
  \textbf{Solución:}

  Sea: \\
  $f(n) = 3logn + loglogn,  g(n) = logn$\\
    Entonces, f(n) = $\Omega (g(n))$ si:\\
  Existe $n_0, c > 0$, Tal que:\\
  $\forall n \geq n_0 > 0 : cg(n) \leq f(n)$\\
  
  Por definición, se tiene que $n > 0$, entonces:\\
  $3logn + loglogn=f(n) \geq 0$ y $logn=g(n) \geq 0$\\
  De donde:\\
  $ n_0 \geq 2$\\
  Entonces, en la ecuación:\\
  $0 \leq c(logn) \leq 3logn + loglogn$\\
  Basta tomar c=3 para que satisfaga la ecuación.
  
  \item Demostrar que $3 \log n + \log \log n$ es $\Theta (\log n)$
  
  \textbf{Solución:}

  Sea: \\
  $f(n) = 3logn + loglogn,  g(n) = logn$\\
    Entonces, f(n) = $\Theta (g(n))$ si:\\
  Existe $n_0, c > 0$, Tal que:\\
  $\forall n \geq n_0 > 0 : c_1 g(n) \leq f(n) \leq c_2 g(n)$\\
  
  Por definición, se tiene que $n > 0$, entonces:\\
  $3logn + loglogn=f(n) \geq 0$ y $logn=g(n) \geq 0$\\
  De donde:\\
  $ n_0 \geq 2$\\
  Entonces, en la ecuación:\\
  $ 0 \leq c_1 (logn) \leq 3logn + loglogn \leq  c_2 (logn) $\\
  De los ejercicios \textbf{g} y \textbf{h}. Basta tomar $c_1=3$ y $c_2=4$ para que satisfaga la ecuación.


  % \item Demostrar que la función $f(n)= 12 n^2 + 6n$ es $o(n^3)$ y es $\omega(n)$
\end{enumerate}


\section{Recurrencias}

Resuelva las siguientes recurrencias usando funciones generatrices y cualquier otro método, es decir, cada recurrencia debe ser resuelto por dos métodos. 
La solución debe ser exacta para infinitos $n$.

\begin{enumerate}[label=\textbf{\alph*.}]
  \item $T(n)=T(n-1)+n-1, T(1)=2$
  
  \textbf{Soluci\'on Por Funci\'on generatriz de probabilidad (basada en la tranformada Z)}
  
  Transformamos T(n) en A(z) y aumentamos el caso base a 0:\\
  $T(n+1)=T(n) + n$\\
  $T(1) = T(0) + 0$, Entonces, $T(0)=2$\\
  Reemplazando por las funciones generatrices:\\
  $\frac{A(z) - A(0)}{z} = A(z) + \frac{z}{(1-z)^2}$\\
  $(1-z)A(z) - \frac{z^2}{(1-z)^2} - 2=0$\\
  $A(z) = \frac{z^2}{(1-z)^3} + \frac{2}{(1-z)}$\\
  Haciendo transformada inversa:\\
  $T(n) = {n \choose 2} + 2$\\
  
  Soluci\'on encontrada por el m\'etodo de función generatriz:
  \begin{center}
    $T(n) = \frac{(n-1)n}{2}$+2
  \end{center}
    
  \textbf{Soluci\'on Por Ecuaci\'on Caracter\'istica}
  
  Se debe generar un polinomio caracter\'istico basado en la recursi\'on.\\
  $T(n+1)=T(n) + n$\\
  
  \textbf{C\'aculo de la soluci\'on homog\'enea:}\\
  Sea $ T(n)_H = r^n $, cambiamos la recursi\'on por:\\
  $T(n+1)_H - T(n)_H = 0$\\
  $r^{n+1} - r^{n} = 0$\\
  Dividimos entre $r^{n}$ y obtenemos la ra\'iz $(r-1)$\\
  Por lo cual la soluci\'on homog\'enea es:\\
  $T(n)_H = \alpha (1)^n$
  
  \textbf{C\'aculo de la soluci\'on particular:}\\
  Sabemos que $T(n)_P = A_2n^2 + A_1n + A_0$, por lo que el caso particular ser\'a:\\
  Entonces hacemos:\\
  $T(n+1)_P=T(n)_P + n$\\
  $A_2(n+1)^2+A_1(n+1)+A_0 = A_2 n^2 + A_1n + A_0 + n$\\
  Por lo tanto $A_2 =\frac{1}{2}, A_1=-\frac{1}{2}, A_0=0$, se tiene la soluci\'on particular:\\
  $T(n)_P =\frac{1}{2} n^2-\frac{1}{2}n = \frac{(n-1)n}{2}$
  
  \textbf{C\'aculo de la soluci\'on general:}\\
  Tenemos:\\
  $S_G = S_H + S_P$\\
  $T(n)_G = \alpha (1)^n + \frac{(n-1)n}{2}$\\
  Usando el caso base T(1) = 2:\\
  $T(1) = \alpha (1)^1 + \frac{(1-1)1}{2}$\\
  $2 = \alpha $\\
  Por lo tanto, la funci\'on recursiva es:\\
  $T(n) = 2 + \frac{(n-1)n}{2}$\\
  
  Soluci\'on encontrada por el m\'etodo de ecuaci\'on caracter\'istica:
  \begin{center}
    $T(n) = \frac{(n-1)n}{2}$+2
  \end{center}
  
  \item $T(n)=3T(n-1)+2, T(1)=1$
  
  \textbf{Soluci\'on Por Funci\'on generatriz de probabilidad (basada en la tranformada Z)}
  
  Transformamos T(n) en A(z) y aumentamos el caso base a 0:\\
  $T(n+1)=3T(n) + 2$\\
  $T(1) = 3T(0) + 2$, Entonces, $T(0)=-\frac{1}{3}$\\
  Reemplazando por las funciones generatrices:\\
  $\frac{A(z) - A(0)}{z} = 3A(z) + \frac{2}{(1-z)}$\\
  $(1-3z)A(z) - \frac{2z}{(1-z)} + \frac{1}{3}=0$\\
  $A(z) = \frac{2z}{(1-z)(1-3z)} - (\frac{1}{3})\frac{1}{1-3z}$\\
  $A(z) = (\frac{2}{3})\frac{1}{1-3z} - \frac{1}{(1-z)} $\\
  Haciendo transformada inversa:\\
  $T(n) = (\frac{2}{3})3^n -1$\\
  
  Soluci\'on encontrada por el m\'etodo de función generatriz:
  \begin{center}
    $T(n) = 2*3^{n-1} - 1$
  \end{center}
  
  \textbf{Soluci\'on Por Ecuaci\'on Caracter\'istica}
  
  Se debe generar un polinomio caracter\'istico basado en la recursi\'on.\\
  $T(n+1) = 3T(n) +2, T(1) = 1$\\
  Separamosla soluci\'on general como:\\
  $S_G = S_p + S_h$\\
  Donde:\\
  $S_G$: Es la soluci\'on general.\\
  $S_H$: Es la soluci\'on homog\'enea.\\
  $S_P$: Es la soluci\'on particular.\\
  
  \textbf{C\'aculo de la soluci\'on homog\'enea:}\\
  Sea $ T(n)_H = r^n $, cambiamos la recursi\'on por:\\
  $T(n+1)_H - T(n)_H = 0$\\
  $r^{n+1} - 3r^{n} = 0$\\
  Dividimos entre $r^{n}$ y obtenemos la ra\'iz $(r-3)$\\
  Por lo cual la soluci\'on homog\'enea es:\\
  $T(n)_H = \alpha (3)^n$
  
  \textbf{C\'aculo de la soluci\'on particular:}\\
  Sabemos que $T(n)_P = A_1n+A_0$, por lo que el caso particular ser\'a:\\
  Entonces hacemos:\\
  $T(n+1)_P=3*T(n)_P + 2$\\
  $A_1(n+1)+A_0 = 3*(A_1n+A_0) + 2$\\
  Por lo tanto $A_1=0, A_0=-1$, se tiene la soluci\'on particular:\\
  $T(n)_P = -1$
  
  \textbf{C\'aculo de la soluci\'on general:}\\
  Tenemos:\\
  $S_G = S_H + S_P$\\
  $T(n)_G = \alpha (3)^n - 1$\\
  Usando el caso base T(1) = 1:\\
  $T(1) = \alpha (3)^1 - 1$\\
  $1 = \alpha (3) - 1$\\
  $\alpha = \frac{2}{3}$\\
  Por lo tanto, la funci\'on recursiva es:\\
  $T(n) = \frac{2}{3} 3^n - 1$\\
  
  Soluci\'on encontrada por el m\'etodo de ecuaci\'on caracter\'istica:
  \begin{center}
  	$T(n) = 2*3^{n-1} - 1$
  \end{center}
  
  \item $T(n)=6T(n/6)+2n+3, T(1)=1$
  
  \textbf{Soluci\'on Por Funci\'on generatriz de probabilidad (basada en la tranformada Z)}
  
  Haciendo cambio de variable $n=6^m$\\
  Tendríamos:\\
  $T(6^m)=6T(6^{m-1})+2*6^m+3, T(1)=1$\\
  Sea $T(6^m) = G(m)$, tendríamos que T(1) = G(0)=1.\\
  Reemplazando:\\
  $G(m)=6G(m-1)+2*6^m+3$\\
  Transformamos G(m) en A(z) y aumentamos el caso base a 0:\\
  $G(m+1)=6G(m)+2*6*6^{m}+3, G(0)=1$\\
  Reemplazando por las funciones generatrices:\\
  $\frac{A(z) - A(0)}{z} = 6A(z) + 2*6*\frac{1}{(1-6z)} +\frac{3}{(1-z)}$\\
  $(1-6z)A(z) - 2*6*\frac{z}{(1-6z)} - \frac{3z}{(1-z)} - 1=0$\\
  $A(z) = 2*6*\frac{z}{(1-6z)^2} + \frac{3z}{(1-z)(1-6z)} + \frac{1}{(1-6z)}$\\
  $A(z) = 2*6*\frac{z}{(1-6z)^2} - (\frac{3}{5})\frac{1}{(1-z)} + (\frac{3}{5})\frac{1}{(1-6z)} + \frac{1}{(1-6z)}$\\
  $A(z) = 2*\frac{6z}{(1-6z)^2} - (\frac{3}{5})\frac{1}{(1-z)} + (\frac{8}{5})\frac{1}{(1-6z)}$\\
  Haciendo transformada inversa:\\
  $G(m) = 2*6^m *m - \frac{3}{5} + \frac{8}{5}6^m$\\
  Reemplazando por T(n):\\
  $T(n) = 2nlog_6(n) + \frac{8}{5}n - \frac{3}{5}$\\
  
  Soluci\'on encontrada por el m\'etodo de función generatriz:
  \begin{center}
  	$T(n) = (2log_6(2))(nlog(n)) + \frac{8}{5}n - \frac{3}{5}$\\
  \end{center}
  
  \textbf{Soluci\'on Por Iteraci\'on}
  
  $T(n)=6T(n/6)+2n+3, T(1)=1$\\
  Hacemos cambio de variable $n = 6^m$\\
  Entonces para cada n multiplo de 6 se tendr\'ia en T(n) = G(m) y $n = 6^m$:\\
  $G(m)= 6*G(m-1) + 2*6^m + 3$\\
  Expandiendo G(m-1):\\
  $G(m)= 6(6*G(m-2) + 2*6^{m-1}+3) + 2*6^m + 3$\\
  $G(m)= 6^2*G(m-2) + 2*(2*6^{m}) + 3*6 + 3$\\
  Expandiendo G(m-2):\\
  $G(m)= 6^2(6*G(m-3) + 2*6^{m-2} +3)+2*6^{m}+3*6 + 2*6^m + 3$\\
  $G(m)= 6^3*G(m-3) + 3*(2*6^{m}) + 3*6^2 + 3*6 + 3$\\
  Expandiendo G(m-3):\\
  $G(m)= 6^3(6*G(m-4)+2*6^{m-3}+3) + 3*(2*6^{m}) + 3*6^2 + 3*6 + 3$\\
  $G(m)= 6^4*G(m-4) + 4*(2*6^{m}) + 3*6^3 + 3*6^2 + 3*6 + 3$\\
  Entonces, expandiendo hasta (m):\\
  $G(m)= 6^{m}*G(m-(m)) + (m)*(2*6^{m}) + 3*(\sum_{i=0}^{m-1}{6^i})$\\
  Tenemos:\\
  $G(m)= 6^{m}*G(0) + (m)*(2*6^{m}) + 3*(\sum_{i=0}^{m-1}{6^i})$\\
  $G(m)= 6^{m}*G(0) + (m)*(2*6^{m}) + 3*(\frac{6^{m}-1}{5})$\\
  reemplazando por n:\\
  $T(n) = n * T(1) + log_6 (n)*(2*n) + \frac{3}{5}*(n-1)$\\
  
  Soluci\'on encontrada por el m\'etodo de iteraci\'on y cambio de variable:
  \begin{center}
  	$T(n) = 2nlog_6 (n) + \frac{8}{5}n - \frac{3}{5}$\\
  \end{center}
  O tambi\'en:
  \begin{center}
  	$T(n) = (2log_6(2))(nlog(n)) + \frac{8}{5}n - \frac{3}{5}$\\
  \end{center}
 
  
  \item $T(n)=4T(n/3)+3n-5, T(1)=2$
  
  \textbf{Soluci\'on Por Funci\'on generatriz de probabilidad (basada en la tranformada Z)}
  
  Haciendo cambio de variable $n=3^m$\\
  Tendríamos:\\
  $T(3^m)=4T(3^{m-1})+3*3^m-5$\\
  Sea $T(3^m) = G(m)$, tendríamos que T(1) = G(0)=2.\\
  Reemplazando:\\
  $G(m)=4G(m-1)+3*3^m-5$\\
  Transformamos G(m) en A(z) y aumentamos el caso base a 0:\\
  $G(m+1)=4G(m)+3*3*3^{m}-5, G(0)=2$\\
  Reemplazando por las funciones generatrices:\\
  $\frac{A(z) - A(0)}{z} = 4A(z) + 3*3*\frac{1}{(1-3z)} -\frac{5}{(1-z)}$\\
  $(1-4z)A(z) - 3*3*\frac{z}{(1-3z)} + \frac{5z}{(1-z)} - 2=0$\\
  $A(z) = 3*3*\frac{z}{(1-3z)(1-4z)} - \frac{5z}{(1-z)(1-4z)} + \frac{2}{(1-4z)}$\\
  $A(z) = 3*3*\frac{z}{(1-3z)(1-4z)} + (\frac{5}{3})\frac{1}{(1-z)} - (\frac{5}{3})\frac{1}{(1-4z)} + \frac{2}{(1-4z)}$\\
  $A(z) = 3*3*\frac{1}{(1-4z)} - 3*3*\frac{1}{(1-3z)} + (\frac{5}{3})\frac{1}{(1-z)} - (\frac{5}{3})\frac{1}{(1-4z)} + \frac{2}{(1-4z)}$\\
  $A(z) = (\frac{28}{3})\frac{1}{(1-4z)} - 3*3*\frac{1}{(1-3z)} + (\frac{5}{3})\frac{1}{(1-z)}$\\
  Haciendo transformada inversa:\\
  $G(m) = (\frac{28}{3})4^m - 9*3^m + (\frac{5}{3})$\\
  Reemplazando por T(n):\\
  $T(n) = (\frac{28}{3})4^{log_3(n)} - 9*3^{log_3(n)} + (\frac{5}{3})$\\
  
  Soluci\'on encontrada por el m\'etodo de función generatriz:
  \begin{center}
  	$T(n) =  \frac{28}{3}n^{log_3(4)} - 9n  + \frac{5}{3} $\\
  \end{center}
  
  \textbf{Soluci\'on Por Iteraci\'on}
  
  Similar al caso anterior, expandimos:\\
  $T(n) = 4 T(n/3) + 3n - 5$, T(1)=2\\
  $T(n) = 4^2 T(n/3^2) + \frac{4}{3} 3n - 4 . 5  + 3n - 5$\\
  $T(n) = 4^3 T(n/3^3) + (\frac{4}{3})^2 3n + \frac{4}{3} 3n + 3n -4^2 . 5 - 4 . 5 - 5$\\
  Expandiendo hasta $k = log_3 (n)$ Tenemos:\\
  $T(n) = 4^k T(1) +  3n\sum_{i=0}^{k-1} (\frac{4}{3})^i - 5\sum_{i=0}^{k-1} (4)^i $\\ 
  $T(n) = 4^k T(1) +  9n ((\frac{4}{3})^k - 1)  - \frac{5}{3}(4^k - 1) $\\
  $T(n) = 2 n^{log_3(4)} +  9 n^{log_3(4) - log_3(3) + 1} - 9n  - \frac{5}{3}(n^{log_3(4)} - 1) $\\
  
  Soluci\'on encontrada por el m\'etodo de iteración:
  \begin{center}
  	$T(n) =  \frac{28}{3}n^{log_3(4)} - 9n  + \frac{5}{3} $\\
  \end{center}
  
  \item $T(n)=T(n/4)+n-1, T(1)=2$
  
  \textbf{Soluci\'on Por Funci\'on generatriz de probabilidad (basada en la tranformada Z)}
  
  Haciendo cambio de variable $n=4^m$\\
  Tendríamos:\\
  $T(4^m)=T(4^{m-1})+4^m-1$\\
  Sea $T(4^m) = G(m)$, tendríamos que T(1) = G(0)=2.\\
  Reemplazando:\\
  $G(m)=G(m-1)+4^m-1$\\
  Transformamos G(m) en A(z) y aumentamos el caso base a 0:\\
  $G(m+1)=G(m)+4*4^{m}-1, G(0)=2$\\
  
  Reemplazando por las funciones generatrices:\\
  $\frac{A(z) - A(0)}{z} = A(z) + 4*\frac{1}{(1-4z)} -\frac{1}{(1-z)}$\\
  $(1-z)A(z) - 4*\frac{z}{(1-4z)} + \frac{z}{(1-z)} - 2=0$\\
  $A(z) = 4*\frac{z}{(1-z)(1-4z)} - \frac{z}{(1-z)^2} + \frac{2}{(1-z)}$\\
  $A(z) = (\frac{4}{3})\frac{1}{(1-4z)} - (\frac{4}{3})\frac{1}{(1-z)} - \frac{z}{(1-z)^2} + \frac{2}{(1-z)}$\\
  $A(z) = (\frac{4}{3})\frac{1}{(1-4z)} - \frac{z}{(1-z)^2} + (\frac{2}{3})\frac{1}{(1-z)}$\\
  Haciendo transformada inversa:\\
  $G(m) = (\frac{4}{3})4^m - m + \frac{2}{3}$\\
  Reemplazando por T(n):\\
  $T(n) = \frac{4}{3}n - log_4(n) + \frac{2}{3}$\\
  
  Soluci\'on encontrada por el m\'etodo de función generatriz:
  \begin{center}
  	$T(n) = \frac{4}{3}n - \frac{1}{2}log(n) + \frac{2}{3}$\\
  \end{center}
  
  \textbf{Soluci\'on Por Iteraci\'on}
  
  Similar al caso anterior, expandimos:\\
  $T(n) = T(n/4) + n - 1$, T(1)=2\\
  $T(n) = T(n/4^2) + \frac{1}{4}n + n - 2$\\
  $T(n) = T(n/4^3) + \frac{1}{4^2}n + \frac{1}{4}n + n - 3$\\
  $T(n) = T(n/4^4) + \frac{1}{4^3}n + \frac{1}{4^2}n + \frac{1}{4}n + n - 4$\\
  Expandiendo hasta un termino $k=log_4(n)$:\\
  $T(n) = T(n/4^k) + \sum_{i=0}^{k-1}{(\frac{1}{4})^i}n - k$\\
  $T(n) = T(1) + \frac{4}{3}n - \frac{4}{3} - log_4(n)$\\
  
  Soluci\'on encontrada por el m\'etodo de iteración:
  \begin{center}
  	$T(n) = \frac{4}{3}n - \frac{1}{2}log(n) + \frac{2}{3}$\\
  \end{center}
  
  \item $T(n)=T(n-2)+n, T(0)=c,T(1)=d$, Resuelva para todo $n$
  
  \textbf{Soluci\'on Por Funci\'on generatriz de probabilidad (basada en la tranformada Z)}
  
  Hacemos cambio de variable:\\
  $T(n+2)=T(n)+n+2$\\
  Transformamos T(n) en A(z):\\  
  Reemplazando por las funciones generatrices:\\
  $\frac{\frac{A(z) - A(0)}{z}-A(1)}{z} = A(z) + \frac{z}{(1-z)^2} + \frac{2}{(1-z)}$\\
  $A(z) = \frac{z^3}{(1-z)^3(1+z)} + \frac{2z^2}{(1-z)^2(1+z)} + \frac{dz}{(1-z)(1+z)} + \frac{c}{(1-z)(1+z)}$\\
  $A(z) = (\frac{1}{8})\frac{7z^2 - 4z + 1}{(1-z)^3} - (\frac{1}{8})\frac{1}{(1+z)} + (\frac{1}{2})\frac{3z-1}{(1-z)^2}+(\frac{1}{2})\frac{1}{(1+z)} + (\frac{d}{2})\frac{1}{(1-z)} - (\frac{d}{2})\frac{1}{(1+z)} + (\frac{c}{2})\frac{1}{(1-z)} + (\frac{c}{2})\frac{1}{(1+z)}$\\
  $A(z) = (\frac{1}{8})\frac{7z^2 - 4z + 1}{(1-z)^3} + (\frac{1}{2})\frac{3z-1}{(1-z)^2}+(\frac{3}{8})\frac{1}{(1+z)} + (\frac{d}{2})\frac{1}{(1-z)} - (\frac{d}{2})\frac{1}{(1+z)} + (\frac{c}{2})\frac{1}{(1-z)} + (\frac{c}{2})\frac{1}{(1+z)}$\\
  $A(z) = (\frac{1}{8})\frac{7z^2 - 4z + 1}{(1-z)^3} + (\frac{1}{2})\frac{3z-1}{(1-z)^2}+(\frac{4c+4d}{4})\frac{1}{(1-z)} + (\frac{4c-4d+3}{8})\frac{1}{(1+z)}$\\
  $A(z) = (\frac{1}{8})\frac{7z^2 - 4z + 1}{(1-z)^3} + \frac{1}{(1-z)^2} - (\frac{3}{2})\frac{1}{(1-z)}+(\frac{4c+4d}{8})\frac{1}{(1-z)} + (\frac{4c-4d+3}{8})\frac{1}{(1+z)}$\\
  $A(z) = (\frac{1}{8})\frac{7z^2 - 4z + 1}{(1-z)^3} + \frac{1}{(1-z)^2} +(\frac{4c+4d-12}{8})\frac{1}{(1-z)} + (\frac{4c-4d+3}{8})\frac{1}{(1+z)}$\\
  $A(z) = (\frac{7}{8})\frac{z^2}{(1-z)^3} + (\frac{1}{8})\frac{1-4z}{(1-z)^3} + \frac{1}{(1-z)^2} +(\frac{4c+4d-12}{8})\frac{1}{(1-z)} + (\frac{4c-4d+3}{8})\frac{1}{(1+z)}$\\
  $A(z) = (\frac{7}{8})\frac{z^2}{(1-z)^3} + (\frac{1}{8})\frac{-3}{(1-z)^3} + (\frac{1}{8})\frac{4-4z}{(1-z)^3} + \frac{1}{(1-z)^2} +(\frac{4c+4d-12}{8})\frac{1}{(1-z)} + (\frac{4c-4d+3}{8})\frac{1}{(1+z)}$\\
  $A(z) = (\frac{7}{8})\frac{z^2}{(1-z)^3} + (\frac{1}{8})\frac{-3}{(1-z)^3} + (\frac{4}{8})\frac{1}{(1-z)^2} + \frac{1}{(1-z)^2} +(\frac{4c+4d-12}{8})\frac{1}{(1-z)} + (\frac{4c-4d+3}{8})\frac{1}{(1+z)}$\\
  $A(z) = (\frac{1}{8})\frac{7z^2}{(1-z)^3} + (\frac{1}{8})\frac{-3}{(1-z)^3} + (\frac{12}{8})\frac{1}{(1-z)^2} + (\frac{4c+4d-12}{8})\frac{1}{(1-z)} + (\frac{4c-4d+3}{8})\frac{1}{(1+z)}$\\
  $A(z) = (\frac{1}{8})(\frac{7(z^2-1)}{(1-z)^3} + \frac{4}{(1-z)^3}) + (\frac{12}{8})\frac{1}{(1-z)^2} + (\frac{4c+4d-12}{8})\frac{1}{(1-z)} + (\frac{4c-4d+3}{8})\frac{1}{(1+z)}$\\
  $A(z) = (\frac{1}{8})(\frac{4}{(1-z)^3} - \frac{7(z+1)}{(1-z)^2}) + (\frac{12}{8})\frac{1}{(1-z)^2} + (\frac{4c+4d-12}{8})\frac{1}{(1-z)} + (\frac{4c-4d+3}{8})\frac{1}{(1+z)}$\\
  $A(z) = (\frac{4}{8})\frac{1}{(1-z)^3} + (\frac{7}{8})\frac{-z-1}{(1-z)^2} + (\frac{12}{8})\frac{1}{(1-z)^2} + (\frac{4c+4d-12}{8})\frac{1}{(1-z)} + (\frac{4c-4d+3}{8})\frac{1}{(1+z)}$\\
  $A(z) = (\frac{4}{8})\frac{1}{(1-z)^3} + (\frac{1}{8})\frac{-7z-7}{(1-z)^2} + (\frac{1}{8})\frac{12}{(1-z)^2} + (\frac{4c+4d-12}{8})\frac{1}{(1-z)} + (\frac{4c-4d+3}{8})\frac{1}{(1+z)}$\\
  $A(z) = (\frac{4}{8})\frac{1}{(1-z)^3} + (\frac{1}{8})\frac{5-7z}{(1-z)^2} + (\frac{4c+4d-12}{8})\frac{1}{(1-z)} + (\frac{4c-4d+3}{8})\frac{1}{(1+z)}$\\
  $A(z) = (\frac{4}{8})\frac{1}{(1-z)^3} + (\frac{1}{8})\frac{7-7z}{(1-z)^2} - (\frac{1}{8})\frac{2}{(1-z)^2} + (\frac{4c+4d-12}{8})\frac{1}{(1-z)} + (\frac{4c-4d+3}{8})\frac{1}{(1+z)}$\\
  $A(z) = (\frac{4}{8})\frac{1}{(1-z)^3} + (\frac{1}{8})\frac{7}{(1-z)} - (\frac{1}{8})\frac{2}{(1-z)^2} + (\frac{4c+4d-12}{8})\frac{1}{(1-z)} + (\frac{4c-4d+3}{8})\frac{1}{(1+z)}$\\
  $A(z) = (\frac{4}{8})\frac{1}{(1-z)^3} - (\frac{1}{8})\frac{2}{(1-z)^2} + (\frac{4c+4d-5}{8})\frac{1}{(1-z)} + (\frac{4c-4d+3}{8})\frac{1}{(1+z)}$\\
  $A(z) = (\frac{4}{8})\frac{1}{(1-z)^3} - (\frac{2}{8})\frac{1}{(1-z)^2} - (\frac{2}{8})\frac{1-z}{(1-z)^2} + (\frac{2}{8})\frac{1-z}{(1-z)^2} + (\frac{4c+4d-5}{8})\frac{1}{(1-z)} + (\frac{4c-4d+3}{8})\frac{1}{(1+z)}$\\
  $A(z) = (\frac{4}{8})\frac{1}{(1-z)^3} - (\frac{2}{8})\frac{2-z}{(1-z)^2} + (\frac{4c+4d-3}{8})\frac{1}{(1-z)} + (\frac{4c-4d+3}{8})\frac{1}{(1+z)}$\\
  $A(z) = (\frac{4}{8})\frac{1}{(1-z)^3} - (\frac{4}{8})\frac{1}{(1-z)^2} + (\frac{2}{8})\frac{z}{(1-z)^2} + (\frac{4c+4d-3}{8})\frac{1}{(1-z)} + (\frac{4c-4d+3}{8})\frac{1}{(1+z)}$\\
  
  Haciendo transformada inversa:\\
  $T(n) = (\frac{4}{8}){n+2 \choose n} - (\frac{4}{8}){n + 1 \choose n} + (\frac{2}{8}){n \choose 1} + (\frac{4c+4d-3}{8})(1)^n + (\frac{4c-4d+3}{8})(-1)^n$\\
  $T(n) = (\frac{4}{8}) \frac{(n+2)(n+1)}{2} - (\frac{4}{8})(n+1) + (\frac{2}{8})n + (\frac{4c+4d-3}{8})(1)^n + (\frac{4c-4d+3}{8})(-1)^n$\\
  $T(n) = (\frac{4}{8}) \frac{(n+2)(n+1)}{2}- (\frac{2}{8})n - \frac{4}{8}  + (\frac{4c+4d-3}{8})(1)^n + (\frac{4c-4d+3}{8})(-1)^n$\\
  $T(n) = \frac{(n+2)(n+1)}{4}- \frac{n}{4} - \frac{2}{4} + (\frac{4c+4d-3}{8})(1)^n + (\frac{4c-4d+3}{8})(-1)^n$\\
  $T(n) = \frac{n^2+3n+2}{4}- \frac{n}{4} - \frac{2}{4} + (\frac{4c+4d-3}{8})(1)^n + (\frac{4c-4d+3}{8})(-1)^n$\\
  $T(n) = \frac{n^2+2n}{4} + (\frac{4c+4d-3}{8})(1)^n + (\frac{4c-4d+3}{8})(-1)^n$\\
  
  Soluci\'on encontrada por el m\'etodo de función generatriz:
  \begin{center}
    $T(n) = (\frac{4c + 4d - 3}{8})(1)^n + (\frac{4c - 4d + 3}{8})(-1)^n + \frac{n(n+2)}{4}$
  \end{center}
  
  \textbf{Soluci\'on Por Ecuaci\'on Caracter\'istica}
  
  Cambiamos por:\\
  $T(n+2)=T(n)+(n+2), T(0)=c,T(1)=d$\\
  
  \textbf{Caso homogéneo:}\\
  Sea $T(n)_H = r^n$:\\
  Entonces:\\
  $T(n+2)_H-T(n)_H=0$\\
  $r^{n+2} - r^{n} = 0$\\
  Diviendo entre $r^{n}$, se tiene:\\
  $r^2 - 1 = 0$, por lo cual\\
  Encontramos las raíces $(r-1)(r+1)$\\
  Por lo cual nuestra solución homogénea sería:\\
  $T(n)_H = \alpha (1)^n + \beta (-1)^n$\\
  
  \textbf{Caso particular:}\\
  Sea $T(n)_P = A_2n^2+A_1n+A_0$\\
  Entonces:\\
  $T(n+2)_P - T(n)_P = (n+2)$\\
  $A_2(n+2)^2+A_1(n+2)+A_0 = A_2n^2+A_1n+A_0 + (n+2)$\\
  Se tiene:\\
  $A_2=\frac{1}{4}, A_1=\frac{1}{2}$\\
  Por lo cual nuestra solución particular sería:\\
  $T(n)_P = \frac{1}{4}n^2 +\frac{1}{2}n = \frac{n(n+2)}{4}$\\
  
  \textbf{Caso de solución general:}\\
  $T(n)_G = T(n)_H+T(n)_P$\\
  $T(n) = \alpha (1)^n + \beta (-1)^n + \frac{n(n+2)}{4}$\\
  Probando los casos iniciales:\\
  $T(0) = \alpha + \beta = c$\\
  $T(1) = \alpha - \beta + \frac{3}{4} = d$\\
  $\alpha = \frac{4c + 4d - 3}{8}, \beta = \frac{4c - 4d + 3}{8}$\\
  
  Soluci\'on encontrada por el m\'etodo de ecuaci\'on caracter\'istica:
  \begin{center}
  	$T(n) = (\frac{4c + 4d - 3}{8})(1)^n + (\frac{4c - 4d + 3}{8})(-1)^n + \frac{n(n+2)}{4}$
  \end{center}
  
  % \item $x_{n+2}-3x_{n+1}+2x_n=n, x_0=x_1=1$
\end{enumerate}

\section{Recurrencias mas Complejas}
Resuelva las siguientes recurrencias usando funciones generatrices y cualquier otro método, es decir, cada recurrencia debe ser resuelto por dos métodos. 
La solución debe ser exacta para infinitos $n$.

\begin{enumerate}[label=\textbf{\alph*.}]
  \item Resuelva $T(0)=0, T(n)=1+\sum^{n-1}_{i=0} T(i)$
  
  \textbf{Soluci\'on Por Funci\'on generatriz de probabilidad (basada en la tranformada Z)}
  
  Hacemos cambio de base:\\
  T(n+1)=1+$\sum_{i=0}^{n} T(i)$, $T(0) = A_0 = 0$\\
  Transformamos T(n) en A(z):\\
  $\frac{A(z) - A_0}{z} = \frac{1}{1-z} + \frac{A(z)}{1-z}$\\
  $A(z)(\frac{1}{z}-\frac{1}{1-z})=\frac{1}{1-z}$\\
  $A(z) = \frac{z}{1-2z}$\\
  Haciendo transformada inversa:\\
  $T(n) = 2^{n-1}$

  Soluci\'on encontrada por el m\'etodo de función generatriz:
  \begin{center}
    $T(n) = 2^{n-1}$
  \end{center}
  
  \textbf{Soluci\'on Por Iteración}
    
    Expandiendo por iteración, tenemos:\\
    $T(n) = 1 + T(0) + T(1) + .... + T(n-1)$\\
    Debido a que cada T(n) crece linealmente (se puede ver en la recursión, tenemos:\\
    $T(n) = 1  + 2^0 + 2^1 + .... + 2^{n-2}$\\
    $T(n) = 1  + 2^{n-1} - 1$\\
    $T(n) = 2^{n-1}$

  Soluci\'on encontrada por iteración:
  \begin{center}
    $T(n) = 2^{n-1}$
  \end{center}

  \item Intente resolver lo anterior si la suma llega hasta $n$, ¿Que ocurre? ¿Cual es la explicación?
  
  \textbf{Soluci\'on Por Funci\'on generatriz de probabilidad (basada en la tranformada Z)}
  
  Se tiene:\\
  T(n)=1+$\sum_{i=0}^{n} T(i)$, $T(0) = A_0 = 0$\\
  Transformamos T(n) en A(z):\\
  $A(z) = \frac{1}{1-z} + \frac{A(z)}{1-z}$\\
  $A(z)(1-\frac{1}{1-z})=\frac{1}{1-z}$\\
  $A(z) = -\frac{1}{z} = -(z)^{-1}$\\
  Haciendo transformada inversa:\\
  $n=-1$

  Soluci\'on encontrada por el m\'etodo de función generatriz:
  \begin{center}
    No existe, seria una recurrencia infinita debido a que el termino T(n) se elimina.
  \end{center}

  \item Resuelva $T(0)=0, T(n)=1+\sum^{n-1}_{i=0}(T(i)+T(n-i))$
  
  \textbf{Soluci\'on Por Funci\'on generatriz de probabilidad (basada en la tranformada Z)}
  
  Hacemos cambio de base:\\
  T(n+1)=1+$\sum_{i=0}^{n} (T(i)+T(n+1-i))$, $T(0) = A_0 = 0$\\
  T(n+1)=1+$\sum_{i=0}^{n} T(i) + \sum_{i=0}^{n}T(n-i) + T(n+1)$\\
  0=1+$\sum_{i=0}^{n} T(i) + \sum_{i=0}^{n}T(n-i)$\\
  Transformamos T(n) en A(z):\\
  $0 = \frac{1}{1-z} + \frac{A(z)}{1-z} + \frac{A(z)}{1-z}$\\
  $A(z)(\frac{2}{1-z})=-\frac{1}{1-z}$\\
  $A(z) = -\frac{1}{2}$\\
  Haciendo transformada inversa:\\
  No esta definido.

  Soluci\'on encontrada por el m\'etodo de función generatriz:
  \begin{center}
    No esta definida la recurrencia, la cual se convierte en infinita debido a la eliminación de términos.
  \end{center}


\item Resuelva el sistema de recurrencias   $a_{n+1}=an+2b_{n}$, $b_{n+1}=3a_n+2b_n$ con $a_0=1$ y $b_0=-1$
  
  \textbf{Soluci\'on Por Funci\'on generatriz de probabilidad (basada en la tranformada Z)}
  
  Transformamos $a_n$ en A(z):\\
  $\frac{A(z)-1}{z}$ = A(z) + 2B(z)\\
  Entonces:\\
  A(z)(1-z) = 2zB(z) + 1 ...(1)\\
  
  Transformamos $b_n$ en B(z):\\
  $\frac{B(z)+1}{z}$ = 3A(z) + 2B(z)\\
  B(z)(1-2z) = 3zA(z)-1 ...(2)\\
  
  Reemplazamos (2) en (1):\\
  B(z)(1-2z) = 3z$(\frac{2zB(z)+1}{1-z})$-1\\
  $B(z)[1-2z-\frac{6z^2}{1-z}] = \frac{3z}{1-z}-1$\\
  $B(z)[\frac{1+2z^2-3z-6z^2}{1-z}]=\frac{3z-1+z}{1-z}$\\
  $B(z) = \frac{4z-1}{1-3z-4z^2} = \frac{4z-1}{(1-4z)(1+z)}$\\
  $B(z)=-\frac{1}{1+z}=\frac{-1}{1-(-1)z}$\\
  
  Reemplazando en A(z):\\
  $A(z) = \frac{1}{1-z}[\frac{-2z}{1+z}+1]$\\
  $A(z) =\frac{1}{1-z}[\frac{1-z}{1+z}]$\\
  $A(z) =\frac{1}{1+z}$\\
  Haciendo transformada inversa:
  $a_n = (-1)^n$\\
  $b_n = (-1)^{n+1}$
  
  Soluci\'on encontrada por el m\'etodo de función generatriz:
  \begin{center}
    $a_n = (-1)^n$\\
    $b_n = (-1)^{n+1}$
  \end{center}
  

  \item Prueba una versión mas general del Teorema Maestro, donde el paso recursivo dice $T(n)=aT(n/c)+bn^k$

    \textbf{Soluci\'on Por Funci\'on generatriz de probabilidad (basada en la tranformada Z)}
  
    Haciendo cambio de variable $n=c^m$ o $m=log_c(n)$\\
  Tendríamos:\\
  $T(c^m)=aT(c^{m-1})+b(c^{km})$\\
  Sea $T(c^m) = G(m)$%, tendríamos que T(1) = G(0)=2.\\
  Reemplazando:\\
  $G(m)=aG(m-1)++b(c^{km})$\\
  Transformamos G(m) en A(z) y aumentamos el caso base a 0:\\
  $G(m+1)=aG(m)++b((c^k)^{m+1})$\\
  
  Reemplazando por las funciones generatrices:\\
  $\frac{A(z) - A(0)}{z} = aA(z) + b(\frac{1}{(1-c^k z)})$\\
  $A(z) = azA(z) + b(\frac{z}{(1-c^k z)}) + \frac{A(0)}{1-z}$\\
  $(1-az)A(z) = b(\frac{z}{(1-c^k z)}) + \frac{A(0)}{1-z}$\\
  $A(z) = (\frac{bz}{(1-c^k z)(1-az)}) + \frac{A(0)}{(1-z)(1-az)}$\\
  $A(z) = (\frac{b}{c^k - a})\frac{1}{(1-c^k z)} - (\frac{b}{c^k - a})\frac{1}{(1-az)} + (\frac{A(0)}{1-a})\frac{1}{(1-z)} - (\frac{aA(0)}{1-a})\frac{1}{(1-az)}$\\
  Haciendo transformada inversa:\\
  $G(m) = (\frac{b}{c^k - a})(c^k)^m - (\frac{b}{c^k - a})(a^m) + (\frac{A(0)}{1-a}) - (\frac{aA(0)}{1-a})(a^m)$\\
  $T(n) = (\frac{b}{c^k - a})(c^k)^{log_c(n)} - (\frac{b}{c^k - a})(a^{log_c(n)}) + (\frac{A(0)}{1-a}) - (\frac{aA(0)}{1-a})(a^{log_c(n)})$\\
  $T(n) = (\frac{b}{c^k - a})(n^k) - (\frac{b}{c^k - a})(n^{log_c(a)}) + (\frac{A(0)}{1-a}) - (\frac{aA(0)}{1-a})(n^{log_c(a)})$\\
  
  Tenemos 3 casos:
    
    $a=c^k$:\\
    $T(n)=n^{k}T(1)$\\
    Pero en el ultimo nivel del caso base n=1, se tiene $log_c(n) operaciones$\\
    $T(n)=O(n^{k}log_c(n))$\\
    
    $a<c^k$:\\
    $log_c(a) < log_c(c^k)=k$\\
    $T(n)=O(n^k)$\\
    
    $a>c^k$:\\
    $log_c(a) > log_c(c^k)=k$\\
    $T(n)=O(n^{log_c(a)})$\\
  
  
  Reemplazando por T(n):\\
  $T(n) = \frac{4}{3}n - log_4(n) + \frac{2}{3}$\\
  
  Soluci\'on encontrada por el m\'etodo de función generatriz:
  \begin{center}
  	$T(n) = \frac{4}{3}n - \frac{1}{2}log(n) + \frac{2}{3}$\\
  \end{center}

    \textbf{Soluci\'on Por Iteración}
    
    Expandiendo por iteración, tenemos:\\
    $T(n)=aT(n/c)+bn^k$\\
    $T(n)=a^2T(n/c^2) + (\frac{a}{c^k})b n^k +bn^k$\\
    $T(n)=a^3T(n/c^3) + (\frac{a}{c^k})^2b n^k + (\frac{a}{c^k})b n^k +bn^k$\\
    Expandiendo hasta un termino $m=log_c(n)$:\\
    $T(n)=a^mT(n/c^m) + \sum_{i=0}^{m-1} (\frac{a}{c^k})^i b n^k $\\
    $T(n)=a^mT(n/c^m) + \frac{(\frac{a}{c^k})^m -1}{(\frac{a}{c^k})-1} bn^k$\\
    Reemplazando:\\
    $T(n)=n^{log_c(a)}T(1) + \frac{n^{log_c(a)-log_c(c^k)} -1}{(\frac{a-c^k}{c^k})} bn^k$\\
    $T(n)=n^{log_c(a)}T(1) + \frac{bc^k}{a-c^k}(n^{k+log_c(a)-k} -1)n^k$\\
    $T(n)=n^{log_c(a)}T(1) + (\frac{bc^k}{a-c^k})n^{log_c(a)} - (\frac{bc^k}{a-c^k})n^k$\\
    
    Tenemos 3 casos:
    
    $a=c^k$:\\
    $T(n)=n^{k}T(1)$\\
    Pero en el ultimo nivel del caso base n=1, se tiene $log_c(n) operaciones$\\
    $T(n)=O(n^{k}log_c(n))$\\
    
    $a<c^k$:\\
    $log_c(a) < log_c(c^k)=k$\\
    $T(n)=O(n^k)$\\
    
    $a>c^k$:\\
    $log_c(a) > log_c(c^k)=k$\\
    $T(n)=O(n^{log_c(a)})$\\

  \item Resuelva
  $$ T(n)=\sum^{\ln{n}}_{i=1}T(n/e^i)+\ln(n)^2$$
  para $n\geq 1$, $e$ indica para que valores de $n$ es exacta su solución

  \item Resuelva
  $$ a_n= \frac{a^{3/2}_{n/2}a^{3/2}_{n/4}}{\sqrt{2}a_{n/8}}$$
  Con $a_1=1, a_2=2, a_4=4$ en forma exacta para potencias de 2. Encuentre el orden del resultado y exprese el error de la aproximación con notación $O()$.

    \textbf{Soluci\'on Por Funci\'on generatriz de probabilidad (basada en la tranformada Z)}
  
    Haciendo cambio de variable $n=2^k$ o $k=log_2(n)$\\
  Tendríamos:\\
  $ a_n= \frac{a^{3/2}_{n/2}a^{3/2}_{n/4}}{\sqrt{2}a_{n/8}}$\\
  $ a_{2^k}= \frac{a^{3/2}_{2^{k-1}}a^{3/2}_{2^{k-2}}}{2^{1/2}a_{2^{k-3}}}$\\
  $2^{1/2}a_{2^{k-3}} a_{2^k}= a^{3/2}_{2^{k-1}}a^{3/2}_{2^{k-2}}$\\
  Haciendo otro cambio de variable a función:\\
  $a_{2^k}=2^{b(k)}$\\
  De tal manera que:\\
  b(0)= 0, b(1)=1, b(2) = 2.\\
  $(2^{1/2})(2^{b(k-3)}) (2^{b(k)})= (2^{(3/2)b(k-1)}) (2^{(3/2)b(k-2)})$\\
  Entonces, de los exponentes tenemos:\\
  (1/2) + b(k-3) + b(k) = (3/2)b(k-1) + (3/2)b(k-2)\\
  Transformamos b(k) en B(z):\\
  $(\frac{1}{2})(\frac{1}{1-z})+ ((\frac{B(z)}{z}-1)\frac{1}{z}-2)\frac{1}{z} + B(z) = (3/2)\frac{B(z)}{z} + (3/2)(\frac{B(z)}{z}-1)(\frac{1}{z})$\\
  $B(z)(\frac{1}{z^3}-\frac{3}{z^2}-\frac{3}{z}+1) = \frac{1}{z^2} + \frac{2}{z} - \frac{3}{2z} - (\frac{1}{2})(\frac{1}{1-z})$\\
  $B(z)(2z^3-3z^2-3z+2)=-3z^2 - \frac{z^3}{1-z}+2z+4z^2$\\
  $B(z)(2z^3-3z^2-3z+2)=z^2-\frac{z^3}{1-z}+2z$\\
  $B(z)(2z^3-3z^2-3z+2)=-z(-z+\frac{z^2}{1-z}-2)$\\
  $B(z)(2z^3-3z^2-3z+2)=-z(\frac{z^2-(2-z-z^2)}{1-z})$\\
  $B(z)(2z^3-3z^2-3z+2)=-z(\frac{2z^2+z-2}{1-z})$\\
  $B(z)=-z(\frac{2z^2+z-2}{(1-z)(2z^3-3z^2-3z+2)})$\\
  $B(z)=(-\frac{8}{9})(\frac{1}{1-(1/2)z})+(\frac{1}{2})(\frac{1}{1-z}) + (\frac{4}{9})(\frac{1}{1-2z})+(-\frac{1}{18})(\frac{1}{1+z})$\\
  Haciendo transformada inversa:\\
  $b(k) = (-\frac{8}{9})(1/2)^k+(\frac{1}{2}) + (\frac{4}{9})(2)^k+(-\frac{1}{18})(-1)^k$\\
  Regresando hacia $a(2^k)$:\\
  $a(2^k) = 2^{((-\frac{8}{9})(1/2)^k+(\frac{1}{2}) + (\frac{4}{9})(2)^k+(-\frac{1}{18})(-1)^k)}$\\
  
  Soluci\'on encontrada por el m\'etodo de función generatriz:
  \begin{center}
  	$a(n) = 2^{((-\frac{8}{9})(1/n)+(\frac{1}{2}) + (\frac{4}{9})n+(-\frac{1}{18})(-1)^{log(n)})}$\\
  \end{center}

  \item Resolver $S(n)=nS(n/2),S(1)=1$

    \textbf{Soluci\'on Por Funci\'on generatriz de probabilidad (basada en la tranformada Z)}
  
    Haciendo cambio de variable $n=2^k$ o $k=log_2(n)$\\
  Tendríamos:\\
  $ S(2^k)=2^{b(k)}$\\
  De donde:\\
  $S(2^0)=S(1) = 2$, $S(2^0) = 2^{b(0)}$, b(0)=0.\\
  $ 2^{b(k)} = 2^k (2^{b(k-1)})$\\
  $ 2^{b(k)} = 2^{k+b(k-1)}$\\
  $ b(k) =k+b(k-1)$\\
  Cambio de base:\\
  $ b(k+1) =b(k)+k+1$\\
  Transformamos b(k) en B(z):\\
  $\frac{B(z)-b_0}{z} = B(z) + \frac{z}{(1-z)^2}+\frac{1}{1-z}$\\
  $B(z)(\frac{1}{z}-1) = \frac{z}{(1-z)^2}+\frac{1-z}{(1-z)^2}$\\
  $B(z)(\frac{1-z}{z}) = \frac{1}{(1-z)^2}$\\
  $B(z) = \frac{z}{(1-z)^3}$\\
  Haciendo transformada inversa:\\
  $b(k) = {(k-1)+2 \choose 2}$\\
  $b(k) = \frac{k(k+1)}{2}$\\
  Regresando hacia $S(2^k)$:\\
  $S(2^k) = 2^{\frac{k(k+1)}{2}}$\\
    
  Soluci\'on encontrada por el m\'etodo de función generatriz:
  \begin{center}
  	$S(n) = n^{log(n)/2}+n^{1/2}$\\
  \end{center}

  \item Obtenga el orden $\Theta()$ de la siguiente recurrencia: $f(n)=f(\alpha n)+f(\beta n)+cn$, donde $\alpha +\beta =1$ y son positivos. Ayuda: Ud. conoce el resultado para $\alpha = \beta = 1/2$
  
\end{enumerate}


%%%%%%%%%%%%%%%%%%%%%%%%%%%%%%%%%%%%%%%%%%%%%%%%%%%%%%%%%%%%%%%%%%%%%%%%%%%%%
%%%%%%%%%%%%%%%%%%%%%%%%%%%%%%%%%%%%%%%%%%%%%%%%%%%%%%%%%%%%%%%%%%%%%%%%%%%%%
%%%%%%%%%%%%%%%%%%%%%%%%%%%%%%%%%%%%%%%%%%%%%%%%%%%%%%%%%%%%%%%%%%%%%%%%%%%%%

\title{Divide y Vencerá}
\date{}
\maketitle
\setcounter{section}{0}


\section{Monge arrays}
An $m \times n$ array $A$ of real numbers is a \textbf{Monge array} if for all $i, j, k$, and $l$ such that $1 \leq i \leq m$ and $1 \leq  j < l \leq n$ we have

$$
A[i, j] + A[k, l] \leq A[i, l] + A[k, j]
$$

In other words, whenever we pick two rows and two columns of a Monge array and consider the four elements at the intersections of the rows and the columns, the sum of the upper-left and lower-right elements is less than or equal to the sum of the lower-left and upper-right elements. For example, the following array is Monge:

\begin{tabular}{ccccc}
10 & 17 & 13 & 28 & 23\\
17 & 22 & 16 & 29 & 23\\
24 & 28 & 22 & 34 & 24\\
11 & 13 &  6 & 17 & 7\\
45 & 44 & 32 & 37 & 23\\
36 & 33 & 19 & 21 & 6\\
75 & 66 & 51 & 53 & 34
\end{tabular}


\begin{enumerate}[label=\textbf{\alph*.}]
  \item Prove that an array is Monge if and only if for all $I=1, 2, ..., M-1$ and $J=1, 2, ..., N-1$, we have
  $$A[i, j] + A[i+1, j+1] \leq A[i, j+1] + A[i+1, j]$$
  (\emph{Hint}: For the ``if'' part, use induction separately on rows and columns.)
  
  \textbf{Solución:}
  
  Para una matriz nxm, Tenemos que demostrar que:\\
  Para $i \leq k \leq n$ y $j < l \leq m$\\
  $A[i, j] + A[k, l] \leq A[i, l] + A[k, j] \Leftrightarrow A[i, j] + A[i+1, j+1] \leq A[i, j+1] + A[i+1, j]$\\
  
  Para demostrar el lado derecho:\\
  $A[i, j] + A[k, l] \leq A[i, l] + A[k, j] \Rightarrow A[i, j] + A[i+1, j+1] \leq A[i, j+1] + A[i+1, j]$\\
  Tenemos que por definición, debe cumplir.\\
  
  
  Para demostrar el lado izquierdo:\\
  $A[i, j] + A[k, l] \leq A[i, l] + A[k, j] \Leftarrow A[i, j] + A[i+1, j+1] \leq A[i, j+1] + A[i+1, j]$\\
  Lo hacemos inducción:\\
  Hacemos la prueba para k:\\
  \begin{center}
    \begin{tabular}{ccc}
      $a_{i,j}$ & ... & $a_{i,l}$\\
      . & . & .\\
      $a_{k,j}$ &  ... & $a_{k,l}$\\
      $a_{k+1,j}$ & ... & $a_{k+1,l}$
    \end{tabular}
  \end{center}
  
  De aqui vemos que por definición del Monge Array:\\
  $A[k, j] + A[k+1, l] \leq A[k, l] + A[k+1, j]$.
  
  Usando nuestra hipótesis de inducción:\\
  $A[k, j] + A[k+1, l] + A[i,j] + A[k,l] \leq A[k, l] + A[k+1, j] + A[i,l] + A[k,j]$\\
  Y de esto obtenemos:\\
  $A[k+1, l] + A[i,j] \leq  A[k+1, j] + A[i,l]$\\
  Tomando desde l = j+1 y k=i+1\\
  $A[i,j] + A[i+1, j+1] \leq A[i,j+1] + A[i+1, j]$\\
  
  \item The following array is not Monge. Change one element in order to make it Monge. (\emph{Hint}: Use part (a))

    \begin{tabular}{cccc}
    37 &23& 22& 32 \\
    21 &6 &7 &10\\
    53 &34 &30 &31\\
    32 &13 &9 &6\\
    43 &21 &15 &8\\
    \end{tabular}

	  \textbf{Solución:}
      
      Tomando la formula general:
      
        \begin{tabular}{cc}
          23& 22 \\
          6 &7 \\
        \end{tabular}
        
      Vemos que no cumple para la submatriz con i = 1, j = 1 ya que:\\
      $23+7 > 22+6$, si cambiamos 22 por 24, tendríamos:\\
      $23+7 \leq 24+6$ que si cumple la relación.
      
      Por lo tanto la nueva matriz es:
      
       \begin{tabular}{cccc}
        37 &23& \textbf{24}& 32 \\
        21 &6 &7 &10\\
        53 &34 &30 &31\\
        32 &13 &9 &6\\
        43 &21 &15 &8\\
        \end{tabular}
      

  \item Let $f(i)$ be the index of the column containing the leftmost minimum element of row $i$. Prove that $f(1) \leq f(2) \leq ... \leq f(m)$ for any $m \times n$ Monge array.
  
  \textbf{Solución:}
  
  Por contradicción:\\
  Asumimos que existe i tal que : $f(i)>f(i+1)$\\
  Construyendo la matriz:\\
  \begin{center}
    \begin{tabular}{ccc}
      $a_{i,f(i+1)}$ & ... & $a_{i,f(i)}$\\
      . & . & .\\
      $a_{i+1,f(i+1)}$ &  ... & $a_{i+1,f(i)}$
      \end{tabular}
  \end{center}
  De la formula general, sabemos que:\\
  $A[i,f(i+1)] + A[i+1, f(i)] \leq A[i,f(i)] + A[i+1, f(i+1)]$ y :\\
  Siendo A [i, f (i)] y A [i + 1, f (i + 1)] los mínimos a la izquierda y vemos que $A[i,f(i+1) < A[i,f(i)]]$ rompe la contradiccón.
  
  Por lo tanto, siempre cumple que:\\
  $f(1) \leq f(2) \leq ... \leq f(m)$

  \item Here is a description of a divide-and-conquer algorithm that computes the left-most minimum element in each row of an $m \times n$ Monge array $A$:
  
  \begingroup
  \leftskip2em
  \rightskip\leftskip
  Construct a submatrix $A'$ of $A$ consisting of the even-numbered rows of A. Recursively determine the leftmost minimum for each row of $A'$. Then compute the leftmost minimum in the odd-numbered rows of $A$.
  \par
  \endgroup

  Explain how to compute the leftmost minimum in the odd-numbered rows of $A$ (given that the leftmost minimum of the even-numbered rows is known) in $O(m+n)$ time.
  
  \textbf{Solución:}
  
  Sabemos, por la propiedad anterior, que para todo indice se cumple:\\
  $f(2i) < f(2i+1) < f(2(i+1))$\\
  Necesitamos encontrar los más pequeños entre ese rango.\\
  El tiempo de ejecución para encontrarlos (como son O(1) por definición) es:\\
  $T(n) = \sum_{i=0}^{\lfloor n/2 \rfloor} ( f(2(i+1)) - f(2i)+1)$\\
  $T(n) = \sum_{i=1}^{\lfloor n/2 \rfloor+1} f(2(i+1)) - \sum_{i=0}^{\lfloor n/2 \rfloor} f(2i) + \lfloor n/2 \rfloor$\\
  $T(n) = f(\lfloor n/2 \rfloor +1) - f(0)+\lfloor n/2 \rfloor$\\
  $T(n) \leq m-1 + \lfloor n/2 \rfloor = O(n+m)$
  
  \item Write the recurrence describing the running time of the algorithm described in part (d). Show that its solution is $O(m+n\log m)$.
  
  \textbf{Solución:}
  
  Dado que cada vez que el tamaño se divide en solo las filas pares y el paso de fusión toma O (n + m), el tiempo de ejecución del algoritmo viene dado por la recurrencia:
  
  $f(n) = f(\lceil n/2 \rceil) + O(n+m)$\\
  Haciendo recursión hasta un k = logn.\\
  El caso base es O(1) ya que m es constante, y como n es una potencia de 2 hay niveles de log(n) y cada uno toma m tiempo constante de modo que:\\
  $f(n) = km + \sum_{i=0}^{k-1} \frac{n}{2^i}$\\
  $f(n) = mlog(n) + n \sum_{i=0}^{k-1} \frac{1}{2^i}$\\
  $f(n) = mlog(n) + n (2-2/n)$\\
  $f(n) = mlog(n) + 2n - 2$\\
  $f(n) = O(mlog(n) +n)$\\
  
\end{enumerate}

\section{Small order statistics}
We showed that the worst-case number $T(n)$ of comparisons used by SELECT to select the $i$th order statistic from $n$ numbers satisfies $T(n) = \Theta(n)$, but the constant hidden by the $\Theta$-notation is rather large. When $i$ is small relative to $n$, we can implement a different procedure that uses SELECT as a subroutine but makes fewer comparisons in the worst case.

\begin{enumerate}[label=\textbf{\alph*.}]
  \item Describe an algorithm that uses $U_i(n)$ comparisons to find the $i$th smallest of $n$ elements, where
  
\[
U_i(n) = \left\{
  \begin{array}{lr}
    T(n) & \textnormal{if } i \geq n/2\\
    \lfloor n/2 \rfloor + U_i(\lceil n/2\rceil) + T(2i) & \textnormal{otherwise}
  \end{array}
\right.
\]
  
  (Hint: Begin with $\lfloor n/2 \rfloor$ disjoint pairwise comparisons, and recurse on the set containing the smaller element from each pair.)
  
  \textbf{Solución:}
  
  Del enunciado, sabemos que si i es pequeño, hacemos la función SELECT que demora T(n). Este algoritmo sirve para obtener el i-esimo elemento más pequeño.

	\begin{itemize}
	\item Si $i \geq n/2$, usamos SELECT.
    \item Sino, dividimos el array en pares y los comparamos.
    \item Tomamos el menor elemento de cada par, y continuamos con el siguiete.
    \item Luego, recursivamente encontramos el i-esimo elemento entre los más pequeños.
    \item El \textit{small order statistics} i-esimo está entre los pares que contienen los elementos más pequeños que obtenemos en el paso anterior. Llamamos a SELECT en esos 2i elementos.
    \item Y dicho i-esimo es el elemento buscado.
	\end{itemize}
  
  \item Show that, if $i < n/2$, then $U_i(n) = n + O(T(2i)\log(n/i))$.
  
  \textbf{Solución:}
  
  para un $i<\frac{n}{2}$ tenemos:
  
  $U_i(n) = \lfloor n/2 \rfloor + U_i(\lceil n/2\rceil) + T(2i)$\\
  Haciendo por iteración:\\
  $U_i(n) = \lfloor n/2 \rfloor + (\lfloor n/2 \rfloor +  U_i(\lceil n/2^2 \rceil)  + T(2i)) + T(2i)$\\
  $U_i(n) = 2\lfloor n/2 \rfloor + U_i(\lceil n/2^2 \rceil) + 2T(2i)$\\
  Hasta un kesimo termino:\\
  $U_i(n) = k\lfloor n/2 \rfloor + U_i(\lceil n/2^k \rceil) + kT(2i)$\\
  Tomando hasta $k=log(n)$\\
  $U_i(n) = log(n)\lfloor n/2 \rfloor + U_i(\lceil 1 \rceil) + log(n)T(2i)$\\
  $U_i(n) = log_2(n)\lfloor n/2 \rfloor + log_2(n)\lfloor n/2/i \rfloor T(2i) + log(n)T(2i) $\\
  $U_i(n) = log_2(n)\lfloor n/2 \rfloor + log_2(n)\lfloor n/2 \rfloor T(2i)- log(i)T(2i) + log(n)T(2i) $\\
  $U_i(n) = log_2(n)\lfloor n/2 \rfloor + (logn - log(i))T(2i)$\\
  Por lo cual queda:\\
  $U_i(n) = n + O(T(2i)\log(n/i))$\\
  
  \item Show that if $i$ is a constant less than $n/2$, then $U_i(n)=n+O(\log n)$.
  
  \textbf{Solución:}
  
  De la expresión anterior:\\
  $U_i(n) = n + O(T(2i)\log(n/i))$\\
  Si $i\leq n/2$ constante, se tiene que $T(2i) <= T(n) = \Theta (n)$.\\
  Y como i es pequeño, $\Theta (n) = O(1)$.\\
  Por lo tanto:\\
  $U_i(n) = n + O(T(2i)\log(n/i))$\\
  $U_i(n) = n + O(O(1)\log(n/i))$\\
  $U_i(n) = n + O(\log(n)-\log(i))$\\
  Como i es constante, log(i) es constante.\\
  $U_i(n) = n + O(\log(n)-O(1))$\\
  $U_i(n) = n + O(\log(n))$\\
  
  \item Show that if $i =n/k$ for $k\geq 2$, then $U_i(n) = n + O(T(2n/k) \log k)$.
  
  \textbf{Solución:}
  
  De la expresión anterior:\\
  $U_i(n) = n + O(T(2i)\log(n/i))$\\
  Como $i =n/k$ para $k\geq 2$\\
  $U_i(n) = n + O(T(2n/k)\log(n/(n/k)))$\\
  $U_i(n) = n + O(T(2n/k)\log(k))$\\
  
  
\end{enumerate}

\section{Comparison-based sorting}
Show that

$$
\log(n!) = \sum_{i=1}^{n}\log i = \Theta(n\log n)
$$

Use this to prove that any comparison-based sorting algorithm has worst-case time complexity $\Omega(n \log n)$\\

    \textbf{Solución:}

    Sabemos que:\\
    $log(n!)=\sum_{i=1}^{n} log(i) = log(n) + log(n-1) + ... + log(2) + log(1)$...(1)

    Supongamos que tenemos un arbol de desición de comparaciones.\\
    del tipo:\\    
    $a_1 < a_2 == True$, entonces evalua $a_2 < a_3$ sino $a_1 < a_3$.\\
    Y así sucesivamente ...\\
    El número total de hojas l en el árbol es al menos n!, que es el número de permutaciones.\\
    Un árbol que tiene k hojas, tendrá una altura de log(k).\\
    Por lo cual el árbol de comparaciones generado tiene una hoja w con altura de al menos log(n!), por lo que hace al menos log(n!) comparaciones.
    
    Volvemos a (1):\\
    $log(n!)=log(n) + log(n-1) + ... + log(2) + log(1) \leq log(n) + log(n) + ... + log(n) + log(n) = nlog(n)$\\
    Por lo tanto:\\
    log(n!) = O(nlog(n))...(a)\\
    
    Y también:\\
    $log(n!)=log(n) + log(n-1) + ... + log(2) + log(1) \geq log(n/2) + log(n/2+1) + ... + log((n-1)/2)+log(n)$\\
    $log(n!)\geq log(n/2) + log(n/2+1) + ... + log(n-1) +log(n)\geq log(n/2) + log(n/2) + ... + log(n/2)$\\
    $log(n!)\geq n/2 * log(n/2)$\\
    $log(n!)\geq n/2 * log(n)-nlog(n)$\\
    $log(n!)\geq n/2 * log(n)$\\
    Por lo tanto:\\
    log(n!) = $\Omega(nlog(n))$...(b)\\
    
    De (a) y (b) tenemos que:\\
    log(n!) = $\Theta (nlog(n))$
    
 
\end{document}