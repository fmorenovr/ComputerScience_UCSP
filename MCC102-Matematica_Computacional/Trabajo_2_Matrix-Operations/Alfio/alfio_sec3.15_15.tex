Sea una matriz no singular $A \in R^{n, n}$. Determine las condiciones en las que la relación
$\frac{\|y\|_2}{\|x\|_2}$, con x e y (tal como se muestra en la ecuación 3.70), aproximen $\|A^{-1}\|_2$.\\

\noindent \textcolor{red}{\bf Solución:}

La ecuación (3.70) denota lo siguiente:\\

Sea una matriz A definida como $Ay=d$, tal que $A = R^T R$ se tiene:
$$
  R^T x = d, Ry=d ...(1)
$$

Tomando esa ecuación como punto de partida:

Sea $A = U \Sigma V^T$, donde $U \Sigma V^T$ es Singular Value Descomposition de A.

Sea $u_i$ y $v_i$ las i-esimas columnas de U y V respectivamente.

Expandiendo el vector d en (3.70) en base a $v_i$ tenemos:\\

$d=\sum_{i=1}^{n} d_i v_i$ y también $x =\sum_{i=1}^{n} (d_i/\sigma_i) u_i$, $y=\sum_{i=1}^{n} (d_i/\sigma_i ^2)v_i$.\\

Donde $\sigma_1$, $\sigma_2$, ... ,$\sigma_n$ son los valores singulares de A.\\

Entonces la relación es:
$$
  \frac{\|y\|_2}{\|x\|_2} = \sqrt{\frac{\sum_{i=1}^{n} (d_i/\sigma_i) u_i}{\sum_{i=1}^{n} (d_i/\sigma_i ^2)v_i}} ...(2)
$$

De la ecuación (1), tenemos que:\\
El valor de y:
$$
  \|y\|_2 = \sqrt{\|R^{-1}\| \|d\|}
$$
El valor de x:
$$
  \|x\|_2 = \sqrt{\|(R^T)^{-1}\| \|d\|}
$$

Entonces de la ecuación (2) tenemos:
$$
  \frac{\|y\|_2}{\|x\|_2} = \sqrt{\frac{\|R^{-1}\|}{\|(R^T)^{-1}\|}}
$$

Multiplicando por $\|R^{-1}\|$:

$$
  \frac{\|y\|_2}{\|x\|_2} = \sqrt{\frac{\|R^{-1}\| \|R^{T}\|}{\|(R^T)^{-1}\| \|R^{T}\|}} = \sqrt{\frac{\|R^{-1}\| \|R^{T}\|}{K(R^{T})}}
$$
Pero como $K(R) = K(A)$:
$$
  \frac{\|y\|_2}{\|x\|_2} = \sqrt{\frac{(\frac{\|y\|_2}{\|x\|_2})}{K(A^T)}} 
$$
Entonces:

$$
  (\frac{\|y\|_2}{\|x\|_2})^2 = \frac{(\frac{\|y\|_2}{\|x\|_2})}{K(A^T)}
$$
Como A es simétrica, resulta:

$$
  \frac{\|y\|_2}{\|x\|_2} = \frac{1}{K(A)} \leq \|A^{-1}\|
$$

Y este resultado es igual a $\sigma_n ^{-1}$ el cual aproxima a $\|A^{-1}\|_2$.