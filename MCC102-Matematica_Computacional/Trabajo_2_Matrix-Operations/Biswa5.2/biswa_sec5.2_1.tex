\begin{enumerate}
    \item Mostrar que una matriz triangular inferior elemental tiene la forma:\\
    
        $$E = I + m_{k}e_{k}^{T}$$
        donde $m_{k}=(0,0,...,0,m_{k+1,k}, ..., m_{n,k})^{T} $
        
        Solución:
        
        \(
            A =
            \left( {\begin{array}{c}
                a_{1} \\
                a_{2} \\
                ... \\
                a_{n}\\
            \end{array} } \right)
        \)
        
        $$a_{1} \neq 0 $$
        
        tomamos también que $e_{i}^{T}m_{k} = 0$ para $i = 1, 2, ..., k$
        Esto podemos ver de la siguiente matriz:
        
        \[
        E =
            \left( {\begin{array}{cccccccc}
                1 & 0  & 0 & 0 & ... & 0 & 0 & 0\\
                0 & 1  & 0 & 0 & ... & 0 & 0 & 0\\
                0 & 0  & 1 & 0 & ... & 0 & 0 & 0\\
                ... & ... & ... & ... & ... & ... & ... & ...\\
                ... & ... &  & 0 & 1 & ... &  & ...\\
                0 & 0  & 0 & ... & m_{k+1,k} & ... & 0 & 0\\
                0 & 0  & 0 & ... & ... & ... & ... & 0\\
                0 & 0  & 0 & ... & m_{n,k} & ... & 0 & 1\\
            \end{array} } \right)
        \]
        
        
        Decimos que existe una matriz elementaria E tal que  $E_{a}$ es múltiplo de $e_{1}$
        \[
        E =
            \left( {\begin{array}{cccccc}
                1 & 0  & 0 & ... & ... & 0\\
                -\frac{a_{2}}{a_{1}} & 1  & 0 & ... & ... & 0\\
                ... & 0 & ... & ... & ... & 0 \\
                ... & ... & ... & ... & ... & ... \\
                ... & ... & ... & ... & ... & ... \\
                -\frac{a_{n}}{a_{1}} & 0 & 0 & ... & 0 & 1 \\
            \end{array} } \right)
        \]
        
        tiene la forma de una matriz triangular inferior tal que:
        
        \(
            E_{a} =
            \left( {\begin{array}{c}
                a_{1} \\
                0 \\
                0 \\
                ... \\
                ... \\
                0\\
            \end{array} } \right)
        \)
        
        De esta forma si juntamos todos los $m_{k}E_{k}^{T}, para k = 1 ... n$, obtenemos
    
        \[
        E =
            \left( {\begin{array}{cccccccc}
                1 & 0  & 0 & 0 & ... & 0 & 0 & 0\\
                m_{1,0} & 1  & 0 & 0 & ... & 0 & 0 & 0\\
                m_{2,0} & m_{2,1}  & 1 & 0 & ... & 0 & 0 & 0\\
                ... & ... & ... & 1 & ... & ... & ... & ...\\
                ... & ... & ... & ... & 1 & ... &  & ...\\
                ... & ...  & ... & ... & m_{k+1,k} & ... & 0 & 0\\
                ... & ...  & ... & ... & ... & ... & ... & 0\\
                m_{n,0} & m_{n,1}  & m_{n,2} & ... & m_{n,k} & ... & m_{n,n-1} & 1\\
            \end{array} } \right)
        \]
        
        Tiene la forma matriz triangular inferior.
        
    \item Mostrar que la inversa de  E en (a) es dada por:\\
    
        $$E^{-1} = I - m_{k}e_{k}^{T}$$
    
        Si suponemos que $$E^{-1} = I - m_{k}e_{k}^{T}$$ y $$E = I + m_{k}e_{k}^{T}$$, por tanto $$E * E^{-1} = I$$
        \begin{align*}
            E * E^{-1} &= I \\
            (I + m_{k}e_{k}^{T})*(I - m_{k}e_{k}^{T}) &= I\\
            I*I + I*(m_{k}e_{k}^{T}) - i*(m_{k}e_{k}^{T}) - (m_{k}e_{k}^{T}) *(m_{k}e_{k}^{T}) &= I\\
            I - (m_{k}e_{k}^{T}) *(m_{k}e_{k}^{T}) &= I\\
            - (m_{k}e_{k}^{T}) *(m_{k}e_{k}^{T}) &= I - I\\
            - (m_{k}e_{k}^{T}) *(m_{k}e_{k}^{T}) &= 0\\
        \end{align*}
\end{enumerate}





