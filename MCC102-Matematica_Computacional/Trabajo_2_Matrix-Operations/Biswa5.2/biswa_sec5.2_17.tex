\textbf{Dada la siguiente matriz}
\[ A =
    \left( \begin{array}{cccc}
    0 & 1 & 0 & 0\\
    0 & 0 & 1 & 0\\ 
    0 & 0 & 0 & 1\\ 
    2 & 3 & 4 & 5
    \end{array} \right) 
\]
\textbf{Encontrar una matriz de permutación P, una matriz triangular inferior unitaria L y una matriz triangular superior U tal que $PA = LU$} \\

\textbf{Solución}

La matriz de permutación P se forma a partir de los intercambios que se deben realizar entre filas  en la matriz A.\\

\textbf{Iteración 1}\\

La matriz de permutación $P_1 = I$
\[ P_1 =
    \left( \begin{array}{cccc}
    1 & 0 & 0 & 0\\
    0 & 1 & 0 & 0\\ 
    0 & 0 & 1 & 0\\ 
    0 & 0 & 0 & 1
    \end{array} \right) 
\]
En la matriz A, el pivote se encuentra en la fila 4, $r_1 = 4$, por lo tanto haremos el intercambio en la matriz $P_1$
\[ P_1 =
    \left( \begin{array}{cccc}
    0 & 0 & 0 & 1\\
    0 & 1 & 0 & 0\\ 
    0 & 0 & 1 & 0\\ 
    1 & 0 & 0 & 0
    \end{array} \right) 
\]
Calculamos $P_{1}A $
\[ P_{1}A =
    \left( \begin{array}{cccc}
    0 & 0 & 0 & 1\\
    0 & 1 & 0 & 0\\ 
    0 & 0 & 1 & 0\\ 
    1 & 0 & 0 & 0
    \end{array} \right)\left( \begin{array}{cccc}
    0 & 1 & 0 & 0\\
    0 & 0 & 1 & 0\\ 
    0 & 0 & 0 & 1\\ 
    2 & 3 & 4 & 5
    \end{array} \right) =
    \left( \begin{array}{cccc}
    2 & 3 & 4 & 5\\
    0 & 0 & 1 & 0\\ 
    0 & 0 & 0 & 1\\ 
    0 & 1 & 0 & 0
    \end{array} \right) 
\]
Calculamos la matriz $M_1$
\[ M_1 =
    \left( \begin{array}{cccc}
    1 & 0 & 0 & 0\\
    m_{2,1} & 1 & 0 & 0\\ 
    m_{3,1} & 0 & 1 & 0\\ 
    m_{4,1} & 0 & 0 & 1
    \end{array} \right) 
\]
Donde $m_{2,1} = \frac{-a_{2,1}}{a_{1,1}}$, $m_{3,1} = \frac{-a_{3,1}}{a_{1,1}}$ y $m_{4,1} = \frac{-a_{4,1}}{a_{1,1}}$ donde los valores de $a_{i,1}$ son obtenidos del resultado de $P_{1}A $ \\

 $m_{2,1} = \frac{0}{2} = 0$, $m_{3,1} = \frac{0}{2} = 0$ y $m_{4,1} = \frac{0}{2} = 0$
 \[ M_1 =
    \left( \begin{array}{cccc}
    1 & 0 & 0 & 0\\
    0 & 1 & 0 & 0\\ 
    0 & 0 & 1 & 0\\ 
    0 & 0 & 0 & 1
    \end{array} \right) 
\]
Calculamos la matriz $A^{(1)} = M_{1}P_{1}A$
\[ A^{(1)} =
    \left( \begin{array}{cccc}
    1 & 0 & 0 & 0\\
    0 & 1 & 0 & 0\\ 
    0 & 0 & 1 & 0\\ 
    0 & 0 & 0 & 1
    \end{array} \right) 
    \left( \begin{array}{cccc}
    0 & 0 & 0 & 1\\
    0 & 1 & 0 & 0\\ 
    0 & 0 & 1 & 0\\ 
    1 & 0 & 0 & 0
    \end{array} \right)
    \left( \begin{array}{cccc}
    0 & 1 & 0 & 0\\
    0 & 0 & 1 & 0\\ 
    0 & 0 & 0 & 1\\ 
    2 & 3 & 4 & 5
    \end{array} \right) 
\]
\[ A^{(1)} =
    \left( \begin{array}{cccc}
    2 & 3 & 4 & 5\\
    0 & 0 & 1 & 0\\ 
    0 & 0 & 0 & 1\\ 
    0 & 1 & 0 & 0
    \end{array} \right) 
\]
\textbf{Iteración 2}\\
$P_2 = I$
\[ P_2 =
     \left( \begin{array}{cccc}
    1 & 0 & 0 & 0\\
    0 & 1 & 0 & 0\\ 
    0 & 0 & 1 & 0\\ 
    0 & 0 & 0 & 1
    \end{array} \right)
\]
En la matriz $A^{(1)}$, el pivote se encuentra en la fila 4, $r_2 = 4$, por lo tanto haremos el intercambio de la fila 2 por la fila 4 en la matriz $P_2$
\[ P_2 =
    \left( \begin{array}{cccc}
    1 & 0 & 0 & 0\\
    0 & 0 & 0 & 1\\ 
    0 & 0 & 1 & 0\\ 
    0 & 1 & 0 & 0
    \end{array} \right) 
\]
Calculamos $P_{2}A^{(1)} $
\[ P_{2}A^{(1)} =
     \left( \begin{array}{cccc}
    1 & 0 & 0 & 0\\
    0 & 0 & 0 & 1\\ 
    0 & 0 & 1 & 0\\ 
    0 & 1 & 0 & 0
    \end{array} \right)\left( \begin{array}{cccc}
    2 & 3 & 4 & 5\\
    0 & 0 & 1 & 0\\ 
    0 & 0 & 0 & 1\\ 
    0 & 1 & 0 & 0
    \end{array} \right) =
    \left( \begin{array}{cccc}
    2 & 3 & 4 & 5\\
    0 & 1 & 0 & 0\\
    0 & 0 & 0 & 1\\ 
    0 & 0 & 1 & 0
    \end{array} \right) 
\]
Calculamos la matriz $M_2$
\[ M_2 =
    \left( \begin{array}{cccc}
    1 & 0 & 0 & 0\\
    0 & 1 & 0 & 0\\ 
    0 & m_{3,2} & 1 & 0\\ 
    0 & m_{4,2} & 0 & 1
    \end{array} \right) 
\]
Donde $m_{3,2} = \frac{-a_{3,2}}{a_{2,2}}$ y $m_{4,2} = \frac{-a_{4,2}}{a_{2,2}}$ donde los valores de $a_{i,2}$ son obtenidos del resultado de $P_{2}A^{(1)} $ \\
$m_{3,2} = \frac{0}{1} = 0$ y $m_{4,2} = \frac{0}{1} = 0$
\[ M_2 =
    \left( \begin{array}{cccc}
    1 & 0 & 0 & 0\\
    0 & 1 & 0 & 0\\ 
    0 & 0 & 1 & 0\\ 
    0 & 0 & 0 & 1
    \end{array} \right) 
\]
Calculamos la matriz $A^{(2)} = M_{2}P_{2}A^{(1)}$
\[ A^{(2)} =
    \left( \begin{array}{cccc}
    1 & 0 & 0 & 0\\
    0 & 1 & 0 & 0\\ 
    0 & 0 & 1 & 0\\ 
    0 & 0 & 0 & 1
    \end{array} \right) 
    \left( \begin{array}{cccc}
    1 & 0 & 0 & 0\\
    0 & 0 & 0 & 1\\ 
    0 & 0 & 1 & 0\\ 
    0 & 1 & 0 & 0
    \end{array} \right)
    \left( \begin{array}{cccc}
    2 & 3 & 4 & 5\\
    0 & 0 & 1 & 0\\ 
    0 & 0 & 0 & 1\\ 
    0 & 1 & 0 & 0
    \end{array} \right)
\]
\[ A^{(2)} =
    \left( \begin{array}{cccc}
    2 & 3 & 4 & 5\\
    0 & 1 & 0 & 0\\
    0 & 0 & 0 & 1\\
    0 & 0 & 1 & 0
    \end{array} \right) 
\]
\textbf{Iteración 3}\\
$P_3 = I$
\[ P_3 =
    \left( \begin{array}{cccc}
    1 & 0 & 0 & 0\\
    0 & 1 & 0 & 0\\ 
    0 & 0 & 1 & 0\\ 
    0 & 0 & 0 & 1
    \end{array} \right) 
\]
En la matriz $A^{(2)}$, el pivote se encuentra en la fila 4, $r_3 = 4$, por lo tanto haremos el intercambio de la fila 3 por la fila 4 en la matriz $P_3$
\[ P_3 =
    \left( \begin{array}{cccc}
    1 & 0 & 0 & 0\\
    0 & 1 & 0 & 0\\ 
    0 & 0 & 0 & 1\\ 
    0 & 0 & 1 & 0
    \end{array} \right) 
\]
Calculamos $P_{3}A^{(2)} $
\[ P_{3}A^{(2)} =
    \left( \begin{array}{cccc}
    1 & 0 & 0 & 0\\
    0 & 1 & 0 & 0\\ 
    0 & 0 & 0 & 1\\ 
    0 & 0 & 1 & 0
    \end{array} \right) \left( \begin{array}{cccc}
    2 & 3 & 4 & 5\\
    0 & 1 & 0 & 0\\
    0 & 0 & 0 & 1\\
    0 & 0 & 1 & 0
    \end{array} \right) =
    \left( \begin{array}{cccc}
    2 & 3 & 4 & 5\\
    1 & 0 & 0 & 0\\ 
    0 & 0 & 1 & 0\\
    0 & 0 & 0 & 1
    \end{array} \right) 
\]
Calculamos la matriz $M_3$
\[ M_3 =
    \left( \begin{array}{cccc}
    1 & 0 & 0 & 0\\
    0 & 1 & 0 & 0\\ 
    0 & 0 & 1 & 0\\ 
    0 & 0 & m_{4,3} & 1
    \end{array} \right) 
\]
Donde $m_{4,3} = \frac{-a_{4,3}}{a_{3,3}} = \frac{0}{1} = 0 $ 
\[ M_3 =
    \left( \begin{array}{cccc}
    1 & 0 & 0 & 0\\
    0 & 1 & 0 & 0\\ 
    0 & 0 & 1 & 0\\ 
    0 & 0 & 0 & 1
    \end{array} \right) 
\]
Calculamos la matriz $A^{(3)} = M_{3}P_{3}A^{(2)}$
\[ A^{(3)} =
    \left( \begin{array}{cccc}
    1 & 0 & 0 & 0\\
    0 & 1 & 0 & 0\\ 
    0 & 0 & 1 & 0\\ 
    0 & 0 & 0 & 1
    \end{array} \right) 
    \left( \begin{array}{cccc}
    1 & 0 & 0 & 0\\
    0 & 1 & 0 & 0\\ 
    0 & 0 & 0 & 1\\ 
    0 & 0 & 1 & 0
    \end{array} \right)
    \left( \begin{array}{cccc}
    2 & 3 & 4 & 5\\
    0 & 1 & 0 & 0\\
    0 & 0 & 0 & 1\\
    0 & 0 & 1 & 0
    \end{array} \right)
\]
\[ A^{(3)} =
    \left( \begin{array}{cccc}
    2 & 3 & 4 & 5\\
    0 & 1 & 0 & 0\\
    0 & 0 & 1 & 0\\
    0 & 0 & 0 & 1
    
    \end{array} \right) 
\]
La matriz de permutación esta dada por $P = P_{3}P_{2}P_{1}$
\[ P =
    \left( \begin{array}{cccc}
    1 & 0 & 0 & 0\\
    0 & 1 & 0 & 0\\ 
    0 & 0 & 0 & 1\\ 
    0 & 0 & 1 & 0
    \end{array} \right) 
    \left( \begin{array}{cccc}
    1 & 0 & 0 & 0\\
    0 & 0 & 0 & 1\\ 
    0 & 0 & 1 & 0\\ 
    0 & 1 & 0 & 0
    \end{array} \right)
    \left( \begin{array}{cccc}
    0 & 0 & 0 & 1\\
    0 & 1 & 0 & 0\\ 
    0 & 0 & 1 & 0\\ 
    1 & 0 & 0 & 0
    \end{array} \right)
\]
\[ P =
    \left( \begin{array}{cccc}
    0 & 0 & 0 & 1\\ 
    1 & 0 & 0 & 0\\ 
    0 & 1 & 0 & 0\\
    0 & 0 & 1 & 0
    \end{array} \right)
\]
La matriz triangular inferior unitario esta dada por $L = P(M_{3}P_{3}M_{2}P_{2}M_{1}P_{1})^{-1}$
\[ L =
    \left( \begin{array}{cccc}
    0 & 0 & 0 & 1\\ 
    1 & 0 & 0 & 0\\ 
    0 & 1 & 0 & 0\\
    0 & 0 & 1 & 0
    \end{array} \right)
    \left( \begin{array}{cccc}
    0 & 0 & 0 & 1\\
    1 & 0 & 0 & 0\\ 
    0 & 1 & 0 & 0\\ 
    0 & 0 & 1 & 0
    \end{array} \right)^{-1}
\]
\[ L =
    \left( \begin{array}{cccc}
    1 & 0 & 0 & 0\\
    0 & 1 & 0 & 0\\ 
    0 & 0 & 1 & 0\\ 
    0 & 0 & 0 & 1
    \end{array} \right)
\]
La matriz triangular superior esta dada por  $U = ^{(3)}$
\[ U =
    \left( \begin{array}{cccc}
    2 & 3 & 4 & 5\\
    0 & 1 & 0 & 0\\
    0 & 0 & 1 & 0\\
    0 & 0 & 0 & 1
    \end{array} \right)
\]

Se tiene que comprobar $PA = LU$
\[ PA  =  LU\]
\[ 
    \left( \begin{array}{cccc}
    0 & 0 & 0 & 1\\ 
    1 & 0 & 0 & 0\\ 
    0 & 1 & 0 & 0\\
    0 & 0 & 1 & 0
    \end{array} \right)
    \left( \begin{array}{cccc}
    0 & 1 & 0 & 0\\
    0 & 0 & 1 & 0\\ 
    0 & 0 & 0 & 1\\ 
    2 & 3 & 4 & 5
    \end{array} \right) = 
    \left( \begin{array}{cccc}
    1 & 0 & 0 & 0\\
    0 & 1 & 0 & 0\\ 
    0 & 0 & 1 & 0\\ 
    0 & 0 & 0 & 1
    \end{array} \right)
    \left( \begin{array}{cccc}
    2 & 3 & 4 & 5\\
    0 & 1 & 0 & 0\\
    0 & 0 & 1 & 0\\
    0 & 0 & 0 & 1
    \end{array} \right)
\]
\[ 
    \left( \begin{array}{cccc}
    2 & 3 & 4 & 5\\
    0 & 1 & 0 & 0\\
    0 & 0 & 1 & 0\\
    0 & 0 & 0 & 1
    \end{array} \right) = 
    \left( \begin{array}{cccc}
    2 & 3 & 4 & 5\\
    0 & 1 & 0 & 0\\
    0 & 0 & 1 & 0\\
    0 & 0 & 0 & 1
    \end{array} \right)
\]