\begin{enumerate}[]
    \item Calcule la factorización de Cholesky de: 
        \begin{equation}
            A = \begin{pmatrix}
                1 & 1     & 1     \\ 
                1 & 1.001 & 1.001 \\ 
                1 & 1.001 & 2
            \end{pmatrix}
        \end{equation}
        usando:
        \begin{enumerate}[]
            \item Eliminación Gaussiana sin pivote.
            \item Algoritmo de Cholesky.
        \end{enumerate}
    \item En la parte a) verificar que $\max |a^{(k)}_{ij}| \leq \max |a^{(k-1)}_{ij}|, \quad k=1,2$.
    \item ¿Cuál es el factor de crecimiento?
\end{enumerate}

\textbf{Solución:}

\begin{enumerate}[]
    \item Calcule la factorización de Cholesky de: 
        \begin{enumerate}[]
            
            % Parte (i)
            % .....................................
            \item Realizamos la eliminación Gaussiana sin pivote:
            
            \begin{equation}
                \begin{pmatrix}
                    1 & 1     & 1     \\ 
                    1 & 1.001 & 1.001 \\ 
                    1 & 1.001 & 2
                \end{pmatrix}
                \rightarrow 
                \begin{pmatrix}
                    1 & 1     & 1     \\ 
                    0 & 0.001 & 0.001 \\ 
                    0 & 0.001 & 1
                \end{pmatrix}
                \rightarrow 
                \begin{pmatrix}
                    1 & 1     & 1     \\ 
                    0 & 0.001 & 0.001 \\ 
                    0 & 0     & 0.999
                \end{pmatrix}
            \end{equation}

            % Parte (ii)
            % .....................................
            \item Siguiendo el algoritmo de Cholesky.
            
            Fila 1:
            \begin{equation}
                h_{11} = \sqrt{1} = 1
            \end{equation}
            
            Fila 2:
            \begin{equation}
                h_{21} = \frac{a_{21}}{h_{11}} = \frac{1}{1} = 1
            \end{equation}
            \begin{equation}
                h_{22} = \sqrt{a_{22} - h_{21}^2} = \sqrt{ 1.001 - 1} = 0.0316
            \end{equation}
            
            Fila 3:
            \begin{equation}
                h_{31} = \frac{a_{31}}{h_{11}} = \frac{1}{1} = 1 
            \end{equation}
            \begin{equation}
                h_{32} = \frac{1}{ h_{22} } \left( a_{32} - h_{21} h_{31} \right) 
                       = \frac{1}{ 0.0316 } \left(  1.001 - 1 \cdot 1 \right) = 0.0316
            \end{equation}
            \begin{equation}
                h_{33} = \sqrt{a_{33} - \left(h_{31}^2 + h_{31}^2\right)} = \sqrt{2 - \left(1^2 + 0.0316^2\right)} = \sqrt{0.999} = 0.9995
            \end{equation}
            
            Entonces, la factorización de Cholesky es:
            \begin{equation}
                A = \begin{pmatrix}
                    1 & 0      & 0      \\ 
                    1 & 0.0316 & 0      \\ 
                    1 & 0.0316 & 0.9995
                \end{pmatrix} \begin{pmatrix}
                    1 & 1      & 1      \\ 
                    0 & 0.0316 & 0.0316 \\ 
                    0 & 0      & 0.9995
                \end{pmatrix}
            \end{equation}
            
        \end{enumerate}
        
    \item En la parte a) verificar que $\max |a^{(k)}_{ij}| \leq \max |a^{(k-1)}_{ij}|, \quad k=1,2$.
    
    En el procedimiento realizado para la factorización de Cholesky se obtiene: $\max |a^{(0)}_{ij}| = 2$, $\max |a^{(1)}_{ij}| = 1$, $\max |a^{(2)}_{ij}| = 1$, $\max |a^{(3)}_{ij}| = 1$. Es decir:
    \begin{equation}
        \max |a^{(2)}_{ij}| \leq \max |a^{(1)}_{ij}| \leq \max |a^{(0)}_{ij}|
    \end{equation}
    
    \item ¿Cuál es el factor de crecimiento?
    
    El factor de crecimiento es definido como:
    \begin{equation}
        \rho = \frac{\max \left( \alpha, \alpha_1, \alpha_2, \dots, \alpha_{n-1} \right) }{max(a_{ij})}
    \end{equation}
    donde $\alpha = \underset{i,j}{\max} |a_{ij}|$ y $\alpha_k = \underset{i,j}{\max} |a_{ij}^{(k)}|$. Entonces en el problema: $\alpha = 2$, $\alpha_1 = 1$, $\alpha_2 = 1$, $\alpha_3 = 1$. Reemplazando: 
    \begin{equation}
        \rho = \frac{\max \left( 2, 1, 1, 1 \right) }{2} = 1
    \end{equation}
    
\end{enumerate}

