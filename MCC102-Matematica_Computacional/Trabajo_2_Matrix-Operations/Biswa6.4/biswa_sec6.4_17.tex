    Resolver el siguiente sistema
    \[
  	\begin{bmatrix}
    2 & -1 & 0 & 0 \\
    -1 & 2 & -1 & 0 \\
    0 & -1 & 2 & -1 \\
    0 & 0 & -1 & 2  
  	\end{bmatrix}
  	\begin{bmatrix}
    x_{1} \\
    x_{2} \\
    x_{3}  \\
    x_{4}   
  	\end{bmatrix}
  	=
  	\begin{bmatrix}
    1 \\
    1 \\
    1  \\
    1   
  	\end{bmatrix}
	\]\\
	\noindent \textcolor{red}{\bf Solución:}
	\begin{itemize}
		\item Usando Eliminación gaussiana
		La matriz aumentada es:
			\[
  			\begin{bmatrix}
   			 2 & -1 & 0 & 0 & | & 1 \\
    		-1 & 2 & -1 & 0 & | & 1\\
    		 0 & -1 & 2 & -1 & | & 1 \\
    		 0 & 0 & -1 & 2 & | & 1 
  			\end{bmatrix}
  			\]
  			Dividimos entre 2 la $1^{ra}$ fila 
  			\[
  			\begin{bmatrix}
   			 2 & -1 & 0 & 0 & | & 1 \\ 
    		-1 & 2 & -1 & 0 & | & 1\\
    		 0 & -1 & 2 & -1 & | & 1 \\
    		 0 & 0 & -1 & 2 & | & 1 
  			\end{bmatrix}
  			\]
  			\[
  			\begin{bmatrix}
   			 1 & -1/2 & 0 & 0 & | & 1/2 \\ 
    		-1 & 2 & -1 & 0 & | & 1\\
    		 0 & -1 & 2 & -1 & | & 1 \\
    		 0 & 0 & -1 & 2 & | & 1 
  			\end{bmatrix}
			\]	
  			Sumamos $F_{1}$ a $F_{2}$
			\[
  			\begin{bmatrix}
   			 1 & -1/2 & 0 & 0 & | & 1/2 \\ 
    		 0 & 3/2 & -1 & 0 & | & 3/2\\
    		 0 & -1 & 2 & -1 & | & 1 \\
    		 0 & 0 & -1 & 2 & | & 1 
  			\end{bmatrix}
  			\]
  			\[
  			\begin{bmatrix}
   			 1 & -1/2 & 0 & 0 & | & 1/2 \\ 
    		 0 & 3/2 & -1 & 0 & | & 3/2\\
    		 0 & -1 & 2 & -1 & | & 1 \\
    		 0 & 0 & -1 & 2 & | & 1 
  			\end{bmatrix}
  			\]
  			Multiplicamos por (2/3) la $2^{ra}$ fila 
  			\[
  			\begin{bmatrix}
   			 1 & -1/2 & 0 & 0 & | & 1/2 \\ 
    		 0 & 1 & -2/3 & 0 & | & 1\\
    		 0 & -1 & 2 & -1 & | & 1 \\
    		 0 & 0 & -1 & 2 & | & 1 
  			\end{bmatrix}
  			\]
  			Sumamos la $2^{da}$ fila a la $3^{ra}$
  			\[
  			\begin{bmatrix}
   			 1 & -1/2 & 0 & 0 & | & 1/2 \\ 
    		 0 & 1 & -2/3 & 0 & | & 1\\
    		 0 & 0 & 4/3 & -1 & | & 2 \\
    		 0 & 0 & -1 & 2 & | & 1 
  			\end{bmatrix}
  			\]
  			Multiplicamos por (3/4) la $3^{ra}$ fila 
			\[
  			\begin{bmatrix}
   			 1 & -1/2 & 0 & 0 & | & 1/2 \\ 
    		 0 & 1 & -2/3 & 0 & | & 1\\
    		 0 & 0 & 1 & -3/4 & | & 3/2 \\
    		 0 & 0 & -1 & 2 & | & 1 
  			\end{bmatrix}
  			\]	
  			Sumamos la $3^{ra}$ fila a la $4^{ta}$ fila 
  			\[
  			\begin{bmatrix}
   			 1 & -1/2 & 0 & 0 & | & 1/2 \\ 
    		 0 & 1 & -2/3 & 0 & | & 1\\
    		 0 & 0 & 1 & -3/4 & | & 3/2 \\
    		 0 & 0 & 0 & 5/4 & | & 5/2 
  			\end{bmatrix}
  			\]	
  			Multiplicamos la ultima fila por (4/5)
  			\[
  			\begin{bmatrix}
   			 1 & -1/2 & 0 & 0 & | & 1/2 \\ 
    		 0 & 1 & -2/3 & 0 & | & 1\\
    		 0 & 0 & 1 & -3/4 & | & 3/2 \\
    		 0 & 0 & 0 & 1 & | & 2 
  			\end{bmatrix}
  			\]	
  			Entonces tenemos lo siguiente:
  			\begin {equation*} \begin {split}  	
				x_{4} = 2 \\
				x_{3} - 3/4(x_{4}) = 3/2 \\
				x_{3} = 3 \\ 
				x_{2} - 2/3(x_{3}) = 1 \\
				x_{2} = 3 \\
				x_{1} - 1/2(x_{2}) = 1/2 \\
				x_{1} = 2 
  			\end {split} \end {equation*}
  			Por lo tanto, el sistema es el siguiente:
  			\[
  			\begin{bmatrix}
    		2 & -1 & 0 & 0 \\
    		-1 & 2 & -1 & 0 \\
    		0 & -1 & 2 & -1 \\
    		0 & 0 & -1 & 2  
  			\end{bmatrix}
  			\begin{bmatrix}
    		2 \\
    		3 \\
    		3 \\
    		2   
  			\end{bmatrix}
  			=
  			\begin{bmatrix}
    		1 \\
    		1 \\
    		1  \\
    		1   
  			\end{bmatrix}
			\]
		\item Calculando la factorización LU 
		  	\begin {equation*} \begin {split}  	
				m_{1}(2) = -A(2,1)/A(1,1) = 1/2 \\
				m_{1}(3) = -A(3,1)/A(1,1) = 0 \\
				m_{1}(4) = -A(4,1)/A(1,1) = 0  
  			\end {split} \end {equation*}
			\[ M_{1} = 
  			\begin{bmatrix}
   			 1 & 0 & 0 & 0 \\ 
    		 0 & 1 & 0 & 0 \\
    		 0 & 0 & 1 & 0 \\
    		 0 & 0 & 0 & 1  
  			\end{bmatrix}
  			+
  			\begin{bmatrix}
   			 0 \\ 
    		 1/2 \\
    		 0 \\
    		 0  
  			\end{bmatrix}
  			\begin{bmatrix}
   			 1 & 0 & 0 & 0 
  			\end{bmatrix}
  			=
  			\begin{bmatrix}
   			 1 & 0 & 0 & 0 \\ 
    		 1/2 & 1 & 0 & 0 \\
    		 0 & 0 & 1 & 0 \\
    		 0 & 0 & 0 & 1  
  			\end{bmatrix}
  			\]	
  			\[ A_{1} = 
  			\begin{bmatrix}
   			 1 & 0 & 0 & 0 \\ 
    		 1/2 & 1 & 0 & 0 \\
    		 0 & 0 & 1 & 0 \\
    		 0 & 0 & 0 & 1  
  			\end{bmatrix}
  			\begin{bmatrix}
    		2 & -1 & 0 & 0 \\
    		-1 & 2 & -1 & 0 \\
    		0 & -1 & 2 & -1 \\
    		0 & 0 & -1 & 2  
  			\end{bmatrix}
  			=
  			\begin{bmatrix}
    		2 & -1 & 0 & 0 \\
    		0 & 3/2 & -1 & 0 \\
    		0 & -1 & 2 & -1 \\
    		0 & 0 & -1 & 2  
  			\end{bmatrix}
  			\]
  			\[ MM_{1} = 
  			\begin{bmatrix}
   			 1 & 0 & 0 & 0 \\ 
    		 1/2 & 1 & 0 & 0 \\
    		 0 & 0 & 1 & 0 \\
    		 0 & 0 & 0 & 1  
  			\end{bmatrix}
  			\begin{bmatrix}
   			 1 & 0 & 0 & 0 \\ 
    		 0 & 1 & 0 & 0 \\
    		 0 & 0 & 1 & 0 \\
    		 0 & 0 & 0 & 1  
  			\end{bmatrix}
  			= 
  			\begin{bmatrix}
   			 1 & 0 & 0 & 0 \\ 
    		 1/2 & 1 & 0 & 0 \\
    		 0 & 0 & 1 & 0 \\
    		 0 & 0 & 0 & 1  
  			\end{bmatrix}
  			\]
  			\begin {equation*} \begin {split}  	
				m_{2}(3) = -A_{1}(3,2)/A_{1}(2,2) = 2/3 \\
				m_{2}(4) = -A_{1}(4,2)/A_{1}(2,2) = 0  
  			\end {split} \end {equation*}
  			\[ M_{2} = 
  			\begin{bmatrix}
   			 1 & 0 & 0 & 0 \\ 
    		 0 & 1 & 0 & 0 \\
    		 0 & 0 & 1 & 0 \\
    		 0 & 0 & 0 & 1  
  			\end{bmatrix}
  			+
  			\begin{bmatrix}
   			 0 \\ 
    		 0 \\
    		 2/3 \\
    		 0  
  			\end{bmatrix}
  			\begin{bmatrix}
   			 0 & 1 & 0 & 0 
  			\end{bmatrix}
  			=
  			\begin{bmatrix}
   			 1 & 0 & 0 & 0 \\ 
    		 0 & 1 & 0 & 0 \\
    		 0 & 2/3 & 1 & 0 \\
    		 0 & 0 & 0 & 1  
  			\end{bmatrix}
  			\]
  			\[ A_{2} = 
  			\begin{bmatrix}
   			 1 & 0 & 0 & 0 \\ 
    		 0 & 1 & 0 & 0 \\
    		 0 & 2/3 & 1 & 0 \\
    		 0 & 0 & 0 & 1  
  			\end{bmatrix}
			\begin{bmatrix}
    		2 & -1 & 0 & 0 \\
    		0 & 3/2 & -1 & 0 \\
    		0 & -1 & 2 & -1 \\
    		0 & 0 & -1 & 2  
  			\end{bmatrix}
  			=
  			\begin{bmatrix}
    		2 & -1 & 0 & 0 \\
    		0 & 3/2 & -1 & 0 \\
    		0 & 0 & 4/3 & -1 \\
    		0 & 0 & -1 & 2  
  			\end{bmatrix}
  			\]
  			\[ MM_{2} = 
  			\begin{bmatrix}
   			 1 & 0 & 0 & 0 \\ 
    		 0 & 1 & 0 & 0 \\
    		 0 & 2/3 & 1 & 0 \\
    		 0 & 0 & 0 & 1  
  			\end{bmatrix}
  			\begin{bmatrix}
   			 1 & 0 & 0 & 0 \\ 
    		 1/2 & 1 & 0 & 0 \\
    		 0 & 0 & 1 & 0 \\
    		 0 & 0 & 0 & 1  
  			\end{bmatrix}
  			= 
  			\begin{bmatrix}
   			 1 & 0 & 0 & 0 \\ 
    		 1/2 & 1 & 0 & 0 \\
    		 1/3 & 2/3 & 1 & 0 \\
    		 0 & 0 & 0 & 1  
  			\end{bmatrix}
  			\]
   			\begin {equation*} \begin {split}  	
				m_{3}(4) = -A_{2}(4,3)/A_{2}(3,3) = 3/4  
  			\end {split} \end {equation*} 
  			\[ M_{3} = 
  			\begin{bmatrix}
   			 1 & 0 & 0 & 0 \\ 
    		 0 & 1 & 0 & 0 \\
    		 0 & 0 & 1 & 0 \\
    		 0 & 0 & 0 & 1  
  			\end{bmatrix}
  			+
  			\begin{bmatrix}
   			 0 \\ 
    		 0 \\
    		 0 \\
    		 3/4  
  			\end{bmatrix}
  			\begin{bmatrix}
   			 0 & 0 & 1 & 0 
  			\end{bmatrix}
  			=
  			\begin{bmatrix}
   			 1 & 0 & 0 & 0 \\ 
    		 0 & 1 & 0 & 0 \\
    		 0 & 0 & 1 & 0 \\
    		 0 & 0 & 3/4 & 1  
  			\end{bmatrix}
  			\]	
  			\[ A_{3} = 
  			\begin{bmatrix}
   			 1 & 0 & 0 & 0 \\ 
    		 0 & 1 & 0 & 0 \\
    		 0 & 0 & 1 & 0 \\
    		 0 & 0 & 3/4 & 1  
  			\end{bmatrix}
			\begin{bmatrix}
    		2 & -1 & 0 & 0 \\
    		0 & 3/2 & -1 & 0 \\
    		0 & 0 & 4/3 & -1 \\
    		0 & 0 & -1 & 2  
  			\end{bmatrix}
  			=
  			\begin{bmatrix}
    		2 & -1 & 0 & 0 \\
    		0 & 3/2 & -1 & 0 \\
    		0 & 0 & 4/3 & -1 \\
    		0 & 0 & 0 & 5/4  
  			\end{bmatrix}
  			\]	
  			\[ MM_{3} = 
  			\begin{bmatrix}
   			 1 & 0 & 0 & 0 \\ 
    		 0 & 1 & 0 & 0 \\
    		 0 & 0 & 1 & 0 \\
    		 0 & 0 & 3/4 & 1  
  			\end{bmatrix}
  			\begin{bmatrix}
   			 1 & 0 & 0 & 0 \\ 
    		 1/2 & 1 & 0 & 0 \\
    		 1/3 & 2/3 & 1 & 0 \\
    		 0 & 0 & 0 & 1  
  			\end{bmatrix}
  			= 
  			\begin{bmatrix}
   			 1 & 0 & 0 & 0 \\ 
    		 1/2 & 1 & 0 & 0 \\
    		 1/3 & 2/3 & 1 & 0 \\
    		 1/4 & 1/2 & 3/4 & 1  
  			\end{bmatrix}
  			\]
  			Entonces:
			\[ L =   			
  			Inv(\begin{bmatrix}
   			 1 & 0 & 0 & 0 \\ 
    		 1/2 & 1 & 0 & 0 \\
    		 1/3 & 2/3 & 1 & 0 \\
    		 1/4 & 1/2 & 3/4 & 1  
  			\end{bmatrix})
  			\]
			\[ L =   			
  			\begin{bmatrix}
   			 1 & 0 & 0 & 0 \\ 
    		 -1/2 & 1 & 0 & 0 \\
    		 0 & -2/3 & 1 & 0 \\
    		 0 & 0 & -3/4 & 1  
  			\end{bmatrix}
  			\] 			
  			\[ U =   			
  			\begin{bmatrix}
   			 1 & 0 & 0 & 0 \\ 
    		 1/2 & 1 & 0 & 0 \\
    		 1/3 & 2/3 & 1 & 0 \\
    		 1/4 & 1/2 & 3/4 & 1  
  			\end{bmatrix}
  			\begin{bmatrix}
    		2 & -1 & 0 & 0 \\
    		-1 & 2 & -1 & 0 \\
    		0 & -1 & 2 & -1 \\
    		0 & 0 & -1 & 2  
  			\end{bmatrix}
  			\]
  			\[ U =   			
  			\begin{bmatrix}
   			 2 & -1 & 0 & 0 \\ 
    		 0 & 3/2 & -1 & 0 \\
    		 0 & 0 & 4/3 & -1 \\
    		 0 & 0 & 0 & 5/4  
  			\end{bmatrix}
  			\]
  			Sabemos que $Ux = y$, donde  $y = L^{-1}b$
  			\[ y =   			
  			\begin{bmatrix}
   			 1 & 0 & 0 & 0 \\ 
    		 1/2 & 1 & 0 & 0 \\
    		 1/3 & 2/3 & 1 & 0 \\
    		 1/4 & 1/2 & 3/4 & 1  
  			\end{bmatrix}
  			\begin{bmatrix}
   			 1  \\ 
    		 1 \\
    		 1 \\
    		 1  
  			\end{bmatrix}
  			=
  			\begin{bmatrix}
   			 1 \\ 
    		 3/2 \\
    		 2 \\
    		 5/2  
  			\end{bmatrix}
  			\] 
  			Entonces $x = U^{-1}y$:		
  			\[ x =   			
  			\begin{bmatrix}
   			 1/2 & 1/3 & 1/4 & 1/5 \\ 
    		 0 & 2/3 & 1/2 & 2/5 \\
    		 0 & 0 & 3/4 & 3/5 \\
    		 0 & 0 & 0 & 4/5  
  			\end{bmatrix}
  			\begin{bmatrix}
   			 1  \\ 
    		 3/2 \\
    		 2 \\
    		 5/2  
  			\end{bmatrix}
  			=
  			\begin{bmatrix}
   			 2 \\ 
    		 3 \\
    		 3 \\
    		 2  
  			\end{bmatrix}
  			\] 
	\end{itemize}
	Por ambos métodos demostramos que el sistema tiene la misma solución.