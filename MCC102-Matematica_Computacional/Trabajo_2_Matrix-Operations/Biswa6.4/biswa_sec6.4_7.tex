Resolver el sistema lineal $Ax=b$, donde b es un vector con cada componente igual a 1 y con A del ejercicio 18 del capitulo 5,usando:\\
%wilderd

\renewcommand{\labelenumi}{\alph{enumi}}
\noindent \textcolor{red}{\bf Solución:}
\begin{enumerate}
\item Eliminación Gausianna sin pivoteo 

%	\textbf{a} ELIMINACION DE GAUSS SIN PIVOTEO 

\begin{align*}
	(i) A =
\begin{bmatrix}
1 &\frac{1}{2} &\frac{1}{3} \\[6pt]
\frac{1}{2} &\frac{1}{3} &\frac{1}{4} \\[6pt]
\frac{1}{3} &\frac{1}{4} &\frac{1}{5} 
\end{bmatrix}
b = 
\begin{bmatrix}
1\\[6pt]
1\\[6pt]
1
\end{bmatrix}
\qquad
x =
\begin{bmatrix}
	x\\[6pt]
	y\\[6pt]
	z
\end{bmatrix}
\end{align*}
paso 1) para resolver esto aplicamos el metodo de gaus escalonado aplicando para el elemento $a_{21}$  con $m_{21} = -\frac{1}{2}$ tenemos el siguiente resultado:
\begin{align*}
A=
\begin{bmatrix}
1 &\frac{1}{2} &\frac{1}{3} \\[6pt]
0 &\frac{1}{3} &\frac{1}{4} \\[6pt]
\frac{1}{3} &\frac{1}{4} &\frac{1}{5} 
\end{bmatrix}
\end{align*}
tambien aplicando el metodo  $a_{31} = - \frac{1}{3}$

\begin{align*}
A=
\begin{bmatrix}
1 &\frac{1}{2} &\frac{1}{3} \\[6pt]
0 &\frac{1}{3} &\frac{1}{4} \\[6pt]
0 &\frac{1}{4} &\frac{1}{5} 
\end{bmatrix}
\end{align*}
paso 2) ahora podemos tambien aplicar para $a_{32} = -1$, que tambien se convertira en U
\begin{align*}
A=
\begin{bmatrix}
1 &\frac{1}{2} &\frac{1}{3} \\[6pt]
0 &\frac{1}{3} &\frac{1}{4} \\[6pt]
0 &0 &\frac{1}{5} 
\end{bmatrix}
\qquad U =
\begin{bmatrix}
1 &\frac{1}{2} &\frac{1}{3} \\[6pt]
0 &\frac{1}{3} &\frac{1}{4} \\[6pt]
0 &0 &\frac{1}{5} 
\end{bmatrix}
\end{align*}
ahora mostramos la matriz L de lo anterior

\begin{align*}
L =
\begin{bmatrix}
1	&0	&0 \\[6pt]
\frac{1}{2}	&1	&0 \\[6pt]
\frac{1}{3} &1 	&1
\end{bmatrix}
\end{align*}
por la factorizacion LU anterior y el sistema lineal $Ax = b$ tenemos que $LY = b$, haciendo esto para obtener los Y tenemos que :

\begin{align*}
\begin{bmatrix}
1	&0	&0 \\[6pt]
\frac{1}{2}	&1	&0 \\[6pt]
\frac{1}{3} &1 	&1
\end{bmatrix}
\begin{bmatrix}
y_1\\[6pt]
y_2\\[6pt]
y_3
\end{bmatrix}
=
\begin{bmatrix}
1\\[6pt]
1\\[6pt]
1
\end{bmatrix}
\end{align*}
tenemos Y 
\begin{align*}
Y = 
\begin{bmatrix}
1\\[6pt]
\frac{1}{2}\\[6pt]
\frac{1}{6}
\end{bmatrix}
\end{align*}
ahora para calcular \textbf{UX = Y} hacemos la multiplicacion y calculamos los valores de x, de la siguiente forma.
\begin{align*}
\begin{bmatrix}
1	&\frac{1}{2} &\frac{1}{3} \\[6pt]
0	&\frac{1}{12}&\frac{1}{12}\\[6pt]
0	&0			 &\frac{4}{45} 
\end{bmatrix}
\begin{bmatrix}
x\\[6pt]
y\\[6pt]
z
\end{bmatrix}
\end{align*}
de lo anterior podemos deducir facilmente la que los valores de X son $x = 1-\frac{33}{16}-\frac{15}{24}, \qquad y = 6-\frac{15}{8}=\frac{33}{8}, z = \frac{15}{8}$
\item Eliminacion Gausiana con pivoteo parcial y pivoteo total

\textbf{a pivoteo parcial}
\begin{align*}
	(i) A =
\begin{bmatrix}
1 &\frac{1}{2} &\frac{1}{3} \\[6pt]
\frac{1}{2} &\frac{1}{3} &\frac{1}{4} \\[6pt]
\frac{1}{3} &\frac{1}{4} &\frac{1}{5} 
\end{bmatrix}
\end{align*}

	usando el pivoteo parcial vamos a expresar A = MU, para encontrar P y L de tal forma que PA = LU.
	\textbf{paso 1} el pivote parcial entra con $a_{11}= 1$ y $r = 1$
	
	\begin{align*}
		P_{1} &= 
		\begin{bmatrix}
			1 &0 &0\\
			0 &1 &0\\
			0 &0 &1		
		\end{bmatrix}\\
		p_{1}A&=
		\begin{bmatrix}
			 1   &0.50000   &0.33333\\
   			 0.50000   &0.33333   &0.25000\\
  			 0.33333   &0.25000   &0.20000
		\end{bmatrix}\\
		M_1 &=
		\begin{bmatrix}
			1 &0 &0\\
			-\frac{1}{2} &1 &0\\
			-\frac{1}{3} &0 &1			
		\end{bmatrix}\\
		a^{ ( 1 ) } &=M_1P_1A = 
		\begin{bmatrix}
			1   &0.50000   &0.33333\\
		   0   &0.08333   &0.08333\\
		   0   &0.08333   &0.08889		
		\end{bmatrix}
	\end{align*}
	\textbf{paso 2}, ahora el pivote entrante es $a_{22} = 0.833$
	\begin{align*}
		P_2A^{( 1 )} &= 
		\begin{bmatrix}
			1   &0.50000   &0.33333\\
		   0   &0.08333   &0.08333\\
		   0   &0.08333   &0.08889
		\end{bmatrix}
		\\ P_2 &= I_3\\
		M_2 &= 
		\begin{bmatrix}
			1 &0 &0 \\
			0 &1 &0 \\
			0 &-1 &1		
		\end{bmatrix}\\
		U &= A^{( 2 )} =M_2P_2A^{( 1 )} =
		\begin{bmatrix}
		1   &0.50000  & 0.33333\\
  		0   &0.08333   &0.08333\\
	    0   &0		   &0.00556
		\end{bmatrix}\\
		M &= M_2P_2M_1P_1=
		\begin{bmatrix}
			1   &0   &0\\
		  -0.50000   &1   &0\\
		   0.16667   &0   &0
		\end{bmatrix}
	\end{align*}
	para la matriz L hacemos
	\begin{align*}
		P &= P_2P_1=
		\begin{bmatrix}
			1 &0 &0\\
			0 &1 &0\\
			0 &0 &1
		\end{bmatrix}\\
		L &= P(M_2P_2M_1P_1)^{-1} =
		\begin{bmatrix}
			1  &-0.50000   &0.16667\\
		    0   &1   &0\\
		    0   &0   &0
		\end{bmatrix}
	\end{align*}
	
	por la factorizacion LU anterior y el sistema lineal Ax = b, tenemos que LY = b, haciendo este ultimo para obtener los Y tenemos:
	\begin{align*}
		LY &= 
		\begin{bmatrix}
			1  &-0.50000   &0.16667\\
		    0   &1   &0\\
		    0   &0   &0
		\end{bmatrix}
		\begin{bmatrix}
			y_1\\
			y_2\\
			y_3
		\end{bmatrix}
		=
		\begin{bmatrix}
			1\\
			1\\
			1
		\end{bmatrix}
	\end{align*}
	de lo anterior podemos obtener $y_3 = 0, \qquad y_2 = 1; y1 = 1$, entonces podemos hacer UX = Y , y calculamos los valores de x.
	\begin{align*}
		\begin{bmatrix}
		1   &0.50000  & 0.33333\\
  		0   &0.08333   &0.08333\\
	    0   &0		   &0.00556
		\end{bmatrix}
		\begin{bmatrix}
		x\\
		y\\
		z
		\end{bmatrix} 
		=
		\begin{bmatrix}
			1\\
			1\\
			1
		\end{bmatrix}
	\end{align*}
	de donde podemos inferir $z = \frac{1}{0.00556}, \qquad y = \frac{1-14.9874}{0.8333}, x = 1-0.5y-0.333z$
	
\textbf{b} pivoteo total

\begin{align*}
	(i) A =
\begin{bmatrix}
1 &\frac{1}{2} &\frac{1}{3} \\[6pt]
\frac{1}{2} &\frac{1}{3} &\frac{1}{4} \\[6pt]
\frac{1}{3} &\frac{1}{4} &\frac{1}{5} 
\end{bmatrix}
\end{align*}
el pivoteo total elige el mayor elemento dentro de un matriz para asi poder elegir el pivote con el mayor elemento en le primer elemento y la primera columna. y asi hacemos dentro de cada sub matriz obteniendo el mayor elemento en la diagonal principal.

en este ejemplo no podemos
\begin{align*}
A &=
\left(\begin{array}{ccc|c}  
  1          &\frac{1}{2} & \frac{1}{3} & 1 \\[6pt]  
  \frac{1}{2}&\frac{1}{3} & \frac{1}{4} & 1 \\[6pt] 
  \frac{1}{3}&\frac{1}{4} & \frac{1}{5} & 1  
\end{array}\right)\\
\end{align*}
en la anterior reduccion no existe intercambio debido aque 1 es el mayor elemento dentro de nuestra matriz, asi tambien para reducir tenemos nuestro pivote al 1.
\begin{align*}
A &=
\left(\begin{array}{ccc|c}  
  1          &\frac{1}{2} & \frac{1}{3} & 1 \\[6pt]  
  0     &\frac{1}{4} & \frac{1}{12} & \frac{1}{2} \\[6pt] 
  \frac{1}{3}&\frac{1}{4} & \frac{1}{5} & 1  
\end{array}\right)\\
\end{align*}
ahora reduciremos el elemento $a_{31}$ de la siguiente forma
\begin{align*}
A &=
\left(\begin{array}{ccc|c}  
  1          &\frac{1}{2} & \frac{1}{3} & 1 \\[6pt]  
  0     &\frac{1}{4} & \frac{1}{12} & \frac{1}{2} \\[6pt] 
  0     &\frac{1}{12} & \frac{4}{5} & \frac{2}{3}  
\end{array}\right)\\
\end{align*}
ahora pasaremos a reducir el en forma escalonada el elemento $a_{32}$
\begin{align*}
A &=
\left(\begin{array}{ccc|c}  
  1          &\frac{1}{2} & \frac{1}{3} & 1 \\[6pt]  
  0     &\frac{1}{4} & \frac{1}{12} & \frac{1}{2} \\[6pt] 
  0     &0 & \frac{1}{180} & \frac{1}{2}  
\end{array}\right)\\
\end{align*}

ahora resolviendo tendremos que 
\begin{align*}
x &=-15\\
y &=-28\\
z &= 90 
\end{align*}
\item Factorizacion QR
aplicando el metodo de factorizacion QR tenemos
\begin{align*}
	(i) A =
\begin{bmatrix}
1 &\frac{1}{2} &\frac{1}{3} \\[6pt]
\frac{1}{2} &\frac{1}{3} &\frac{1}{4} \\[6pt]
\frac{1}{3} &\frac{1}{4} &\frac{1}{5} 
\end{bmatrix}
\end{align*}
dado el sistema Ax = b si $A= QR \rightarrow QRx = b$, asi $Rx = Q^t b$
entonces primero tenemos que hallar Q y R.
asi primero dividimos la matriz en pequeños bloques verticales de vectores.
\begin{align*}
s=
\left\lbrace 
\begin{bmatrix}
1\\[6pt]
\frac{1}{2}\\[6pt]
\frac{1}{3}
\end{bmatrix}
,
\begin{bmatrix}
\frac{1}{2}\\[6pt]
\frac{1}{3}\\[6pt]
\frac{1}{4}
\end{bmatrix}
,
\begin{bmatrix}
\frac{1}{3}\\[6pt]
\frac{1}{4}\\[6pt]
\frac{1}{5}
\end{bmatrix}
\right\rbrace
=\{ v_1, v_2, v_3\}
\end{align*}
acontinuacion las ecuaciones que debemos hallar son
\begin{align*}
U_1 &= V_1\\
U_2 &= V_2 - \alpha_{12}U_1\\
U_3 &= V_3 -\alpha_{13}U_1-\alpha_{23}U_2
\end{align*}
hallando $U_2$ tenemos $\alpha_{12}=\frac{<U_1,V_2>}{\parallel U_1\parallel} = \frac{49}{27}$
\begin{align*}
U_2 =
\begin{bmatrix}
\frac{1}{2}\\[6pt]
\frac{1}{3}\\[6pt]
\frac{1}{4}
\end{bmatrix}
- \alpha_{12}
\begin{bmatrix}
1\\[6pt]
\frac{1}{2}\\[6pt]
\frac{1}{3}
\end{bmatrix}
\\
U_2 =
\begin{bmatrix}
-1,314\\
-0,5740\\
-0,3548
\end{bmatrix}
\end{align*}
ahora calculamos $U_3$ con $\alpha_{13}=\frac{<U_1,V_3>}{\parallel U_1\parallel} = 0,525$, asi tambien$\alpha_{23}=\frac{<U_2,V_3>}{\parallel U_2\parallel} = 0.3$
\begin{align*}
U_3 &= 
\begin{bmatrix}
\frac{1}{3}\\[6pt]
\frac{1}{4}\\[6pt]
\frac{1}{5}
\end{bmatrix}-
\alpha_{13}U_1-\alpha_{23}U_2 \\
U_3 &= 
\begin{bmatrix}
0,615\\
3,7153\\
0,2314
\end{bmatrix}
\end{align*}
asi ahora tenemos  el Q 
\begin{align*}
Q =
\begin{bmatrix}
U_1 &U_2 &U_3
\end{bmatrix}
=
\begin{bmatrix}
1			&-1,314	 &0,615 \\[6pt]
\frac{1}{2}	&-0,5740 &3,7153\\[6pt]
\frac{1}{3} &-0,3548 &0,2314
\end{bmatrix}
\end{align*}
ademas ahora podemos calcular R.
\begin{align*}
R=
\begin{bmatrix}
1	&1,8148  &0,525\\
0   &1       &0.3\\
0   &0       &1
\end{bmatrix}
\end{align*}

\end{enumerate}