a) Resuelva cada sistema del Ejercicio 7 usando pivoteo parcial pero sin factorización explícita.
%====================================================================================
\\
\\
\noindent \textcolor{red}{\bf Solución:}\\    
a.1) Para el sistema de ecuación:\\
\begin{equation*}
    \begin{pmatrix}
        1 & 1/2 & 1/3 \\
        1/2 & 1/3 & 1/4 \\
        1/3 & 1/4 & 1/5
    \end{pmatrix}
    \begin{pmatrix}
        x_1 \\
        x_2 \\
        x_3
    \end{pmatrix}    
    =    
    \begin{pmatrix}
        1 \\
        1 \\
        1
    \end{pmatrix}
\end{equation*}

\begin{equation*}
    \begin{matrix}
        A=\begin{pmatrix}
            1 & 1/2 & 1/3 \\
            1/2 & 1/3 & 1/4 \\
            1/3 & 1/4 & 1/5
        \end{pmatrix}
        & , &
        X=\begin{pmatrix}
            x_1 \\
            x_2 \\
            x_3
        \end{pmatrix} 
        & , &
        b=\begin{pmatrix}
            1 \\
            1 \\
            1
        \end{pmatrix}
    \end{matrix}
\end{equation*}

Tomando: $m_{21}=-\frac{a_{21}}{a_{11}}=-1/2$ y $m_{31}=-\frac{a_{31}}{a_{11}}=-1/3$

\begin{equation*}
    \begin{matrix}
        A^{(1)}= \begin{pmatrix}
            1 & 1/2 & 1/3 \\
            0 & 0,08333 & 0,08333 \\
            0 & 0,08333 & 0,08888
        \end{pmatrix}
        & , &
        b^{(1)}=\begin{pmatrix}
            1 \\
            0,5 \\
            0,66666
        \end{pmatrix}
    \end{matrix}
\end{equation*}

Tomando: $m_{32}=-\frac{a_{32}}{a_{22}}=-1$

\begin{equation*}
    \begin{matrix}
        A^{(2)}= \begin{pmatrix}
            1 & 1/2 & 1/3 \\
            0 & 0,08333 & 0,08333 \\
            0 & 0 & 0,00555
        \end{pmatrix}
        & , &
        b^{(2)}=\begin{pmatrix}
            1 \\
            0,5 \\
            0,16666
        \end{pmatrix}
    \end{matrix}
\end{equation*}

Obteniendo el sistema de ecuaciones resultante:
\begin{align*}
    1x_1+0,5x_2+0,33333x_3 &= 1 \\
    0,08333x_2+0,08333x_3 &= 0,5 \\
    0,00555x_3 &= 0,16666
\end{align*}

Y sus respuestas son: $x_3=30,02882$ , $x_2=-24,02859$ y $x_1=3,00479$\\

%====================================================================================
a.2) Para el sistema de ecuación:\\
\begin{equation*}
    \begin{pmatrix}
        100 & 99 & 98 \\
        98 & 55 & 11 \\
        0 & 1 & 1
    \end{pmatrix}
    \begin{pmatrix}
        x_1 \\
        x_2 \\
        x_3
    \end{pmatrix}    
    =    
    \begin{pmatrix}
        1 \\
        1 \\
        1
    \end{pmatrix}
\end{equation*}

\begin{equation*}
    \begin{matrix}
        A=\begin{pmatrix}
            100 & 99 & 98 \\
            98 & 55 & 11 \\
            0 & 1 & 1
        \end{pmatrix}
        & , &
        X=\begin{pmatrix}
            x_1 \\
            x_2 \\
            x_3
        \end{pmatrix} 
        & , &
        b=\begin{pmatrix}
            1 \\
            1 \\
            1
        \end{pmatrix}
    \end{matrix}
\end{equation*}

Tomando: $m_{21}=-\frac{a_{21}}{a_{11}}=-0,98$

\begin{equation*}
    \begin{matrix}
        A^{(1)}= \begin{pmatrix}
            100 & 99 & 98 \\
            0 & -42,02 & -85,04 \\
            0 & 1 & 1
        \end{pmatrix}
        & , &
        b^{(1)}=\begin{pmatrix}
            1 \\
            0,02 \\
            1
        \end{pmatrix}
    \end{matrix}
\end{equation*}

Tomando: $m_{32}=-\frac{a_{32}}{a_{22}}=0.0238$

\begin{equation*}
    \begin{matrix}
        A^{(2)}= \begin{pmatrix}
            100 & 99 & 98 \\
            0 & -42,02 & -85,04 \\
            0 & 0 & -1,02395
        \end{pmatrix}
        & , &
        b^{(2)}=\begin{pmatrix}
            1 \\
            0,02 \\
            1,00048
        \end{pmatrix}
    \end{matrix}
\end{equation*}

Obteniendo el sistema de ecuaciones resultante:
\begin{align*}
    100x_1+99x_2+98x_3 &= 1 \\
    -42,02x_2+-85,04x_3 &= 0,02 \\
    -1,02395x_3 &= 1,00048
\end{align*}

Y sus respuestas son: $x_3=-0,97707$ , $x_2=1,97691$ y $x_1=-0,98961$\\

%====================================================================================
a.3) Para el sistema de ecuación:\\
\begin{equation*}
    \begin{pmatrix}
        1 & 0 & 1 \\
        -1 & 1 & 1 \\
        -1 & -1 & 1
    \end{pmatrix}
    \begin{pmatrix}
        x_1 \\
        x_2 \\
        x_3
    \end{pmatrix}    
    =    
    \begin{pmatrix}
        1 \\
        1 \\
        1
    \end{pmatrix}
\end{equation*}

\begin{equation*}
    \begin{matrix}
        A=\begin{pmatrix}
            1 & 0 & 1 \\
            -1 & 1 & 1 \\
            -1 & -1 & 1
        \end{pmatrix}
        & , &
        X=\begin{pmatrix}
            x_1 \\
            x_2 \\
            x_3
        \end{pmatrix} 
        & , &
        b=\begin{pmatrix}
            1 \\
            1 \\
            1
        \end{pmatrix}
    \end{matrix}
\end{equation*}

Tomando: $m_{21}=-\frac{a_{21}}{a_{11}}=1$

\begin{equation*}
    \begin{matrix}
        A^{(1)}= \begin{pmatrix}
            1 & 0 & 1 \\
            0 & 1 & 2 \\
            0 & -1 & 2
        \end{pmatrix}
        & , &
        b^{(1)}=\begin{pmatrix}
            1 \\
            2 \\
            2
        \end{pmatrix}
    \end{matrix}
\end{equation*}

Tomando: $m_{32}=-\frac{a_{32}}{a_{22}}=1$

\begin{equation*}
    \begin{matrix}
        A^{(2)}= \begin{pmatrix}
            1 & 0 & 1 \\
            0 & 1 & 2 \\
            0 & 0 & 4
        \end{pmatrix}
        & , &
        b^{(2)}=\begin{pmatrix}
            1 \\
            2 \\
            4
        \end{pmatrix}
    \end{matrix}
\end{equation*}

Obteniendo el sistema de ecuaciones resultante:
\begin{align*}
    x_1+x_3 &=1 \\
    x_2+2x_3 &=2 \\
    4x_3 &=4
\end{align*}

Y sus respuestas son: $x_3=1$ , $x_2=0$ y $x_1=0$\\

%====================================================================================
a.4) Para el sistema de ecuación:\\
\begin{equation*}
    \begin{pmatrix}
        0,0003 & 1,566 & 1,234 \\
        1,566 & 2,0 & 1,018 \\
        1,234 & 1,018 & -3,0
    \end{pmatrix}
    \begin{pmatrix}
        x_1 \\
        x_2 \\
        x_3
    \end{pmatrix}    
    =    
    \begin{pmatrix}
        1 \\
        1 \\
        1
    \end{pmatrix}
\end{equation*}

\begin{equation*}
    \begin{matrix}
        A=\begin{pmatrix}
            0,0003 & 1,566 & 1,234 \\
            1,566 & 2,0 & 1,018 \\
            1,234 & 1,018 & -3,0
        \end{pmatrix}
        & , &
        X=\begin{pmatrix}
            x_1 \\
            x_2 \\
            x_3
        \end{pmatrix} 
        & , &
        b=\begin{pmatrix}
            1 \\
            1 \\
            1
        \end{pmatrix}
    \end{matrix}
\end{equation*}
Por el pivoteo parcial cambiamos las filas 1ra y 2da:\\
\begin{equation*}
    \begin{matrix}
        A\equiv \begin{pmatrix}
            1,566 & 2,0 & 1,018 \\
            0,0003 & 1,566 & 1,234 \\
            1,234 & 1,018 & -3,0
        \end{pmatrix}
        & , &
        b\equiv \begin{pmatrix}
            1 \\
            1 \\
            1
        \end{pmatrix}
    \end{matrix}
\end{equation*}

Tomando: $m_{21}=-\frac{a_{21}}{a_{11}}=-0,000196$ y $m_{31}=-\frac{a_{21}}{a_{11}}=-0,78799$

\begin{equation*}
    \begin{matrix}
        A^{(1)}= \begin{pmatrix}
            1,566 & 2,0 & 1,018 \\
            0 & 1,56562 & 1,2338 \\
            0 & -0,55799 & -3,80218
        \end{pmatrix}
        & , &
        b^{(1)}=\begin{pmatrix}
            1 \\
            1,0019157 \\
            0,21201
        \end{pmatrix}
    \end{matrix}
\end{equation*}

Tomando: $m_{32}=-\frac{a_{32}}{a_{22}}=0,3564$

\begin{equation*}
    \begin{matrix}
        A^{(2)}= \begin{pmatrix}
            1,566 & 2,0 & 1,018 \\
            0 & 1,56562 & 1,2338 \\
            0 & 0 & -3,36245
        \end{pmatrix}
        & , &
        b^{(2)}=\begin{pmatrix}
            1 \\
            1,0019157 \\
            0,56909
        \end{pmatrix}
    \end{matrix}
\end{equation*}

Obteniendo el sistema de ecuaciones resultante:
\begin{align*}
    1,566x_1+2,0x2+1,018x_3 &=1 \\
    1,56562x_2+1,2338x_3 &=1,0019157 \\
    -3,36245x_3 &=0,56909
\end{align*}

Y sus respuestas son: $x_3=-0,16925 $ , $x_2=0,77333 $ y $x_1=-0,23905 $

%====================================================================================
a.5) Para el sistema de ecuación:\\
\begin{equation*}
    \begin{pmatrix}
        1 & -1 & 0 \\
        -1 & 2 & 0 \\
        0 & -1 & 2
    \end{pmatrix}
    \begin{pmatrix}
        x_1 \\
        x_2 \\
        x_3
    \end{pmatrix}    
    =    
    \begin{pmatrix}
        1 \\
        1 \\
        1
    \end{pmatrix}
\end{equation*}

\begin{equation*}
    \begin{matrix}
        A=\begin{pmatrix}
            1 & -1 & 0 \\
            -1 & 2 & 0 \\
            0 & -1 & 2
        \end{pmatrix}
        & , &
        X=\begin{pmatrix}
            x_1 \\
            x_2 \\
            x_3
        \end{pmatrix} 
        & , &
        b=\begin{pmatrix}
            1 \\
            1 \\
            1
        \end{pmatrix}
    \end{matrix}
\end{equation*}


Tomando: $m_{21}=-\frac{a_{21}}{a_{11}}=1$

\begin{equation*}
    \begin{matrix}
        A^{(1)}= \begin{pmatrix}
            1 & -1 & 0 \\
            0 & 1 & 0 \\
            0 & -1 & 2
        \end{pmatrix}
        & , &
        b^{(1)}=\begin{pmatrix}
            1 \\
            2 \\
            1
        \end{pmatrix}
    \end{matrix}
\end{equation*}

Tomando: $m_{32}=-\frac{a_{32}}{a_{22}}=1$

\begin{equation*}
    \begin{matrix}
        A^{(2)}= \begin{pmatrix}
            1 & -1 & 0 \\
            0 & 1 & 0 \\
            0 & 0 & 2
        \end{pmatrix}
        & , &
        b^{(2)}=\begin{pmatrix}
            1 \\
            2 \\
            3
        \end{pmatrix}
    \end{matrix}
\end{equation*}

Obteniendo el sistema de ecuaciones resultante:
\begin{align*}
    x_1-x2 &=1 \\
    x_2 &=2 \\
    2x_3 &=3
\end{align*}

Y sus respuestas son: $x_3=1,5 $ , $x_2=2 $ y $x_1=3$\\

b) Calcular el vector residual en cada caso

b.1) Para el sistema
\begin{equation*}
    \begin{pmatrix}
        1 & 1/2 & 1/3 \\
        1/2 & 1/3 & 1/4 \\
        1/3 & 1/4 & 1/5
    \end{pmatrix}
    \begin{pmatrix}
        x_1 \\
        x_2 \\
        x_3
    \end{pmatrix}    
    =    
    \begin{pmatrix}
        1 \\
        1 \\
        1
    \end{pmatrix}
\end{equation*}

Y su vector solución es:

\begin{equation*}
    X=
    \begin{pmatrix}
        3,00479 \\
        -24,02859 \\
        30,02882
    \end{pmatrix}    
\end{equation*}

El vector residual $R$ está definido de la siguiente manera:
$$ R=AX-b$$
Y para la presente solución es

\begin{equation*}
    R=
    \begin{pmatrix}
        1 & 1/2 & 1/3 \\
        1/2 & 1/3 & 1/4 \\
        1/3 & 1/4 & 1/5
    \end{pmatrix}
    \begin{pmatrix}
        3,00479 \\
        -24,02859 \\
        30,02882
    \end{pmatrix}    
    -   
    \begin{pmatrix}
        1 \\
        1 \\
        1
    \end{pmatrix}
\end{equation*}

\begin{equation*}
    R=
    \begin{pmatrix}
        0,000102 \\
        0,00007 \\
        0,000213
    \end{pmatrix}
\end{equation*}
\\
b.2) Para el sistema
\begin{equation*}
    \begin{pmatrix}
        100 & 99 & 98 \\
        98 & 55 & 11 \\
        0 & 1 & 1
    \end{pmatrix}
    \begin{pmatrix}
        x_1 \\
        x_2 \\
        x_3
    \end{pmatrix}    
    =    
    \begin{pmatrix}
        1 \\
        1 \\
        1
    \end{pmatrix}
\end{equation*}

Y su vector solución es:

\begin{equation*}
    X=
    \begin{pmatrix}
         \\
         \\
        
    \end{pmatrix}    
\end{equation*}

El vector residual $R$ para la presente solución es:

\begin{equation*}
    R=
    \begin{pmatrix}
        100 & 99 & 98 \\
        98 & 55 & 11 \\
        0 & 1 & 1
    \end{pmatrix}
    \begin{pmatrix}
        -0,98961 \\
        1,97691 \\
        -0,97707
    \end{pmatrix}    
    -   
    \begin{pmatrix}
        1 \\
        1 \\
        1
    \end{pmatrix}
\end{equation*}

\begin{equation*}
    R=
    \begin{pmatrix}
        0,00023 \\
        0,0005 \\
        -0,00016
    \end{pmatrix}
\end{equation*}

b.2) Para el sistema
\begin{equation*}
    \begin{pmatrix}
        1 & 0 & 1 \\
        -1 & 1 & 1 \\
        -1 & -1 & 1
    \end{pmatrix}
    \begin{pmatrix}
        x_1 \\
        x_2 \\
        x_3
    \end{pmatrix}    
    =    
    \begin{pmatrix}
        1 \\
        1 \\
        1
    \end{pmatrix}
\end{equation*}

Y su vector solución es:

\begin{equation*}
    X=
    \begin{pmatrix}
        0 \\
        0 \\
        1
    \end{pmatrix}    
\end{equation*}

El vector residual $R$ para la presente solución es:

\begin{equation*}
    R=
    \begin{pmatrix}
        1 & 0 & 1 \\
        -1 & 1 & 1 \\
        -1 & -1 & 1
    \end{pmatrix}
    \begin{pmatrix}
        0 \\
        0 \\
        1
    \end{pmatrix}    
    -   
    \begin{pmatrix}
        1 \\
        1 \\
        1
    \end{pmatrix}
\end{equation*}

\begin{equation*}
    R=
    \begin{pmatrix}
        0 \\
        0 \\
        0
    \end{pmatrix}
\end{equation*}

b.4) Para el sistema
\begin{equation*}
    \begin{pmatrix}
        0,0003 & 1,566 & 1,234 \\
        1,566 & 2,0 & 1,018 \\
        1,234 & 1,018 & -3,0
    \end{pmatrix}
    \begin{pmatrix}
        x_1 \\
        x_2 \\
        x_3
    \end{pmatrix}    
    =    
    \begin{pmatrix}
        1 \\
        1 \\
        1
    \end{pmatrix}
\end{equation*}

Y su vector solución es:

\begin{equation*}
    X=
    \begin{pmatrix}
        -0,23905 \\
        0,77333 \\
        -0,16925
    \end{pmatrix}    
\end{equation*}

El vector residual $R$ para la presente solución es:

\begin{equation*}
    R=
    \begin{pmatrix}
        0,0003 & 1,566 & 1,234 \\
        1,566 & 2,0 & 1,018 \\
        1,234 & 1,018 & -3,0
    \end{pmatrix}
    \begin{pmatrix}
        -0,23905 \\
        0,77333 \\
        -0,16925
    \end{pmatrix}    
    -   
    \begin{pmatrix}
        1 \\
        1 \\
        1
    \end{pmatrix}
\end{equation*}

\begin{equation*}
    R=
    \begin{pmatrix}
        0,002108 \\
        0,000012 \\
        0,000012
    \end{pmatrix}
\end{equation*}

b.4) Para el sistema
\begin{equation*}
    \begin{pmatrix}
        1 & -1 & 0 \\
        -1 & 2 & 0 \\
        0 & -1 & 2
    \end{pmatrix}
    \begin{pmatrix}
        x_1 \\
        x_2 \\
        x_3
    \end{pmatrix}    
    =    
    \begin{pmatrix}
        1 \\
        1 \\
        1
    \end{pmatrix}
\end{equation*}

Y su vector solución es:

\begin{equation*}
    X=
    \begin{pmatrix}
        3 \\
        2 \\
        1,5
    \end{pmatrix}    
\end{equation*}

El vector residual $R$ para la presente solución es:

\begin{equation*}
    R=
    \begin{pmatrix}
        1 & -1 & 0 \\
        -1 & 2 & 0 \\
        0 & -1 & 2
    \end{pmatrix}
    \begin{pmatrix}
        
    \end{pmatrix}    
    -   
    \begin{pmatrix}
        1 \\
        1 \\
        1
    \end{pmatrix}
\end{equation*}

\begin{equation*}
    R=
    \begin{pmatrix}
        0 \\
        0 \\
        0
    \end{pmatrix}
\end{equation*}
