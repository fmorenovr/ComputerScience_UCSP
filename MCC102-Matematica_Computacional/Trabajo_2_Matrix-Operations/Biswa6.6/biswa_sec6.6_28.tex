\begin{enumerate}[label=(\alph*)]
\item ¿Cómo está relacionado $Cond(A)$ y $Cond(A^{-1})$?
\item Pruebe que:\\
(i) $1 \leq Cond(A)$\\
(ii) $Cond(A^TA)=Cond^2(A)$
\end{enumerate}

\noindent \textcolor{red}{\bf Solución:}
\begin{enumerate}[label=(\alph*)]
    \item ¿Cómo está relacionado $Cond(A)$ y $Cond(A^{-1})$?\\
Ambos numeros condicionantes son los mismos. Por definicion sabemos:
$$Cond(A) =  \begin{Vmatrix} A \end{Vmatrix} \begin{Vmatrix} A^{-1} \end{Vmatrix}$$
Si calculamos $Cond(A^{-1})$:
$$Cond(A^{-1}) =  \begin{Vmatrix} A^{-1} \end{Vmatrix} \begin{Vmatrix} (A^{-1})^{-1} \end{Vmatrix}$$
$$Cond(A^{-1}) =  \begin{Vmatrix} A^{-1} \end{Vmatrix} \begin{Vmatrix} A \end{Vmatrix}$$
Al tratarse de normas se puede escribir:
$$\begin{Vmatrix} A^{-1} \end{Vmatrix} \begin{Vmatrix} A \end{Vmatrix} = \begin{Vmatrix} A \end{Vmatrix}\begin{Vmatrix} A^{-1} \end{Vmatrix}$$
$$Cond(A^{-1}) = Cond(A)$$

\item Pruebe que:

(i) $1 \leq Cond(A)$

Sabiendo que $Cond(AB) \leq Cond(A)Cond(B)$
\begin{gather*}
Cond(A)=Cond(AA^{-1}A) \\
Cond(AA^{-1}A) \leq Cond(A)Cond(A^{-1})Cond(A)
\end{gather*}

Como $Cond(A) = Cond(A^{-1})$

\begin{gather*}
Cond(AA^{-1}A) \leq Cond(A)Cond(A)Cond(A)\\
Cond(A) \leq Cond(A)Cond(A)Cond(A)\\
1 \leq Cond^2(A)\\
1 \leq Cond(A)\\
\end{gather*}

(ii) $Cond(A^TA)=Cond^2(A)$

Puesto que $Cond(A)= \frac{\sigma_{max}}{\sigma_{min}}$ tenemos que
\begin{gather*}
Cond(A^TA)=\frac{\sigma_{max}^1}{\sigma_{min}^1}\\
\begin{Vmatrix} (A^TA)^T(A^TA) - \sigma^1I \end{Vmatrix}=0\\
\begin{Vmatrix} (A^TA)^2 - \sigma^1I \end{Vmatrix}=0\\
\begin{Vmatrix} A^TA - \sqrt{\sigma^1}I \end{Vmatrix} \begin{Vmatrix} A^TA + \sqrt{\sigma^1}I \end{Vmatrix}=0\\
\begin{Vmatrix} A^TA - \sqrt{\sigma^1}I \end{Vmatrix} = 0\\
\sqrt{\sigma^1}=sqrt{\sigma}\\
\frac{\sigma_{max}^1}{\sigma_{min}^1}=\left( \frac{\sigma_{max}}{\sigma_{min}} \right) ^2\\
Cond(A^TA)=Cond^2(A)\\
\end{gather*}

\end{enumerate}