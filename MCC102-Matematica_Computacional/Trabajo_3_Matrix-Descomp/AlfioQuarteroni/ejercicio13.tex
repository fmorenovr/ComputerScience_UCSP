Muestre que el coeficiente $\alpha_k$  y $\beta_k$ en el método de la gradiente conjugada puede ser escrita en la forma alternativa:

\[\alpha_k=\frac{||r^{(K)}||_2^2}{{p^{(k)}}^T}Ap^{(k)} , \beta_k=\frac{||r^{(k+1)}||_2^2}{||r^{(k)}||_2^2} \]

\textbf{Solución}\\
Para $\alpha_k=\frac{||r^{(K)}||_2^2}{{p^{(k)}}^T}Ap^{(k)}$\\
Se sabe que: $r^{k+1} = r^k - \alpha_k AP^{k} $ , por lo que:\\
\[AP^k = \frac{r^k - r^{k+1}}{\alpha_k} ....(I)\]
\[(r^k)^T AP^k = \frac{(r^k)^T r^k - (r^k)^T r^{k+1}}{\alpha_k}\]
Como $(r^k)^T r^{k+1}$ es el producto interno, entonces $(r^k)^T r^{k+1}=0$
\[(r^k)^T AP^k = || r^k ||_2^2 ...(II)\]
\[p^{k+1} = r^{k+1} - \beta_k P\]
\[(P^k)^T AP^k = (r^k - \beta_{k-1})P^{k-1} AP^k\]
\[(P^k)^T AP^k = ((r^k)^T - \beta_{k-1})(P^{k-1})^T)AP^k\]
\[(P^k)^T AP^k = (r^k)^T AP^k - \beta_{k-1}(P^{k-1} AP^k) , (P^{k-1})^T P^T = 0\]
\[(P^k)^T AP^k = (r^k)^T AP^k ...(III)\]
De II y III se tiene que:\\
\[(P^k)^T AP^k = \frac{||r^k||_2^2}{\alpha_k}\]
\[\alpha_k = \frac{||r^k||_2^2}{(P^k) AP^k}\]

Para $\beta_k=\frac{||r^{(k+1)}||_2^2}{||r^{(k)}||_2^2} $\\
Partimos de I se tiene que:
\[(AP^k)^T = \frac{r^k - r^{k+1}}{\alpha_k} r^{k+1}\]
Despejando el producto interno:
\[(AP^k)^T r^{k+1} = \frac{||r^{k+1}||_2^2}{\alpha_k}  ...(IV)\]
Por definición se tiene:
\[\beta_k = \frac{(AP^k)^T r^{k+1}}{(AP^k)^T P^{k+1}}\]
Y del resultado de $\alpha_k=\frac{||r^{(K)}||_2^2}{{p^{(k)}}^T}Ap^{(k)}$ y de IV se tiene:
\[\beta_k = \frac{-\frac{||r^{k+1}||_2^2}{\alpha_k}}{-\frac{||r^k||_2^2}{\alpha_k}}\]
\[\beta_k = \frac{|| r^{k+1} ||_2^2}{||r^k||_2^2}\]
