Sea A $\in$ $R^{nxn}$ una matriz estrictamente diagonal dominante por filas. Mostrar si el método de Gauss-Seidel es convergente para el sistema lineal $(3.2)$.
\\
Ejemplo 3.2 

$$(Ax = b) = \begin{pmatrix}
1 & 1/2 & 1/3 \\
1/2 & 1/3 & 1/4 \\
1/3 & 1/4 & 1/5 \\
\end{pmatrix}\
\begin{pmatrix}
x_{1} \\
x_{2} \\
x_{3} \\
\end{pmatrix}
$$
\\
Para realizar el método de Gauss Seidel y asegurar la convergencia, es necesario que sea diagonal dominante por tanto comprobamos.

$$
A = \begin{pmatrix}
1 & 1/2 & 1/3 \\
1/2 & 1/3 & 1/4 \\
1/3 & 1/4 & 1/5 \\
\end{pmatrix}\
$$

Comprobamos la diagonal por filas, comparando si es mayor.

$$\begin{pmatrix} \textbf{1} & 1/2 & 1/3
\end{pmatrix}$$

en este caso si es mayor.

$$\begin{pmatrix} 1/2 & \textbf{1/3} & 1/4
\end{pmatrix}$$

en este caso el mayor valor se encuentra en la primera fila y pasa de igual forma en la tercera fila.

$$\begin{pmatrix} 1/3 & 1/4 & \textbf{1/5}
\end{pmatrix}$$

por tanto no se pude modificar para que cumpla esta propiedad.

Si aplicamos el radio espectral obtendremos un resultado $>$ 1.

$$\frac{1/2}{1} + \frac{1/3}{1} =< 1$$
$$\frac{1/2}{1/3} + \frac{1/4}{1/3} =< 1$$
$$\frac{1/3}{1/5} + \frac{1/4}{1/5} =< 1$$

obteniendo:

$$\frac{5}{6} =< 1$$
$$\frac{9}{4} =< 1$$
$$\frac{35}{12} =< 1$$

como resultado:

$$0.83 =< 1$$
$$2.25 =< 1$$
$$2.92 =< 1$$

Solo en el primer caso se cumple el criterio de convergencia en los otros 3 no, de esta forma al no cumplirse el criterio de convergencia el método de Gauss-Seidel.