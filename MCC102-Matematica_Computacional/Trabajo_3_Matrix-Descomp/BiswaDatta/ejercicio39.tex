\textbf{Considere el sistema de bloques que converge de la solución de la ecuación discreta de Poisson $u_xx + u_yy = f$.}
% matriz
\begin{align*}
    \begin{bmatrix}
    T      & -I    &  \dots  & \dots  & 0\\
    -I     & T     &  \ddots & \ddots  & \vdots\\
    \vdots &\ddots &  \ddots & \ddots & -I\\
    0     &\dots   &  \dots  & -I     & T
\end{bmatrix}
Donde \quad T=
\begin{bmatrix}
    4 & -1 & x_{13} & \dots  & 0 \\
    -1 & x_{22} & x_{23} & \dots  & \vdots \\
    \vdots & \ddots & \ddots & \ddots & \vdots \\
    0 & x_{d2} & \dots & \dots  & x_{dn}
\end{bmatrix}
\end{align*}

lo anterior muestra que el bloque de iteración de Jacobi, en este caso particular es 
\begin{align*}
    Tx^{k+1}_{1} = x^{k}_{i+1} + x^{k}_{i-1} +b_{i} \quad i = 1,\dots, N
\end{align*}
describiendo los bloques de iteración de Gauss-Seidel y SOR.
para el métdo.

para el metodo en bloque de Jacobi se tiene que:
\begin{align*}
    Tx^{k+1}_{1}=b_i \quad \sum^{N}_{j = 1, i \neq j} A_{ij}x_j^k
,\quad i = 1,\dots, N
\end{align*}
Como $A_{ii} = T$ y $A_{ij, i != j}= -I$, entonces:
\begin{align*}
    Tx^{k+1}_{1}=b_i - \sum^{N}_{j = 1, i \neq j} -Ix_j^k \quad i = 1,\dots, N\\
    Tx^{k+1}_{1}=b_i + \sum^{N}_{j = 1, i \neq j} x_j^k \quad i = 1,\dots, N
\end{align*}

como los unicos elementos existentes alrededor de la diagonal son los términos $(i-1)$ e $(i+1)$, entonces:
\begin{align*}
    Tx^{k+1}_{1} = x^{k}_{i+1} + x^{k}_{i-1} + b_{i} \quad i = 1,\dots, N
\end{align*}

De la misma forma el bloque de iteración del Gaus-Seidel, tiene la forma:

\begin{align*}
    A_{ii}x_{i}^{k+1} = b_i - \sum^{i-1}_{j = i} A_{ij}x^{k}_j -\sum^{N}_{j = i + 1} A_{ij}x^{k}_j \quad i = 1,\dots, N\\
    Tx_{i}^{k+1} = b_i + \sum^{i-1}_{j = 1} Ix^{k + 1}_j +\sum^{N}_{j = i + 1} Ix^{k}_j \quad i = 1,\dots, N\\
    Tx_{i}^{k+1} = b_i + \sum^{i-1}_{j = 1} x^{k + 1}_j +\sum^{N}_{j = i + 1} x^{k}_j \quad i = 1,\dots, N\\
\end{align*}

Como los únicos elementos alrededor de la diagonal son los términos $(i-1)$ e $(i-1)$, entonces :

\begin{align*}
    Tx^{k+1}_{i} = x^{k+1}_{i-1} + x^{k}_{i+1} + b_{i} \quad i = 1,\dots, N
\end{align*}

Por otro lado para el caso del SOR se tiene :
\begin{align*}
    x^{k+1}_{i} = \frac{W}{A_{ii}} \left[b_i - \sum^{i-1}_{j = 1}A_{ij}x^{k+1}_j - \sum^{N}_{j = i+1}A_{ij}x^k_j \right] + (1 -w)x^k_i \quad i = 1,\dots, N\\
    x^{k+1}_{i} = \frac{W}{T} \left[b_i - \sum^{i-1}_{j = 1}x^{k+1}_j - \sum^{N}_{j = i+1}x^k_j \right] + (1 -w)x^k_i \quad i = 1,\dots, N\\
\end{align*}

Ahora como los únicos elementos existentes alrededor de la diagonal son los términos $(i-1)$  e $(i +1)$, entonces:
\begin{align*}
    x^{k+1}_i = \frac{W}{T}\left[b_i + x^{k+1}_{i-1} + x^k_{i+1}
    \right]x^k_i + (1 - W)x^k_i\quad i = 1,\dots, N\\
    Tx^{k+1}_i = W\left[b_i + x^{k+1}_{i-1} + x^k_{i+1} \right]x^k_i + (1 - W)Tx^k_i \quad i = 1,\dots, N\\
\end{align*}