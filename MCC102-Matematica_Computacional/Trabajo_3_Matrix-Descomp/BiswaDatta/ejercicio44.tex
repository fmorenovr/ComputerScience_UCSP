%Vittorino Mandujano Cornejo
Pruebe que para un $$ \alpha = \frac{p^T(Ax-b)}{p^TAp} $$ minimiza la función cuadrática
$$
\begin{array}{lcll}
    \phi_{\alpha} & = & \phi (x - \alpha p ) \\
                & = & \frac{1}{2}(x-\alpha p)^TA(x-\alpha p)-b^T(x-\alpha p)
\end{array}
$$ 

\textbf{Solución:\\}

Re-escribiendo la función $\phi_{\alpha}$ tenemos

$$
\phi_{\alpha} = \frac{1}{2} (x^TAx - \alpha x^TAp - \alpha p^TAx + \alpha^2 p^TAp)-b^T(x-\alpha p)
$$

Y derivamos con respecto a $\alpha$ .

$$
  \frac{d\phi_{\alpha}}{d\phi} = -\frac{1}{2} (x^TAp + p^TAx) + \alpha p^TAp + b^Tp
$$

Puesto que $A$ es una matriz simétrica, tenemos que $\alpha x^TAp = \alpha p^TAx$ y simplificamos nuestra expresión

$$
  \frac{d\phi_{\alpha}}{d\phi} = -x^TAp + \alpha p^TAp + b^Tp
$$

Igualamos $ \frac{d\phi_{\alpha}}{d\phi}$ a cero en pos de encontrar un punto crítico

$$
  \frac{d\phi_{\alpha}}{d\phi} = 0
$$

$$
   -x^TAp + \alpha p^TAp + b^Tp = 0
$$

y despejamos el valor de $\alpha$

$$ \alpha = \frac{p^T(Ax-b)}{p^TAp} $$

Que es justamente la expresión que queremos saber si minimiza la función. Para poder saber si definitivamente cumple dicho propósito, debemos aplicar el criterio de la segunda derivada a $\phi_{\alpha}$

$$
  \frac{d\phi_{\alpha}}{d\phi} = -x^TAp + \alpha p^TAp + b^Tp
$$

$$
  \frac{d^2\phi_{\alpha}}{d\phi^2} = p^TAp
$$

y puesto que $A$ es una matriz simétrica positiva, entonces $p^TAp>0$ y por lo tanto $\frac{d^2\phi_{\alpha}}{d\phi^2}>0$ en $\alpha=\frac{p^T(Ax-b)}{p^TAp}$ también, con lo que se prueba que dicho valor de $\alpha$ sí minimiza la función.