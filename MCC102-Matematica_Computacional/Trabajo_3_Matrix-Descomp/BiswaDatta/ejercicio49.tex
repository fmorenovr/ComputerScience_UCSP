Aplique los Métodos de Jacobi, Gauss-Seidel, y SOR (con factor de relajación óptima) al sistema mostrado en el ejemplo del Teorema 6.10.5 (Ejemplo 6.10.5) y verificar la afirmación sobre el número de iteraciones realizadas por los diferentes métodos.

\textbf{Solución:}

El problema propuesto es resolver el sistema $Ax=b$, con:

\begin{equation}
    A = 
    \begin{pmatrix}
     4 &	-1 &	 0 &	-1 &	 0 &	 0 \\
    -1 &	 4 &	-1 &	 0 &	-1 &	 0 \\
     0 &	-1 &	 4 &	 0 &	 0 &	-1 \\
    -1 &	 0 &	 0 &	 4 &	-1 &	 0 \\
     0 &	-1 &	 0 &	-1 &	 4 &	-1 \\
     0 &	 0 &	-1 &	 0 &	-1 &	 4
    \end{pmatrix}
    , \quad \quad
    b = 
    \begin{pmatrix}
     1 \\
     0 \\
     0 \\
     0 \\
     0 \\
     0 
    \end{pmatrix}
\end{equation}

Haciendo uso del Método de Jacobi:

\begin{table}[h]
    \centering
    \begin{tabular}{r|llllll|l}
        Iteración & $x_1$ &    $x_2$  &     $x_3$  &     $x_4$  &     $x_5$  &     $x_6$  &     Error \\
        \hline
         0  &  1.000000  &  0.000000  &  0.000000  &  0.000000  &  0.000000  &  0.000000  &  -        \\
         1  &  1.000000  &  0.000000  &  0.000000  &  0.000000  &  0.000000  &  0.000000  &  0.829156 \\
         2  &  0.250000  &  0.250000  &  0.000000  &  0.250000  &  0.000000  &  0.000000  &  0.324760 \\
         3  &  0.375000  &  0.062500  &  0.062500  &  0.062500  &  0.125000  &  0.000000  &  0.178836 \\
         4  &  0.281250  &  0.140625  &  0.015625  &  0.125000  &  0.031250  &  0.046875  &  0.106190 \\
         5  &  0.316406  &  0.082031  &  0.046875  &  0.078125  &  0.078125  &  0.011719  &  0.063911 \\
         6  &  0.290039  &  0.110352  &  0.023438  &  0.098633  &  0.042969  &  0.031250  &  0.038555 \\
         7  &  0.302246  &  0.089111  &  0.035400  &  0.083252  &  0.060059  &  0.016602  &  0.023268 \\
         8  &  0.293091  &  0.099426  &  0.026428  &  0.090576  &  0.047241  &  0.023865  &  0.014043 \\
         9  &  0.297501  &  0.091690  &  0.030823  &  0.085083  &  0.053467  &  0.018417  &  0.008476 \\
        10  &  0.294193  &  0.095448  &  0.027527  &  0.087742  &  0.048798  &  0.021072  &  0.005116 \\
        11  &  0.295797  &  0.092629  &  0.029130  &  0.085748  &  0.051065  &  0.019081  &  0.003088 \\
        12  &  0.294594  &  0.093998  &  0.027928  &  0.086716  &  0.049365  &  0.020049  &  0.001864 \\
        13  &  0.295178  &  0.092972  &  0.028512  &  0.085990  &  0.050191  &  0.019323  &  0.001125 \\
        14  &  0.294740  &  0.093470  &  0.028074  &  0.086342  &  0.049571  &  0.019676  &  0.000679 \\
        15  &  0.294953  &  0.093096  &  0.028286  &  0.086078  &  0.049872  &  0.019411  &  0.000410 \\
        16  &  0.294794  &  0.093278  &  0.028127  &  0.086206  &  0.049646  &  0.019540  &  0.000247 \\
        17  &  0.294871  &  0.093142  &  0.028204  &  0.086110  &  0.049756  &  0.019443  &  0.000149 \\
        18  &  0.294813  &  0.093208  &  0.028146  &  0.086157  &  0.049674  &  0.019490  &  0.000090
    \end{tabular}
\end{table}

Haciendo uso del Método de Gauss-Seidel:

\begin{table}[H]
    \centering
    \begin{tabular}{r|llllll|l}[H]
        Iteración & $x_1$ &    $x_2$  &     $x_3$  &     $x_4$  &     $x_5$  &     $x_6$  &     Error \\
        \hline
         0  &  1.000000  &  0.000000  &  0.000000  &  0.000000  &  0.000000  &  0.000000  &  -        \\
         1  &  1.000000  &  0.000000  &  0.000000  &  0.000000  &  0.000000  &  0.000000  &  0.756089 \\
         2  &  0.250000  &  0.062500  &  0.015625  &  0.062500  &  0.031250  &  0.011719  &  0.042713 \\
         3  &  0.281250  &  0.082031  &  0.023438  &  0.078125  &  0.042969  &  0.016602  &  0.013570 \\
         4  &  0.290039  &  0.089111  &  0.026428  &  0.083252  &  0.047241  &  0.018417  &  0.004836 \\
         5  &  0.293091  &  0.091690  &  0.027527  &  0.085083  &  0.048798  &  0.019081  &  0.001755 \\
         6  &  0.294193  &  0.092629  &  0.027928  &  0.085748  &  0.049365  &  0.019323  &  0.000639 \\
         7  &  0.294594  &  0.092972  &  0.028074  &  0.085990  &  0.049571  &  0.019411  &  0.000233 \\
         8  &  0.294740  &  0.093096  &  0.028127  &  0.086078  &  0.049646  &  0.019443  &  0.000085
    \end{tabular}
\end{table}

Haciendo uso del Método SOR (con factor de relajación óptima):

\begin{table}[h]
    \centering
    \begin{tabular}{r|llllll|l}
        Iteración & $x_1$ &    $x_2$  &     $x_3$  &     $x_4$  &     $x_5$  &     $x_6$  &     Error \\
        \hline
         0  &  1.000000  &  0.000000  &  0.000000  &  0.000000  &  0.000000  &  0.000000  &  -        \\
         1  &  0.304137  &  0.122296  &  0.077391  &  0.100767  &  0.154781  &  0.000000  &  0.763377 \\
         2  &  0.297005  &  0.100259  &  0.043263  &  0.090651  &  0.062568  &  0.064588  &  0.352623 \\
         3  &  0.295022  &  0.093962  &  0.029177  &  0.086928  &  0.052218  &  0.022158  &  0.141864 \\
         4  &  0.294859  &  0.093312  &  0.028453  &  0.086190  &  0.050038  &  0.020145  &  0.009832 \\
         5  &  0.294829  &  0.093184  &  0.028193  &  0.086144  &  0.049736  &  0.019564  &  0.002199 \\
         6  &  0.294825  &  0.093170  &  0.028161  &  0.086130  &  0.049696  &  0.019473  &  0.000325 \\
         7  &  0.294824  &  0.093168  &  0.028158  &  0.086129  &  0.049690  &  0.019463  &  0.000037
    \end{tabular}
\end{table}


Tolerancia 0.0001.

