Para cada una de las siguientes matrices, encontrar:
\begin{enumerate}[]
    \item Las matrices de permutación $P_1$ y $P_2$ y las matrices elementales $M_1$ y $M_2$ tal que $MA = M_2P_2M_1P_1A$ es una matriz triangular superior.
    
    \item Las matrices de permutación $P_1$, $P_2$, $Q_1$, $Q_2$ y las matrices elementales $M_1$ y $M_2$ tal que $MAQ = M_2(P_2(M_1P_1 A Q_1)Q_2)$ es una matriz triangular superior.
    
    \begin{enumerate}[]
            \item 
            $A = \begin{pmatrix}
                1 & \frac{1}{2} & \frac{1}{3} \\ 
                \frac{1}{2} & \frac{1}{3} & \frac{1}{4} \\
                \frac{1}{3} & \frac{1}{4} & \frac{1}{5} \\
            \end{pmatrix}$
            \\\\
            \item
            $A = \begin{pmatrix}
                100 & 99 & 98 \\ 
                98 & 55 & 11 \\
                0 & 1 & 1 \\
            \end{pmatrix}$
            \\\\
            \item
            $A = \begin{pmatrix}
                1 & 0 & 1 \\ 
                -1 & 1 & 1 \\
                -1 & -1 & 1 \\
            \end{pmatrix}$
            \\\\
            \item
            $A = \begin{pmatrix}
                0.0003 & 1.566 &  1.234 \\ 
                1.5660 & 2.000 &  1.018 \\
                1.2340 & 1.018 & -3.000 \\
            \end{pmatrix}$
            \\\\
            \item
            $A = \begin{pmatrix}
                1 & -1 & 0 \\ 
                -1 & 2 & 0 \\
                0 & -1 & 2 \\
            \end{pmatrix}$
        \end{enumerate}
    
    \item Expresa cada descomposición en la forma $PAQ = LU$ (Note que para la eliminación de Gauss sin y con pivoteo parcial, $Q=I$).
    
    \item Calcule el factor de crecimiento en cada caso.
\end{enumerate}

\noindent \textcolor{red}{\bf Solución:}\\    
\begin{enumerate}[]
    \item 
    $A = \begin{pmatrix}
            1 & \frac{1}{2} & \frac{1}{3} \\ 
            \frac{1}{2} & \frac{1}{3} & \frac{1}{4} \\
            \frac{1}{3} & \frac{1}{4} & \frac{1}{5} \\
        \end{pmatrix}$
    
    \begin{enumerate}[]
        \item Pivoteo parcial\\
        El máximo valor de la primera columna en la matriz $A$ es $a_{11}$; por lo tanto, se tiene $P_1$ y $M_1$\\
        \\
        $P_1 = I$; 
        $M_1 = \begin{pmatrix}
            1 & 0 & 0 \\ 
            -\frac{1}{2} & 1 & 0 \\
            -\frac{1}{3} & 0 & 1 \\
        \end{pmatrix} $ 
        $\xrightarrow{}$
        $A^{(1)}= M_1P_1A = \begin{pmatrix}
            1 & \frac{1}{2} & \frac{1}{3} \\ 
            0 & \frac{1}{12} & \frac{1}{12} \\
            0 & \frac{1}{12} & \frac{4}{45} \\
        \end{pmatrix}$
        \\\\\\
        El máximo valor de la segunda columna en la matriz $A^{(1)}$, por debajo de la primera fila es $a_{22}$; por lo tanto, se tiene $P_2$ y $M_2$\\
        \\
        $P_2 = I$; 
        $M_2 = \begin{pmatrix}
            1 & 0 & 0 \\ 
            0 & 1 & 0 \\
            0 & -1 & 1 \\
        \end{pmatrix} $ 
        $\xrightarrow{}$
        $A^{(2)}= M_2P_2A^{(1)} = \begin{pmatrix}
            1 & \frac{1}{2} & \frac{1}{3} \\ 
            0 & \frac{1}{12} & \frac{1}{12} \\
            0 & 0 & \frac{1}{180} \\
        \end{pmatrix}$
        \\\\\\
        $\xrightarrow{}$ Se verifica que $MA = M_2P_2M_1P_1A$ es una matriz triangular superior\\
        
        \item Pivoteo completo\\
        El máximo valor de todas las columnas y filas de la matriz $A$ es $a_{11}$; por lo tanto, se tiene $P_1$, $Q_1$ y $M_1$\\
        \\
        $P_1 = Q_1 = I$
        $\xrightarrow{}$ $P_1AQ_1 = A$; 
        $M_1 = \begin{pmatrix}
            1 & 0 & 0 \\ 
            -\frac{1}{2} & 1 & 0 \\
            -\frac{1}{3} & 0 & 1 \\
        \end{pmatrix} $ \\\\
        $\xrightarrow{}$
        $A^{(1)}= M_1P_1AQ_1 = \begin{pmatrix}
            1 & \frac{1}{2} & \frac{1}{3} \\ 
            0 & \frac{1}{12} & \frac{1}{12} \\
            0 & \frac{1}{12} & \frac{4}{45} \\
        \end{pmatrix}$
        \\\\\\
        El máximo valor de las columnas y filas de la matriz $A^{(1)}$, por debajo de la primera fila es $a_{33}$; por lo tanto, se tiene $P_2$, $Q_2$ y $M_2$\\
        \\
        $P_2 = \begin{pmatrix}
            1 & 0 & 0 \\ 
            0 & 0 & 1 \\
            0 & 1 & 0 \\
        \end{pmatrix} $;
        $Q_2 = \begin{pmatrix}
            1 & 0 & 0 \\ 
            0 & 0 & 1 \\
            0 & 1 & 0 \\
        \end{pmatrix} $
        $\xrightarrow{}$ $P_2A^{(1)}Q_2 = \begin{pmatrix}
            1 & \frac{1}{3} & \frac{1}{2} \\ 
            0 & \frac{4}{45} & \frac{1}{12} \\
            0 & \frac{1}{12} & \frac{1}{12} \\
        \end{pmatrix}$
        \\\\\\
        $M_2 = \begin{pmatrix}
            1 & 0 & 0 \\ 
            0 & 1 & 0 \\
            0 & -\frac{15}{16} & 1 \\
        \end{pmatrix} $ 
        $\xrightarrow{}$
        $A^{(2)}= M_2P_2A^{(1)}Q_2 = \begin{pmatrix}
            1 & \frac{1}{3} & \frac{1}{2} \\ 
            0 & \frac{4}{45} & \frac{1}{12} \\
            0 & 0 & \frac{1}{192} \\
        \end{pmatrix}$
        \\\\\\
        $\xrightarrow{}$ Se verifica que $MAQ = M_2(P_2(M_1P_1AQ_1)Q_2$ es una matriz triangular superior\\
        
        \item
        \begin{enumerate}[]
            \item Para el caso de pivoteo parcial
            \begin{align*}
                PA &= LU\\
                \begin{pmatrix}
                1 & 0 & 0 \\ 
                0 & 1 & 0 \\
                0 & 0 & 1 \\
                \end{pmatrix}
                \begin{pmatrix}
                1 & \frac{1}{2} & \frac{1}{3} \\ 
                \frac{1}{2} & \frac{1}{3} & \frac{1}{4} \\
                \frac{1}{3} & \frac{1}{4} & \frac{1}{5} \\
                \end{pmatrix} &= 
                \begin{pmatrix}
                1 & 0 & 0 \\ 
                \frac{1}{2} & 1 & 0 \\
                \frac{1}{3} & 1 & 1 \\
                \end{pmatrix}
                \begin{pmatrix}
                1 & \frac{1}{2} & \frac{1}{3} \\ 
                0 & \frac{1}{12} & \frac{1}{12} \\
                0 & 0 & \frac{1}{180} \\
                \end{pmatrix}
            \end{align*}
            \\
            
            \item Para pivoteo completo
            \begin{align*}
                PAQ &= LU\\
                \begin{pmatrix}
                1 & 0 & 0 \\ 
                0 & 0 & 1 \\
                0 & 1 & 0 \\
                \end{pmatrix}
                \begin{pmatrix}
                1 & \frac{1}{2} & \frac{1}{3} \\ 
                \frac{1}{2} & \frac{1}{3} & \frac{1}{4} \\
                \frac{1}{3} & \frac{1}{4} & \frac{1}{5} \\
                \end{pmatrix}
                \begin{pmatrix}
                1 & 0 & 0 \\ 
                0 & 0 & 1 \\
                0 & 1 & 0 \\
                \end{pmatrix} &= 
                \begin{pmatrix}
                1 & 0 & 0 \\ 
                \frac{1}{3} & 1 & 0 \\
                \frac{1}{2} & \frac{15}{16} & 1 \\
                \end{pmatrix}
                \begin{pmatrix}
                1 & \frac{1}{3} & \frac{1}{2} \\ 
                0 & \frac{4}{45} & \frac{1}{12} \\
                0 & 0 & \frac{1}{192} \\
                \end{pmatrix}
            \end{align*}
            \\
        \end{enumerate}
        
        \item Factor de crecimiento
        \begin{enumerate}[]
            \item Para el caso del pivoteo parcial
            \begin{align*}
                \rho &=  \frac{max |a_{ij}^{(2)}|}{max |a_{ij}|}\\
                &=  \frac{1}{1}\\
                &=  1\\
            \end{align*}
            
             \item Para pivoteo completo
             \begin{align*}
                \rho &=  \frac{max |a_{ij}^{(2)}|}{max |a_{ij}|}\\
                &=  \frac{1}{1}\\
                &=  1\\
            \end{align*}
        \end{enumerate}
        
    \end{enumerate}
    
    %%%%%%%%%%%%%%%%%%%%%%%%%%%%%
    \item 
    $A = \begin{pmatrix}
            100 & 99 & 98 \\ 
            98 & 55 & 11 \\
            0 & 1 & 1 \\
        \end{pmatrix}$
    
    \begin{enumerate}[]
        \item Pivoteo parcial\\
        El máximo valor de la primera columna en la matriz $A$ es $a_{11}$; por lo tanto, se tiene $P_1$ y $M_1$\\
        \\
        $P_1 = I$; 
        $M_1 = \begin{pmatrix}
            1 & 0 & 0 \\ 
            -\frac{98}{100} & 1 & 0 \\
            0 & 0 & 1 \\
        \end{pmatrix} $ 
        $\xrightarrow{}$
        $A^{(1)}= M_1P_1A = \begin{pmatrix}
            100 & 99 & 98 \\ 
            0 & -\frac{2101}{50} & -\frac{2126}{25} \\
            0 & 1 & 1 \\
        \end{pmatrix}$
        \\\\\\
        El máximo valor de la segunda columna en la matriz $A^{(1)}$, por debajo de la primera fila es $a_{22}$; por lo tanto, se tiene $P_2$ y $M_2$\\\\
        $P_2 = I$; 
        $M_2 = \begin{pmatrix}
            1 & 0 & 0 \\ 
            0 & 1 & 0 \\
            0 & \frac{50}{2101} & 1 \\
        \end{pmatrix} $ 
        $\xrightarrow{}$
        $A^{(2)}= M_2P_2A^{(1)} = \begin{pmatrix}
             100 & 99 & 98 \\ 
            0 & -\frac{2101}{50} & -\frac{2126}{25} \\
            0 & 0 & -\frac{2151}{2101} \\
        \end{pmatrix}$
        \\\\\\
        $\xrightarrow{}$ Se verifica que $MA = M_2P_2M_1P_1A$ es una matriz triangular superior\\
        
        \item Pivoteo completo\\
        El máximo valor de todas las columnas y filas de la matriz $A$ es $a_{11}$; por lo tanto, se tiene $P_1$, $Q_1$ y $M_1$\\
        \\
        $P_1 = Q_1 = I$
        $\xrightarrow{}$ $P_1AQ_1 = A$; 
        $M_1 = \begin{pmatrix}
            1 & 0 & 0 \\ 
            -\frac{98}{100} & 1 & 0 \\
            0 & 0 & 1 \\
        \end{pmatrix} $ \\\\
        $\xrightarrow{}$
        $A^{(1)}= M_1P_1AQ_1 = \begin{pmatrix}
            100 & 99 & 98 \\ 
            0 & -\frac{2101}{50} & -\frac{2126}{25} \\
            0 & 1 & 1 \\
        \end{pmatrix}$
        \\\\\\
        El máximo valor de las columnas y filas de la matriz $A^{(1)}$, por debajo de la primera fila es $a_{32}$; por lo tanto, se tiene $P_2$, $Q_2$ y $M_2$\\
        \\
        $P_2 = I $;
        $Q_2 = \begin{pmatrix}
            1 & 0 & 0 \\ 
            0 & 0 & 1 \\
            0 & 1 & 0 \\
        \end{pmatrix} $
        $\xrightarrow{}$ $P_2A^{(1)}Q_2 = \begin{pmatrix}
            100 & 99 & 98 \\ 
            0 & -\frac{2126}{25} & -\frac{2101}{50} \\
            0 & 1 & 1 \\
        \end{pmatrix}$
        \\\\\\
        $M_2 = \begin{pmatrix}
            1 & 0 & 0 \\ 
            0 & 1 & 0 \\
            0 & \frac{25}{2126} & 1 \\
        \end{pmatrix} $ 
        $\xrightarrow{}$
        $A^{(2)}= M_2P_2A^{(1)}Q_2 = \begin{pmatrix}
            100 & 99 & 98 \\ 
            0 & -\frac{2126}{25} & -\frac{2101}{50} \\
            0 & 1 & \frac{2151}{4252} \\
        \end{pmatrix}$
        \\\\\\
        $\xrightarrow{}$ Se verifica que $MAQ = M_2(P_2(M_1P_1AQ_1)Q_2$ es una matriz triangular superior\\
        
        \item
        \begin{enumerate}[]
            \item Para el caso de pivoteo parcial
            \begin{align*}
                PA &= LU\\
                \begin{pmatrix}
                1 & 0 & 0 \\ 
                0 & 1 & 0 \\
                0 & 0 & 1 \\
                \end{pmatrix}
                \begin{pmatrix}
                100 & 99 & 98 \\ 
                98 & 55 & 11 \\
                0 & 1 & 1 \\
                \end{pmatrix} &= 
                \begin{pmatrix}
                1 & 0 & 0 \\ 
                \frac{98}{100} & 1 & 0 \\
                0 & -\frac{50}{2101} & 1 \\
                \end{pmatrix}
                \begin{pmatrix}
                100 & 99 & 98 \\ 
                0 & -\frac{2101}{50} & -\frac{2126}{25} \\
                0 & 0 & -\frac{2151}{2101} \\
                \end{pmatrix}
            \end{align*}
            \\
            
            \item Para pivoteo completo
            \begin{align*}
                PAQ &= LU\\
                \begin{pmatrix}
                1 & 0 & 0 \\ 
                0 & 1 & 0 \\
                0 & 0 & 1 \\
                \end{pmatrix}
                \begin{pmatrix}
                100 & 99 & 98 \\ 
                98 & 55 & 11 \\
                0 & 1 & 1 \\
                \end{pmatrix}
                \begin{pmatrix}
                1 & 0 & 0 \\ 
                0 & 0 & 1 \\
                0 & 1 & 0 \\
                \end{pmatrix} &= 
                \begin{pmatrix}
                1 & 0 & 0 \\ 
                \frac{98}{100} & 1 & 0 \\
                0 & -\frac{25}{2126} & 1 \\
                \end{pmatrix}
                \begin{pmatrix}
                100 & 98 & 99 \\ 
                0 & -\frac{2126}{25} & -\frac{2101}{50} \\
                0 & 0 & \frac{2151}{4252} \\
                \end{pmatrix}
            \end{align*}
            \\
        \end{enumerate}
        
        \item Factor de crecimiento
        \begin{enumerate}[]
            \item Para el caso del pivoteo parcial
            \begin{align*}
                \rho &=  \frac{max |a_{ij}^{(2)}|}{max |a_{ij}|}\\
                &=  \frac{100}{100}\\
                &=  1\\
            \end{align*}
            
             \item Para pivoteo completo
             \begin{align*}
                \rho &=  \frac{max |a_{ij}^{(2)}|}{max |a_{ij}|}\\
                &=  \frac{100}{100}\\
                &=  1\\
            \end{align*}
        \end{enumerate}        
    \end{enumerate}
    
    %%%%%%%%%%%%%%%%%%%%%%%%%%%%%
    \item 
    $A = \begin{pmatrix}
            1 & 0 & 1 \\ 
            -1 & 1 & 1 \\
            -1 & -1 & 1 \\
        \end{pmatrix}$
    
    \begin{enumerate}[]
        \item Pivoteo parcial\\
        El máximo valor de la primera columna en la matriz $A$ es $a_{11}$; por lo tanto, se tiene $P_1$ y $M_1$\\
        \\
        $P_1 = I$; 
        $M_1 = \begin{pmatrix}
            1 & 0 & 0 \\ 
            1 & 1 & 0 \\
            1 & 0 & 1 \\
        \end{pmatrix} $ 
        $\xrightarrow{}$
        $A^{(1)}= M_1P_1A = \begin{pmatrix}
            1 & 0 & 1 \\ 
            0 & 1 & 2 \\
            0 & -1 & 2 \\
        \end{pmatrix}$
        \\\\\\
        El máximo valor de la segunda columna en la matriz $A^{(1)}$, por debajo de la primera fila es $a_{22}$; por lo tanto, se tiene $P_2$ y $M_2$\\\\
        $P_2 = I$; 
        $M_2 = \begin{pmatrix}
            1 & 0 & 0 \\ 
            0 & 1 & 0 \\
            0 & 1 & 1 \\
        \end{pmatrix} $ 
        $\xrightarrow{}$
        $A^{(2)}= M_2P_2A^{(1)} = \begin{pmatrix}
             1 & 0 & 1 \\ 
            0 & 1 & 2 \\
            0 & 0 & 4 \\
        \end{pmatrix}$
        \\\\\\
        $\xrightarrow{}$ Se verifica que $MA = M_2P_2M_1P_1A$ es una matriz triangular superior\\
        
        \item Pivoteo completo\\
        El máximo valor de todas las columnas y filas de la matriz $A$ es $a_{11}$; por lo tanto, se tiene $P_1$, $Q_1$ y $M_1$\\
        \\
        $P_1 = Q_1 = I$
        $\xrightarrow{}$ $P_1AQ_1 = A$; 
        $M_1 = \begin{pmatrix}
            1 & 0 & 0 \\ 
            1 & 1 & 0 \\
            1 & 0 & 1 \\
        \end{pmatrix} $ \\\\
        $\xrightarrow{}$
        $A^{(1)}= M_1P_1AQ_1 = \begin{pmatrix}
            1 & 0 & 1 \\ 
            0 & 1 & 2 \\
            0 & -1 & 2 \\
        \end{pmatrix}$
        \\\\\\
        El máximo valor de las columnas y filas de la matriz $A^{(1)}$, por debajo de la primera fila es $a_{22}$; por lo tanto, se tiene $P_2$, $Q_2$ y $M_2$\\
        \\
        $P_2 = I $;
        $Q_2 = \begin{pmatrix}
            1 & 0 & 0 \\ 
            0 & 0 & 1 \\
            0 & 1 & 0 \\
        \end{pmatrix} $
        $\xrightarrow{}$ $P_2A^{(1)}Q_2 = \begin{pmatrix}
            1 & 1 & 0 \\ 
            0 & 2 & 1 \\
            0 & 2 & -1 \\
        \end{pmatrix}$
        \\\\\\
        $M_2 = \begin{pmatrix}
            1 & 0 & 0 \\ 
            0 & 1 & 0 \\
            0 & -1 & 1 \\
        \end{pmatrix} $ 
        $\xrightarrow{}$
        $A^{(2)}= M_2P_2A^{(1)}Q_2 = \begin{pmatrix}
            1 & 1 & 0 \\ 
            0 & 2 & 1 \\
            0 & 0 & -2 \\
        \end{pmatrix}$
        \\\\\\
        $\xrightarrow{}$ Se verifica que $MAQ = M_2(P_2(M_1P_1AQ_1)Q_2$ es una matriz triangular superior\\
        
        \item
        \begin{enumerate}[]
            \item Para el caso de pivoteo parcial
            \begin{align*}
                PA &= LU\\
                \begin{pmatrix}
                1 & 0 & 0 \\ 
                0 & 1 & 0 \\
                0 & 0 & 1 \\
                \end{pmatrix}
                \begin{pmatrix}
                1 & 0 & 1 \\ 
                -1 & 1 & 1 \\
                -1 & -1 & 1 \\
                \end{pmatrix} &= 
                \begin{pmatrix}
                1 & 0 & 0 \\ 
                -1 & 1 & 0 \\
                -1 & -1 & 1 \\
                \end{pmatrix}
                \begin{pmatrix}
                1 & 0 & 1 \\ 
                0 & 1 & 2 \\
                0 & 0 & 4 \\
                \end{pmatrix}
            \end{align*}
            \\
            
            \item Para pivoteo completo
            \begin{align*}
                PAQ &= LU\\
                \begin{pmatrix}
                1 & 0 & 0 \\ 
                0 & 1 & 0 \\
                0 & 0 & 1 \\
                \end{pmatrix}
                \begin{pmatrix}
                1 & 0 & 1 \\ 
                -1 & 1 & 1 \\
                -1 & -1 & 1 \\
                \end{pmatrix}
                \begin{pmatrix}
                1 & 0 & 0 \\ 
                0 & 0 & 1 \\
                0 & 1 & 0 \\
                \end{pmatrix} &= 
                \begin{pmatrix}
                1 & 0 & 0 \\ 
                -1 & 1 & 0 \\
                -1 & 1 & 1 \\
                \end{pmatrix}
                \begin{pmatrix}
                1 & 1 & 0 \\ 
                0 & 2 & 1 \\
                0 & 0 & -2 \\
                \end{pmatrix}
            \end{align*}
            \\
        \end{enumerate}
        
        \item Factor de crecimiento
        \begin{enumerate}[]
            \item Para el caso del pivoteo parcial
            \begin{align*}
                \rho &=  \frac{max |a_{ij}^{(2)}|}{max |a_{ij}|}\\
                &=  \frac{4}{1}\\
                &=  1\\
            \end{align*}
            
             \item Para pivoteo completo
             \begin{align*}
                \rho &=  \frac{max |a_{ij}^{(2)}|}{max |a_{ij}|}\\
                &=  \frac{2}{1}\\
                &=  1\\
            \end{align*}
        \end{enumerate}
    \end{enumerate}
    
    %%%%%%%%%%%%%%%%%%%%%%%%%%%%%
    \item 
    $A = \begin{pmatrix}
            0.0003 & 1.566 &  1.234 \\ 
            1.5660 & 2.000 &  1.018 \\
            1.2340 & 1.018 & -3.000 \\
        \end{pmatrix}$
    
    \begin{enumerate}[]
        \item Pivoteo parcial\\
        El máximo valor de la primera columna en la matriz $A$ es $a_{11}$; por lo tanto, se tiene $P_1$ y $M_1$\\
        \\
        $P_1 = I$; 
        $M_1 = \begin{pmatrix}
            1 & 0 & 0 \\ 
            1 & 1 & 0 \\
            1 & 0 & 1 \\
        \end{pmatrix} $ 
        $\xrightarrow{}$
        $A^{(1)}= M_1P_1A = \begin{pmatrix}
            1 & 0 & 1 \\ 
            0 & 1 & 2 \\
            0 & -1 & 2 \\
        \end{pmatrix}$
        \\\\\\
        El máximo valor de la segunda columna en la matriz $A^{(1)}$, por debajo de la primera fila es $a_{22}$; por lo tanto, se tiene $P_2$ y $M_2$\\\\
        $P_2 = I$; 
        $M_2 = \begin{pmatrix}
            1 & 0 & 0 \\ 
            0 & 1 & 0 \\
            0 & 1 & 1 \\
        \end{pmatrix} $ 
        $\xrightarrow{}$
        $A^{(2)}= M_2P_2A^{(1)} = \begin{pmatrix}
             1 & 0 & 1 \\ 
            0 & 1 & 2 \\
            0 & 0 & 4 \\
        \end{pmatrix}$
        \\\\\\
        $\xrightarrow{}$ Se verifica que $MA = M_2P_2M_1P_1A$ es una matriz triangular superior\\
        
        \item Pivoteo completo\\
        El máximo valor de todas las columnas y filas de la matriz $A$ es $a_{11}$; por lo tanto, se tiene $P_1$, $Q_1$ y $M_1$\\
        \\
        $P_1 = Q_1 = I$
        $\xrightarrow{}$ $P_1AQ_1 = A$; 
        $M_1 = \begin{pmatrix}
            1 & 0 & 0 \\ 
            1 & 1 & 0 \\
            1 & 0 & 1 \\
        \end{pmatrix} $ \\\\
        $\xrightarrow{}$
        $A^{(1)}= M_1P_1AQ_1 = \begin{pmatrix}
            1 & 0 & 1 \\ 
            0 & 1 & 2 \\
            0 & -1 & 2 \\
        \end{pmatrix}$
        \\\\\\
        El máximo valor de las columnas y filas de la matriz $A^{(1)}$, por debajo de la primera fila es $a_{22}$; por lo tanto, se tiene $P_2$, $Q_2$ y $M_2$\\
        \\
        $P_2 = I $;
        $Q_2 = \begin{pmatrix}
            1 & 0 & 0 \\ 
            0 & 0 & 1 \\
            0 & 1 & 0 \\
        \end{pmatrix} $
        $\xrightarrow{}$ $P_2A^{(1)}Q_2 = \begin{pmatrix}
            1 & 1 & 0 \\ 
            0 & 2 & 1 \\
            0 & 2 & -1 \\
        \end{pmatrix}$
        \\\\\\
        $M_2 = \begin{pmatrix}
            1 & 0 & 0 \\ 
            0 & 1 & 0 \\
            0 & -1 & 1 \\
        \end{pmatrix} $ 
        $\xrightarrow{}$
        $A^{(2)}= M_2P_2A^{(1)}Q_2 = \begin{pmatrix}
            1 & 1 & 0 \\ 
            0 & 2 & 1 \\
            0 & 0 & -2 \\
        \end{pmatrix}$
        \\\\\\
        $\xrightarrow{}$ Se verifica que $MAQ = M_2(P_2(M_1P_1AQ_1)Q_2$ es una matriz triangular superior\\
        
        \item
        \begin{enumerate}[]
            \item Para el caso de pivoteo parcial
            \begin{align*}
                PA &= LU\\
                \begin{pmatrix}
                1 & 0 & 0 \\ 
                0 & 1 & 0 \\
                0 & 0 & 1 \\
                \end{pmatrix}
                \begin{pmatrix}
                1 & 0 & 1 \\ 
                -1 & 1 & 1 \\
                -1 & -1 & 1 \\
                \end{pmatrix} &= 
                \begin{pmatrix}
                1 & 0 & 0 \\ 
                -1 & 1 & 0 \\
                -1 & -1 & 1 \\
                \end{pmatrix}
                \begin{pmatrix}
                1 & 0 & 1 \\ 
                0 & 1 & 2 \\
                0 & 0 & 4 \\
                \end{pmatrix}
            \end{align*}
            \\
            
            \item Para pivoteo completo
            \begin{align*}
                PAQ &= LU\\
                \begin{pmatrix}
                1 & 0 & 0 \\ 
                0 & 1 & 0 \\
                0 & 0 & 1 \\
                \end{pmatrix}
                \begin{pmatrix}
                1 & 0 & 1 \\ 
                -1 & 1 & 1 \\
                -1 & -1 & 1 \\
                \end{pmatrix}
                \begin{pmatrix}
                1 & 0 & 0 \\ 
                0 & 0 & 1 \\
                0 & 1 & 0 \\
                \end{pmatrix} &= 
                \begin{pmatrix}
                1 & 0 & 0 \\ 
                -1 & 1 & 0 \\
                -1 & 1 & 1 \\
                \end{pmatrix}
                \begin{pmatrix}
                1 & 1 & 0 \\ 
                0 & 2 & 1 \\
                0 & 0 & -2 \\
                \end{pmatrix}
            \end{align*}
            \\
        \end{enumerate}
        
        \item Factor de crecimiento
        \begin{enumerate}[]
            \item Para el caso del pivoteo parcial
            \begin{align*}
                \rho &=  \frac{max |a_{ij}^{(2)}|}{max |a_{ij}|}\\
                &=  \frac{4}{1}\\
                &=  1\\
            \end{align*}
            
             \item Para pivoteo completo
             \begin{align*}
                \rho &=  \frac{max |a_{ij}^{(2)}|}{max |a_{ij}|}\\
                &=  \frac{2}{1}\\
                &=  1\\
            \end{align*}
        \end{enumerate}
    \end{enumerate}

    %%%%%%%%%%%%%%%%%%%%%%%%%%%%%
    \item 
    $A = \begin{pmatrix}
            1 & -1 & 0 \\ 
            -1 & 2 & 0 \\
            0 & -1 & 2 \\
        \end{pmatrix}$
    
    \begin{enumerate}[]
        \item Pivoteo parcial\\
        El máximo valor de la primera columna en la matriz $A$ es $a_{11}$; por lo tanto, se tiene $P_1$ y $M_1$\\
        \\
        $P_1 = I$; 
        $M_1 = \begin{pmatrix}
            1 & 0 & 0 \\ 
            1 & 1 & 0 \\
            0 & 0 & 1 \\
        \end{pmatrix} $ 
        $\xrightarrow{}$
        $A^{(1)}= M_1P_1A = \begin{pmatrix}
            1 & -1 & 0 \\ 
            0 & 1 & 0 \\
            0 & -1 & 2 \\
        \end{pmatrix}$
        \\\\\\
        El máximo valor de la segunda columna en la matriz $A^{(1)}$, por debajo de la primera fila es $a_{22}$; por lo tanto, se tiene $P_2$ y $M_2$\\\\
        $P_2 = I$; 
        $M_2 = \begin{pmatrix}
            1 & 0 & 0 \\ 
            0 & 1 & 0 \\
            0 & 1 & 1 \\
        \end{pmatrix} $ 
        $\xrightarrow{}$
        $A^{(2)}= M_2P_2A^{(1)} = \begin{pmatrix}
             1 & -1 & 0 \\ 
            0 & 1 & 0 \\
            0 & 0 & 2 \\
        \end{pmatrix}$
        \\\\\\
        $\xrightarrow{}$ Se verifica que $MA = M_2P_2M_1P_1A$ es una matriz triangular superior\\
        
        \item Pivoteo completo\\
        El máximo valor de todas las columnas y filas de la matriz $A$ es $a_{22}$; por lo tanto, se tiene $P_1$, $Q_1$ y $M_1$\\
        \\
        $P_1 = \begin{pmatrix}
            0 & 1 & 0 \\ 
            1 & 0 & 0 \\
            0 & 0 & 1 \\
        \end{pmatrix} $;
        $Q_1 = \begin{pmatrix}
            0 & 1 & 0 \\ 
            1 & 0 & 0 \\
            0 & 0 & 1 \\
        \end{pmatrix} $
        $\xrightarrow{}$ $P_1AQ_1 = \begin{pmatrix}
            2 & -1 & 0 \\ 
            -1 & 1 & 0 \\
            -1 & 0 & 2 \\
        \end{pmatrix} $; 
        $M_1 = \begin{pmatrix}
            1 & 0 & 0 \\ 
            \frac{1}{2} & 1 & 0 \\
            \frac{1}{2} & 0 & 1 \\
        \end{pmatrix} $ \\\\
        $\xrightarrow{}$
        $A^{(1)}= M_1P_1AQ_1 = \begin{pmatrix}
            1 & 0 & 1 \\ 
            0 & 1 & 2 \\
            0 & -1 & 2 \\
        \end{pmatrix}$
        \\\\\\
        El máximo valor de las columnas y filas de la matriz $A^{(1)}$, por debajo de la primera fila es $a_{22}$; por lo tanto, se tiene $P_2$, $Q_2$ y $M_2$\\
        \\
        $P_2 = I $;
        $Q_2 = \begin{pmatrix}
            1 & 0 & 0 \\ 
            0 & 0 & 1 \\
            0 & 1 & 0 \\
        \end{pmatrix} $
        $\xrightarrow{}$ $P_2A^{(1)}Q_2 = \begin{pmatrix}
            1 & 1 & 0 \\ 
            0 & 2 & 1 \\
            0 & 2 & -1 \\
        \end{pmatrix}$
        \\\\\\
        $M_2 = \begin{pmatrix}
            1 & 0 & 0 \\ 
            0 & 1 & 0 \\
            0 & -1 & 1 \\
        \end{pmatrix} $ 
        $\xrightarrow{}$
        $A^{(2)}= M_2P_2A^{(1)}Q_2 = \begin{pmatrix}
            1 & 1 & 0 \\ 
            0 & 2 & 1 \\
            0 & 0 & -2 \\
        \end{pmatrix}$
        \\\\\\
        $\xrightarrow{}$ Se verifica que $MAQ = M_2(P_2(M_1P_1AQ_1)Q_2$ es una matriz triangular superior\\
        
        \item
        \begin{enumerate}[]
            \item Para el caso de pivoteo parcial
            \begin{align*}
                PA &= LU\\
                \begin{pmatrix}
                1 & 0 & 0 \\ 
                0 & 1 & 0 \\
                0 & 0 & 1 \\
                \end{pmatrix}
                \begin{pmatrix}
                1 & 0 & 1 \\ 
                -1 & 1 & 1 \\
                -1 & -1 & 1 \\
                \end{pmatrix} &= 
                \begin{pmatrix}
                1 & 0 & 0 \\ 
                -1 & 1 & 0 \\
                -1 & -1 & 1 \\
                \end{pmatrix}
                \begin{pmatrix}
                1 & 0 & 1 \\ 
                0 & 1 & 2 \\
                0 & 0 & 4 \\
                \end{pmatrix}
            \end{align*}
            \\
            
            \item Para pivoteo completo
            \begin{align*}
                PAQ &= LU\\
                \begin{pmatrix}
                1 & 0 & 0 \\ 
                0 & 1 & 0 \\
                0 & 0 & 1 \\
                \end{pmatrix}
                \begin{pmatrix}
                1 & 0 & 1 \\ 
                -1 & 1 & 1 \\
                -1 & -1 & 1 \\
                \end{pmatrix}
                \begin{pmatrix}
                1 & 0 & 0 \\ 
                0 & 0 & 1 \\
                0 & 1 & 0 \\
                \end{pmatrix} &= 
                \begin{pmatrix}
                1 & 0 & 0 \\ 
                -1 & 1 & 0 \\
                -1 & 1 & 1 \\
                \end{pmatrix}
                \begin{pmatrix}
                1 & 1 & 0 \\ 
                0 & 2 & 1 \\
                0 & 0 & -2 \\
                \end{pmatrix}
            \end{align*}
            \\
        \end{enumerate}
        
        \item Factor de crecimiento
        \begin{enumerate}[]
            \item Para el caso del pivoteo parcial
            \begin{align*}
                \rho &=  \frac{max |a_{ij}^{(2)}|}{max |a_{ij}|}\\
                &=  \frac{2}{2}\\
                &=  1\\
            \end{align*}
            
             \item Para pivoteo completo
             \begin{align*}
                \rho &=  \frac{max |a_{ij}^{(2)}|}{max |a_{ij}|}\\
                &=  \frac{2}{2}\\
                &=  1\\
            \end{align*}
        \end{enumerate}
    \end{enumerate}
\end{enumerate}