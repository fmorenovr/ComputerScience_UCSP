Dado el sistema $Ax=b$, donde:

$$
    A=\begin{pmatrix}
     1 &  2 & -2 \\
     1 &  1 &  1 \\
     2 &  2 &  1 
    \end{pmatrix},\ 
    b=\begin{pmatrix}
    1\\
    4\\
    5
    \end{pmatrix}
$$

¿Es posible resolver el sistema por los métodos de Gauss-Seidel y Jacobi? en caso de ser factible halle la solución.

\textbf{Solución:}

Para saber si es posible resolver el sistema por los métodos mencionados se debe hallar el radio de convergencia de cada método:

$$
    B_{J}=\begin{pmatrix}
     2 &  -2 & 2 \\
     -1 &  0 &  -1 \\
     -2 &  -2 &  0 
    \end{pmatrix},\ \rho(B_{J}) = 1.0809*10^{-5}
$$

y

$$
    B_{GS}=\begin{pmatrix}
     0 &  -2 & 2 \\
     0 &  -2 &  1 \\
     0 &  -8 &  6 
    \end{pmatrix},\ \rho(B_{GS}) = 4.8284
$$

De donde se observa que solo es posible resolver el sistema usando el método de Jacobi. Entonces:

Siendo:

$$
    b_{J}=D^{-1}b = \begin{pmatrix}
    1\\
    4\\
    5
    \end{pmatrix}
$$

Entonces:

$$
    x^{(0)}=\begin{pmatrix}
    0\\
    0\\
    0
    \end{pmatrix}
$$

$$
    x^{(1)}=\begin{pmatrix}
    -1\\
    -4\\
    -5
    \end{pmatrix}
$$

$$
    x^{(2)}=\begin{pmatrix}
    -3\\
    2\\
    5
    \end{pmatrix}
$$

$$
    x^{(3)}=\begin{pmatrix}
    5\\
    -6\\
    -3
    \end{pmatrix}
$$

$$
    x^{(4)}=\begin{pmatrix}
    5\\
    -6\\
    -3
    \end{pmatrix}
$$

Se puede observar que en la tercera iteración ya se alcanzó la solución:

$$
    x=\begin{pmatrix}
    5\\
    -6\\
    -3
    \end{pmatrix}
$$

