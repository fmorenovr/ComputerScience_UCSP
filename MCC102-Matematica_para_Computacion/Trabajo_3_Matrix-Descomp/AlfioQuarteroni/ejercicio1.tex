El radio espectral de la matriz 
\begin{center}
\[
B
=
\begin{pmatrix}
     $a$ &  $4$\\
     $0$ &  $a$\\
\end{pmatrix}
\]
\end{center}
es $p(B) = a$. Verificar que si $0<a<1$, entonces $p(B)<1$, mientras $||B^m||_2^{\frac{1}{m}}$ puede ser mas grande que 1 

\textbf{Solución:}

Matriz B multiplicada $m$ veces:
\[
    B^m = B * B * B ... B
\]
Entonces, para la norma 2 de $B^m$ se multiplica m veces B por m veces $B^T$:
\[
    ||B^m||_2 = \sqrt{max \ \lambda \ de \ (B^m)^T*B^m}
\]
Sabemos que $p(B^m*(B^m)^T) = ||B^m||_2$, entonces:
\[
    ||B^m||_2 = p(B*B*B...*B*B^T*B^T*B^T*...*B^T)   
\]
Considerando que los autovalores de la matriz $B$ coinciden con los de $B^T$ y por propiedad de radio espectral, tenemos radio espectral de $B$ multiplicado $2m$ veces:
\[
    ||B^m||_2 \leq p(B)*p(B)*...*p(B)
\]
\[
    ||B^m||_2 \leq p^{2m}(B)
\]
\[
    ||B^m||_2^{\frac{1}{m}} \leq p^{2}(B)
\]
Por definición $p(B) = a$
\[
    ||B^m||_2^{\frac{1}{m}} \leq a^2
\]
$0 < a < 1$, entonces $||B^m||_2^{\frac{1}{m}}$ no sera mas grande que 1.

Por lo tanto, $||B^m||_2^{\frac{1}{m}} \leq 1 $
