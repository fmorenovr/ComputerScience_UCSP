Considere el sistema lineal $Ax=b$ con:\\\\
\begin{center}
A = 
\[
\begin{bmatrix}
    62 & 24 &  1 &  8 & 15 \\
    23 & 50 &  7 & 14 & 16 \\
     4 &  6 & 58 & 20 & 22 \\
    10 & 12 & 19 & 66 &  3 \\
    11 & 18 & 25 &  2 & 54
\end{bmatrix}
, b = 
\begin{bmatrix}
    110 \\
    110 \\
    110 \\
    110 \\
    110
\end{bmatrix}
\]
\end{center}

\begin{itemize}

\item \textbf{Verificar si los métodos de Jacobi y Gauss-Seidel pueden ser aplicados para resolver el sistema.}\\
Como A no es diagonal dominante o simétrica definida positiva, entonces es necesario calcular el radio espectral de las matrices de iteracion de Jacobi y Gauss-Seidel:

\begin{align}
    B_j  &= -D^{-1}(L+U)  \Rightarrow \rho(B_j) &= 0.9280 \\
    B_gs &= -(D+L)^{-1}U  \Rightarrow \rho(B_gs)&= 0.3066
\end{align}~\\
Como ambos radios espectrales son menores a $1$, entonces el sistema converge con ambos métodos.

\item \textbf{Verificar si el método estacionario de Richardson con parámetro óptimo puede ser aplicado con $P=I$ y $P=D$, donde $D$ es la parte diagonal de A y calcule los valores de $\alpha_{opt}$ y  $\phi_{opt}$.}\\
\begin{align}
    R_p = I - P^{-1}A \Rightarrow R_{(\alpha_k)} = I - \alpha_k P^{-1}A
\end{align}
Para el método estacionario de Richardson:
\begin{align}
(\alpha_k = \alpha) \Rightarrow R_{\alpha} = I - \alpha P^{-1} A  
\end{align}
La iteración $k+1$ del sistema tiene la siguiente forma.
\begin{align}
    x^{k+1} &= x^k + \alpha P^{-1} r^k
\end{align}
Sabiendo que $\alpha_{opt} = \frac{2}{\lambda_1 + \lambda_n}$ y $P^{-1}A$ tiene autovalores reales positivos, es decir:
\begin{align}
    \lambda_1 \geq \lambda_2 \geq \dots \geq \lambda_0 \geq 0
\end{align}
Luego, el sistema converge si y solo si $0 < \alpha < \lambda_1$ y el radio espectral de la matriz iterativa $R_{\alpha}$ es el mínimo si $\alpha = \alpha_{opt}$ con:
\begin{align}
    \rho_{opt} &= min(\rho_{R_{\alpha}}) = \frac{\lambda_1 - \lambda_n}{\lambda_1 + \lambda_n}
\end{align}~\\\\
Caso 1: $P = I$\\\\
\begin{align}
    P^{-1}A = I^{-1}A = A \Rightarrow \lambda(A) &= \{110;66.2;58.1;31.8;23.7\} \\
    \alpha_{opt} &= 0.00149 \\
    \rho_{opt} &= 0.6452
\end{align}
Los autovalores de A son reales positivos y $\phi_{opt} < 1$, por lo que el método puede ser aplicado.\\\\

Caso 2: $P = D$\\\\
\begin{align}
    P^{-1}A = D^{-1}A \Rightarrow \lambda(D^{-1}A) &= \{1.92;1.12;0.95;0.577;0.422\} \\
    \alpha_{opt} &= 0.8510 \\
    \rho_{opt} &= 0.0.6407
\end{align}
Los autovalores de A son reales positivos y $\phi_{opt} < 1$, por lo que el método puede ser aplicado.

\end{itemize}