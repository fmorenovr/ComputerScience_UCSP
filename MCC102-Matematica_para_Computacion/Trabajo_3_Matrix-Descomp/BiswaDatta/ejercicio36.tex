Considere el siguiente sistema lineal

\begin{equation*}
    A x = b
\end{equation*}

\begin{equation*}
    \begin{pmatrix}
        4 & -1 & -1 & 0 \\
        -1 & 4 & 0 & -1 \\
        -1 & 0 & 4 & -1 \\
        0 & -1 & -1 & 4
    \end{pmatrix}
    \begin{pmatrix}
        x_1 \\
        x_2 \\
        x_3 \\
        x_4
    \end{pmatrix}    
    =    
    \begin{pmatrix}
        2 \\
        2 \\
        2 \\
        2
    \end{pmatrix}
\end{equation*}


\begin{itemize}
    \item ¿Por qué ambos métodos, Jacobi y Gauss-Seidel, convergen con una aproximación inicial elegida arbitrariamente para este sistema?
    
    \item Realice 5 iteraciones para ambos métodos con la misma aproximación inicial.
\end{itemize}

\textbf{Solución:}

\begin{itemize}
    \item El sistema del enunciado converge con ambos métodos, Jacobi y Gauss-Seidel para cualquier aproximación inicial porque cumple con las siguientes condiciones.
    
    \begin{itemize}
        \item La matriz $A$ es diagonal dominante, esto es, el valor de la diagonal es el mayor valor de las filas y columnas en la matriz.
        
        \item El radio espectral de $B$ es menor a 1. $\rho(B) < 1$
        
        \begin{itemize}
            \item Para el método de Jacobi\\
            \begin{equation*}
                B = 
                \begin{pmatrix}
                    0 & \frac{1}{4} & \frac{1}{4} & 0 \\
                    \frac{1}{4} & 0 & 0 & \frac{1}{4} \\
                    \frac{1}{4} & 0 & 0 & \frac{1}{4} \\
                    0 & \frac{1}{4} & \frac{1}{4} & 0
                \end{pmatrix}
            \end{equation*}
            \\
            \begin{equation*}
                \rho(B) = 0.5
            \end{equation*}
            
            %\item Para el método de Gauss-Seidel\\
            %\begin{equation*}
            %    B = 
            %    \begin{pmatrix}
            %        0 & \frac{1}{4} & \frac{1}{4} & 0 \\
            %        \frac{1}{4} & 0 & 0 & \frac{1}{4} \\
            %        \frac{1}{4} & 0 & 0 & \frac{1}{4} \\
            %        0 & \frac{1}{4} & \frac{1}{4} & 0
            %    \end{pmatrix}
            %\end{equation*}
            %\\
            %\begin{equation*}
            %    \rho(B) = 0.5
            %\end{equation*}
        \end{itemize}
    \end{itemize}
    
     \item 
    
    \begin{itemize}
        \item Para el método de Jacobi después de 5 iteraciones:
        \begin{equation*}
            x = 
            \begin{pmatrix}
                -0.125\\
                -0.25\\
                -0.25\\
                0.0
            \end{pmatrix}
        \end{equation*}
    
        %\item Para el método de Gauss-Seidel después de 5 iteraciones:
        %\begin{equation*}
        %    x = 
        %    \begin{pmatrix}
        %        -0.125\\
        %        -0.25\\
        %        -0.25\\
        %        0.0
        %    \end{pmatrix}
        %\end{equation*}
    \end{itemize} 
    
\end{itemize}