    Construye un ejemplo para mostrar que cuando el metodo de Jacobi converge, no necesariamente
    implica que el metodo de Gauss-Seidel también converge.\\\\
    Ya que la convergencia de estos algoritmos depende de su radio espectral
    Para un sistema: 
    \[
        x^{(k+1)} = B*x^{(k)} + c
    \]
    La condición necesaria y suficiente para la convergencia de la iteración es que: $\rho(B) < 1$.  \textit{(B. Datta, 6.10.2)}\\ 

    El teórema de \textit{Stein-Rosenberg}, dice lo siguiente para una matriz con diagonal positiva.
    y los demás términos de la matriz negativos.
    \begin{itemize}
        \item Jacobi y Gauss Seidel o bien convergen o bien divergen.
        \item Cuando ambos convergen el metodo de Gauss-Seidel converge más rápido
        que el metodo de Jacabi.
    \end{itemize}
    Ya que en este teorema no esta contemplado el caso que buscamos, esto es: \textit{Si el método
    de Jacobi converge, el metodo de Gauss Seidel no lo hace}.
    Se puede buscar matrices donde los términos de la diagonal no son todos positivos, y/ó donde los otros términos
    no necesariamente son negativos.
    
    Sea el sistema $3x3$ 
    \[
        A =
            \begin{bmatrix}
                1 & 2 & -2 \\
                1 & 0.9 & 1\\
                2 & 2 & 1
            \end{bmatrix}
    \]
    Por tanto: 
    \[
        T_{J} =
            \begin{bmatrix}
                0 & 2 & -2 \\
                1.11 & 0 & 1.11\\
                2 & 2 & 0
            \end{bmatrix}   
    \]
    
    y: 
    \[
        T_{GS} =
            \begin{bmatrix}
                0 & 2 & -2 \\
                0 & 2.22 & 1.11\\
                0 & 8.44 & -6.22
            \end{bmatrix}   
    \]
    Calculando el radio espectral para ambas matrices:
    \[
            \rho(T_{J}) = 0.666
    \]
    \[
            \rho(T_{GS}) = 4.90
    \]
    Por tanto para este caso el metodo de Jacobi converge y el método de Gauss-Seidel no.
