Deje que la matriz A de n x n sea particionada dentro de la forma
\[\begin{pmatrix} 
A_{11} & A_{12} & \cdots & A_{1N} \\ 
A_{21} & A_{22} & \cdots & A_{2N} \\ 
\vdots & \vdots &     & \vdots \\ 
A_{N,1} & A_{N,2} & \cdots & A_{N,N}  
\end{pmatrix} \]
donde cada bloque de la diagonal $A_{ii}$ es cuadrado y no singular. Considere el sistema lineal
\[Ax=b\]
con A como el anterior y $x$ y $b$ particionados de manera proporcional.
\begin{itemize}
\item Escriba el bloque Jacobi, bloque Gauss-Seidel, y el bloque de iteraciones SOR para el sistema lineal $Ax =b$. (Consejo: Escribe $A=L+D+U$, donde $D=diag(A_{11},...,A_{NN})$ y $L$ y $U$ son estrictamente bloques superiores e inferiores matrices triangulares)\\
\textbf{Solución}\\
Bloque de Iteración de Jacobi:\\
\[A_{ii}X_i^{k+1} = B_i - \sum_{j=1, j\neq i}^N A_{ij}X_j^{(k)} , i=1,2,...,N\]
Bloque de Iteración de Gauss-Seidel:\\
\[A_{ii}X_i^{k+1} = B_i - \sum_{j=1}^{i-1} A_{ij}X_j^{(k+1)} - \sum_{j=i+1}^N A_{ij}X_j^{(k)}, i=1,2,...,N\]
Bloque de iteración del SOR:\\
\[A_{ii}X_i^{k+1} =\omega(B_i - \sum_{j=1}^{i-1} A_{ij}X_j^{(k+1)} - \sum_{j=i+1}^N A_{ij}X_j^{(k)}) + (1-\omega)X_i^{(k)}, i=1,2,...,N\]

\item Si $A$ es simétrica definida positiva, luego muestra que $U=L^T$ y $D$ es positiva definida. En este caso, usando los resultados correspondientes en los casos escalares, pruebe que, con una elección arbitraria de la aproximación inicial, el bloque Gauss-Seidel siempre converge y el bloque SOR converge si y solo si $0<\omega<2$\\
\textbf{Solución}\\
SAbiendo que:\\
\[A= L + D + U\]
\[L_{ij} =
\begin{cases} 
      A_{ij}    & \mbox{si }  j < i  \\
      0         & \mbox{si } i \leq j 
\end{cases}\]
\[D_{ij} =
\begin{cases} 
      A_{ij}    & \mbox{si }  j = i  \\
      0         & \mbox{si } i \neq j 
\end{cases}\]
\[U_{ij} =
\begin{cases} 
      A_{ij}    & \mbox{si }  j > i  \\
      0         & \mbox{si } i \geq j 
\end{cases}\]
Si A es definida positiva, entonces mostrar que $U= L ^T$ y $D$ es positiva.\\
Muestra que $U=L^T$:
\[\text{Si A es simétrica} =
\begin{cases} 
      A_{ij}^T=A_{ji}    & \mbox{si }  j \neq i  \\
      A_{ij} \text{ es simétrica} & \mbox{si } i = j 
\end{cases}\]
$$
    U=\begin{pmatrix}
      0 &  A_{12} & A_{13} & \ldots &  A_{1N} \\
      0 &  0      & A_{23} & \ldots &  A_{2N} \\
      \vdots & \vdots  & \ddots & \vdots & \vdots \\
      0 &  0      & 0      & \ldots &  A_{N-1 N} \\
      0 &  0      & 0      & \ldots &  0 
    \end{pmatrix},
    L=\begin{pmatrix}
      0 &  0 & 0 & \ldots &  0 \\
      A_{21} &  0   & 0 & \ldots &  0 \\
      \vdots & \vdots  & \ddots & \vdots & \vdots \\
      A_{N-1 1} &  A_{N-1 2}      & A_{N-1 3}      & \ldots &  0 \\
      A_{N 1} &  A_{N 2}      & A_{N 3}      & \ldots &  0 
    \end{pmatrix},
$$
$$
    L^T=\begin{pmatrix}
      0 &  A_{21}^T & A_{31}^T & \ldots &  A_{N1}^T \\
      0 &  0      & A_{32}^T & \ldots &  A_{N2}^T \\
      \vdots & \vdots  & \ddots & \vdots & \vdots \\
      0 &  0      & 0      & \ldots &  A_{N N-1 }^T \\
      0 &  0      & 0      & \ldots &  0 
    \end{pmatrix}
    =\begin{pmatrix}
      0 &  A_{12} & A_{13} & \ldots &  A_{1N} \\
      0 &  0      & A_{23} & \ldots &  A_{2N} \\
      \vdots & \vdots  & \ddots & \vdots & \vdots \\
      0 &  0      & 0      & \ldots &  A_{N-1 N} \\
      0 &  0      & 0      & \ldots &  0 
    \end{pmatrix} L^T =U
$$
Entonces para mostrar que $D$  es definida positiva partimos de que $A$ es definida positiva, es decir $a_{ii}>0$
Entonces tendríamos a $D$:
$$
    D=\begin{pmatrix}
      0 &  A_{11} & 0 & \ldots &  0 \\
      0 &  A_{22} & 0 & \ldots &  0 \\
      \vdots & \vdots  & \ddots & \vdots & \vdots \\
      0 &  0      & 0      & \ldots &  0 \\
      0 &  0      & 0      & \ldots &  A_{N N} 
    \end{pmatrix}
$$
$diag(A) = diag(D)$ entonces $d_{ii}>0$
Por lo tanto se demuestra que $S$ es definida positiva.

\textbf{Probar que, con una elección arbitraria de la aproximación inicial, el bloque Gauss-Seidel siempre converge y el bloque SOR converge si y solo si $0<\omega<2$}\\
\textbf{Probando que el bloque Gauss-Seidel converge}\\
$X^{(k+1)} = B_{GS}X^{(k)} + B_{GS}$, converge si $B_{GS}\rightarrow 0$ para $k \rightarrow \infty \Longleftrightarrow \rho(B_{GS})<1$
Probaremos que $\rho(B_{GS})<1$.\\
Partimos de que $A$ es  simétrica:\\
\[A=L+ D+ L^T\]
Entonces $B_{GS} = - (D+L)^{-1}L^T$ y si $-\lambda$ es un autovalor de $B_{GS}$ y $u$ su correspondiente autovector.
\[(D+L)^{-1}L^Tu = \lambda u\]
\[L^T u =\lambda(D+L)u\]
\[u*L^T u = \lambda u*(D+L)u, u*=( \bar{u})^T\]
\[u*Au - u*(L+D)u = \lambda u*(L+D)u, L^T = A-(L+D)\]
\[u*Au=(\lambda+1)u*(L+D)u\]
Tomando la conjugada transpuesta:
\[u*Au = ( \bar{\lambda} + 1)u*(L^T+D^T)u\]
Al considerar I y II se tiene:
\[(\frac{1}{1+\lambda}+\frac{1}{1+\lambda})u*Au=u*(L+D)u + u*(L^T+D^T)u\]
\[=u*(L+D+L^T+D^T)u\]
\[=u*(A+D^T)u , D^T =D\]
\[=u*(A+D)u\]
Como $A$ y $D$ son definidos positivos, $(u*Au > 0, u*Au>0)$.
\[u*(A+D)u > u*Au\]
\[(\frac{1}{1+\lambda}+\frac{1}{1+ \bar{\lambda}})u*Au > u*Au\]
\[\frac{1}{1+\lambda}+\frac{1}{1+\lambda} > 1\]
\[ \frac{2+\lambda + \bar{\lambda}}{(1+\lambda)(1+ \bar{\lambda})}> 1\]
Sea $\lambda=\alpha+i\beta $, $ \bar{\lambda}=\alpha-i \beta$
\[\frac{2(1+\alpha)}{(1+\alpha)^2 + \beta^2}>1\]
\[1 > |\lambda|, \text{para todo } \lambda \text{de A}\]
\[1 > |\lambda_{MAX}|, \rho(B_{GS})<1\]
Por lo tanto si $A$ es matriz definida positiva y simétrica,$B_{GS} $converge\\
\textbf{Probando que el bloque SOR converge}\\
EL método converge sí $B_{SOR} $ converge siendo:
\[B_{SOR}= (D+WL)^{-1}[(1-W)D - WU]\]
$$
    (D+WL)^{-1}=\begin{pmatrix}
      A_{11}^{-1} &  0 & 0 & \ldots &  0 \\
      * &  A_{22}^{-1} & 0 & \ldots &  0 \\
      \vdots & \vdots  & \ddots & \vdots & \vdots \\
      * &  *      & *      & \ldots &  0 \\
      * &  *      & *      & \ldots &  A_{N N}^{-1} 
    \end{pmatrix},\text{ triangular inferior}
$$

$$
    |(1-W)D - WU|=\begin{pmatrix}
      (1-\omega)A_{11} &  * & * & \ldots &  * \\
      0 &  (1-\omega)A_{22} & * & \ldots &  * \\
      \vdots & \vdots  & \ddots & \vdots & \vdots \\
      0 & 0  & 0 & \ldots &  0 \\
      0 & 0  & 0 & \ldots &  (1-\omega)A_{N N}
    \end{pmatrix},\text{ triangular inferior}
$$
\[\det(B_{SOR}) = det((D+WL)^{-1})det[(1-W)D - WU]\]
\[\prod_{n=1}^N |A_{NN}^{-1}|\prod_{n=1}^N(1-\omega)^P|A_{NN}|\]
\[det(B_{SOR}) = (1-\omega)^{Np}, (A_{ij} \textbf{es de orden pxp})\]
Como el determinante de una matriz es igual al producto de sus autovalores.
\[\rho(B_{SOR}) \geq |1-\omega|\]
\[\rho(B_{SOR}) \geq 1-\omega \cup \rho(B_{SOR}) \geq \omega - 1, \text{además} \rho(B_{SOR})<1\]
\[\omega \geq 0 \cup 2\geq \omega\]
Por lo tanto se cumple que $0\leq \omega \leq 2$

\end{itemize}

