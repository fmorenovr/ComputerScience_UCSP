Derivar el sistema lineal usando la aproximación de Diferencias Finitas en la ecuación diferencial parcial elíptica:
\begin{center}
  $ \frac{\partial^2 T}{\partial x^2} + \frac{\partial^2 T}{\partial y^2} = f(x,y)$    
\end{center}

El paso de las diferencias son: $ \Delta x = \Delta y = 0.25 $ y las condiciones de frontera son:
\begin{center}
  $T(x,0) = 1-x$\\
  $T(1,y)=y$\\
  $T(0,y)=1$\\
  $T(x,1)=1$    
\end{center}

\textbf{Solución}:

De los datos, obtenemos los dominios del problema [0, 1].

Entonces haciendo diferencias finitas tenemos:
\begin{center}
  $ \frac{\partial^2 T(x_i , y_i)}{\partial x^2} = \frac{T_{i+1,j} - 2 T_{i,j} + T_{i-1,j}}{\Delta x ^2}$\\
  $ \frac{\partial^2 T(x_i , y_i)}{\partial y^2} = \frac{T_{i,j+1} - 2 T_{i,j} + T_{i,j-1}}{\Delta y ^2}$\\
\end{center}

Entonces, por aproximación tenemos que:
\begin{center}
  $f(x,y) = \frac{T_{i+1,j} - 2 T_{i,j} + T_{i-1,j}}{\Delta x ^2} + \frac{T_{i,j+1} - 2 T_{i,j} + T_{i,j-1}}{\Delta y ^2}$\\
  $f(x,y) = \frac{T_{i+1,j} + T_{i-1,j} - 4 T_{i,j} + T_{i,j+1} + T_{i,j-1}}{0.25 ^2} $\\
  $f(x,y) = 16*T_{i+1,j} + 16*T_{i-1,j} - 48* T_{i,j} + 16*T_{i,j+1} + 16*T_{i,j-1} $\\
\end{center}

Como el intervalo es de [0,1] con paso de 0.25, tenemos 16 puntos: 0, 0.25, 0.5, 0.75, 1 en X y Y. Generando una malla en dichos intervalos.

De los cuales, faltaría calcular los puntos: (0.25, 0.25), (0.25, 0.5), (0.25, 0.75), (0.5,0.25), (0.5,0.5), (0.5, 0.75), (0.75, 0.25), (0.75, 0.5) y (0.75, 0.75).\\

Los cuales se calculan como:

i=1, j=1(0.25,0.25):\\
$f(x_1,y_1) = 16*T_{2,1} + 16*T_{0,1} - 48 T_{1,1} + 16*T_{1,2} + 16*T_{1,0}$\\

i=1, j=2(0.25,0.5):\\
$f(x_1,y_2) = 16*T_{2,2} + 16*T_{0,2} - 48 T_{1,2} + 16*T_{1,3} + 16*T_{1,1}$\\

i=1, j=3(0.25,0.75):\\
$f(x_1,y_3) = 16*T_{2,3} + 16*T_{0,3} - 48 T_{1,3} + 16*T_{1,4} + 16*T_{1,2}$\\

i=2, j=1(0.5,0.25):\\
$f(x_2,y_1) = 16*T_{3,1} + 16*T_{1,1} - 48 T_{2,1} + 16*T_{2,2} + 16*T_{2,0}$\\

i=2, j=2(0.5,0.5):\\
$f(x_2,y_2) = 16*T_{3,2} + 16*T_{1,2} - 48 T_{2,2} + 16*T_{2,3} + 16*T_{2,1}$\\

i=2, j=3(0.5,0.75):\\
$f(x_2,y_3) = 16*T_{3,3} + 16*T_{1,3} - 48 T_{2,3} + 16*T_{2,4} + 16*T_{2,2}$\\

i=3, j=1(0.5,0.25):\\
$f(x_3,y_1) = 16*T_{4,1} + 16*T_{2,1} - 48 T_{3,1} + 16*T_{3,2} + 16*T_{3,0}$\\

i=3, j=2(0.5,0.5):\\
$f(x_3,y_2) = 16*T_{4,2} + 16*T_{2,2} - 48 T_{3,2} + 16*T_{3,3} + 16*T_{3,1}$\\

i=3, j=3(0.5,0.75):\\
$f(x_3,y_3) = 16*T_{4,3} + 16*T_{2,3} - 48 T_{3,3} + 16*T_{3,4} + 16*T_{3,2}$\\

Calculando los respectivos valores:

$T_{0,1} = T(0, 0.25) = 1 $, $T_{4,1} = T(1,0.25) = 0.25$\\
$T_{0,2} = T(0, 0.5) = 1$, $T_{4,2} = T(1,0.5) = 0.5 $\\
$T_{0,3} = T(0, 0.75) = 1$, $T_{4,3} = T(1,0.75) = 0.75 $\\
$T_{1,0} = T(0.25, 0) = 0.75$, $T_{2,0} = T(0.5, 0) = 0.5$, $T_{3,0} = T(0.75, 0) = 0.25$\\
$T_{1,4} = T(0.25, 1) = 1$, $T_{2,4} = T(0.5,1) = 1$, $T_{3,4} = T(0.75,1) = 1$\\
Solo quedaría calcular los puntos:

$T_{1,1} $, $T_{2,1} $, $T_{3,1} $\\
$T_{1,2}$, $T_{2,2} $, $T_{3,2} $\\
$T_{1,3} $, $T_{2,3} $, $T_{3,3} $\\
Los cuales se obtiene de la solución del sistema de ecuaciones formado arriba, el cual se expresa de la siguiente manera:

$f(x_1,y_1) = 16*T_{2,1} + 16 - 48 T_{1,1} + 16*T_{1,2} + 16*0.25$\\
$f(x_1,y_2) = 16*T_{2,2} + 16 - 48 T_{1,2} + 16*T_{1,3} + 16*T_{1,1}$\\
$f(x_1,y_3) = 16*T_{2,3} + 16 - 48 T_{1,3} + 16 + 16*T_{1,2}$\\
$f(x_2,y_1) = 16*T_{3,1} + 16*T_{1,1} - 48 T_{2,1} + 16*T_{2,2} + 16*0.5$\\
$f(x_2,y_2) = 16*T_{3,2} + 16*T_{1,2} - 48 T_{2,2} + 16*T_{2,3} + 16*T_{2,1}$\\
$f(x_2,y_3) = 16*T_{3,3} + 16*T_{1,3} - 48 T_{2,3} + 16 + 16*T_{2,2}$\\
$f(x_1,y_1) = 16*0.25 + 16*T_{2,1} - 48 T_{3,1} + 16*T_{3,2} + 16*0.25$\\
$f(x_2,y_2) = 16*0.5 + 16*T_{2,2} - 48 T_{3,2} + 16*T_{3,3} + 16*T_{3,1}$\\
$f(x_3,y_3) = 16*0.75 + 16*T_{2,3} - 48 T_{3,3} + 16 + 16*T_{3,2}$\\

Dandole una forma de Matrices y vectores tenemos:
\[
\begin{bmatrix}
    -48 & 16 & 0 & 16 & 0 & 0 & 0 & 0\\
    16 & -48 & 16 & 0 & 16 & 0 & 0 & 0 & 0 \\
    0 &  0 & -48 & 16 & 0 & 16 & 0 & 0 & 0 \\
    16 & 0 & 0 & -48 & 16 & 0 & 16 & 0 & 0 \\
    0 & 16 & 0 & 16 & -48 & 16 & 0 & 16 & 0 \\
    0 & 0 & 16 & 0 & 16 & -48 & 0 & 0 & 16  \\
    0 & 0 & 0& 16 & 0 & 0 & -48 & 16 & 0\\
    0 & 0 & 0 & 0 & 16 & 0 & 16 & -48 & 16 \\
    0 & 0 & 0 & 0 & 0 & 16 & 0 & 16 & -48 \\
\end{bmatrix}
+
\begin{bmatrix}
    T_{1,1} \\
    T_{1,2} \\
    T_{1,3} \\
    T_{2,1} \\
    T_{2,2} \\
    T_{2,3} \\
    T_{3,1} \\
    T_{3,2} \\
    T_{3,3} 
\end{bmatrix}
=
\begin{bmatrix}
    f(0.25, 0.25) - 20 \\
    f(0.25, 0.5) - 16\\
    f(0.25, 0.75) - 32\\
    f(0.5, 0.25) - 8\\
    f(0.5, 0.5) \\
    f(0.5, 0.75) - 16\\
    f(0.75, 0.25) -8 \\
    f(0.75, 0.5) - 8 \\
    f(0.75, 0.75) - 28 \\
\end{bmatrix}
\]

Como se puede observar, se generó una matriz pentadiagonal (2 vecinos iguales de la diagonal), para resolver dicha ecuación depende exclusivamente de la función f(x,y).