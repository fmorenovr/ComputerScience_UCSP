Se ha visto en el Teorema 8.41 que $\rho(A)$ es un valor singular de $A$ si $A$ es positiva. En este ejercicio, se usará la extensión de este resultado que dice que $\rho(A)$ es un valor singular de $A$ si $A$ es no negativa. Para los siguientes enunciados asuma que $A$ es una matriz no negativa de $m$ x $m$.

\begin{itemize}
    \item Muestre que $\rho(I_m + A) = 1 + \rho(A)$
    
    \item Muestre que si $ A^k > (0) $ para algún entero positivo $k$, entonces $\rho(A)$ es un simple valor singular de $A$
    
    \item Aplique b) en la matriz $(I_m + A)$ para probar el Teorema 8.49; esto es, probar que para cualquier matriz no negativa irreducible $A$, $\rho(A)$ debe ser un simple valor singular.
\end{itemize}


\textbf{Solución:}

\begin{itemize}

\item Muestre que $\rho(I_m + A) = 1 + \rho(A)$\\
    
    Partimos de la ecuación que nos permite obtener los autovalores de una matriz. Denominaremos $\lambda_1$ a los autovalores de esta matriz.
    
    \begin{equation*}
        det(A - \lambda_1 I) = 0
    \end{equation*}
    
    Definimos una segunda matriz $B = A + I$, y denominaremos a sus autovalores como $\lambda_2$. 
    
    %\[  
        $$det(B - \lambda_2 I) = 0$$
        $$det(A + I - \lambda_2 I) = 0 $$
        $$det(A - I ( \lambda_2 - 1) = 0 $$
    %\]
    
    Si igualamos ambas expresiones, se tiene que:
 
    $$\lambda_1 = \lambda_2 - 1$$
    
    Suponiendo que ambos $\lambda$ ($\lambda_1$ y $\lambda_2$) representan los radios espectrales de las matrices $ A $ y $ I_m + A $ respectivamente, se concluye que:
    
    $$\rho(A) = \rho(I_m + A) - 1$$
    
\item Muestre que si $ A^k > (0) $ para algún entero positivo $k$, entonces $\rho(A)$ es un simple valor singular de $A$\\
    
    Primero, partiendo del presente enunciado, que indica que $ A^k > (0) $, se puede concluir que la matriz $A$ es positiva, ya que para que una potencia de la matriz $A$ sea positiva, cada componente de la matriz $A$ debió ser positivo y diferente de cero. Por lo tanto, la matriz $A$ no solo es no negativa (como lo indica el enunciado general), sino que es positiva.
    
    Segundo, existe un Teorema llamado el Teorema de Perron, que demuestra mediante la relación que existe entre los autovalores y autovectores de una matriz positiva que, el radio espectral de una matriz positiva siempre tiene multiplicidad algebraica de 1, es decir, que existe . Por lo tanto, se puede concluir que $\rho (A)$ es un autovalor simple.
    
\item 

\end{itemize}