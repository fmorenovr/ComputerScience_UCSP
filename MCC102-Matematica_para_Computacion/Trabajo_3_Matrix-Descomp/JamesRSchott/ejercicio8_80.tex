Considera cadena homogénea de \textit{Markov} que tiene 3 estados y la matriz
    de transición de transición dado por:
    \[
        P =
            \begin{bmatrix}
                0.50 & 0.25 & 0 \\
                0.50 & 0.50 & 0.25\\
                0 & 0.25 & 0.75
            \end{bmatrix}
    \]
    \begin{itemize}
        \item Mostrar que \textit{P} es primitiva.

        Una matriz $mxm$ no negativa $A$ se dice que es primitiva si existe un $k$, tal que
        $A^k > 0$, esto es, $a_{ij} > 0$ $\forall$ $i$ $\&$ $j$. 
        Para $k = 2$:\\
        \[
            P =
                \begin{bmatrix}
                    0.375 & 0.25 & 0.0625 \\
                    0.50 & 0.4375 & 0.3125\\
                    0.125 & 0.3125 & 0.625
                \end{bmatrix}
        \]
        Ya que la matriz es P es no negativa, además existe un $k = 2$ tal que $P^* = P^2$: $p^*_{ij} > 0$ $\forall$ $i$ $\&$ $j$, por tanto la matriz $P$ es primitiva
        \item Determina la distribución de equilibrio, esto es, encuentra el vector
        $\pi$ tal que $\lim_{t \to \infty} p^{(t)} = \pi$\\\\
        En general:
        \[
            p^{(t)} = P^tp^{(0)} 
        \]
        Ya que $p^{(t)} = P^{(t)}p^{(0)}$, también puede ser expresado como:
        \[
            p^{(t)} = Pp^{(t-1)}
        \]
        El sistema apunta al punto de equilibrio en donde las proporsiones para los estados
        estan dados por las componentes de $\pi$ y esta proporsion no cambia con el tiempo.
        Esto es: $p^{(t)} = p^{(t-1)}$, por tanto:
        \[
                \pi = P\pi  \quad \dots \quad(1)
        \]
        Además ya que este punto representa la probabilidad de un conjunto, la suma de este es igual a 1.
        Esto es:
        \[
                \pi'1_m = 1 \quad \dots \quad(2)
        \]

        Hallando el autovalor $\pi$ a partir de la ecuación $(1)$:
        \[
            \pi = 
                \begin{bmatrix}
                    1 \\
                    2\\
                    2
                \end{bmatrix}
        \]
        De la ecuación $(2)$: (Las probabilidades deben sumar 1).
        Esto es:
        \[
            \pi = 
                \begin{bmatrix}
                    0.2 \\
                    0.4\\
                    0.4
                \end{bmatrix}
        \]
    \end{itemize}